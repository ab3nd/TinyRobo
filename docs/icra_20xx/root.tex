%%%%%%%%%%%%%%%%%%%%%%%%%%%%%%%%%%%%%%%%%%%%%%%%%%%%%%%%%%%%%%%%%%%%%%%%%%%%%%%%
%2345678901234567890123456789012345678901234567890123456789012345678901234567890
%        1         2         3         4         5         6         7         8

\documentclass[letterpaper, 10 pt, conference]{ieeeconf}  % Comment this line out if you need a4paper

%\documentclass[a4paper, 10pt, conference]{ieeeconf}      % Use this line for a4 paper

\IEEEoverridecommandlockouts                              % This command is only needed if 
                                                          % you want to use the \thanks command

\overrideIEEEmargins                                      % Needed to meet printer requirements.

\usepackage{xargs} 
\usepackage[colorinlistoftodos,prependcaption,textsize=tiny]{todonotes}
\newcommandx{\unsure}[2][1=]{\todo[linecolor=red,backgroundcolor=red!25,bordercolor=red,#1]{#2}}
\newcommandx{\change}[2][1=]{\todo[linecolor=blue,backgroundcolor=blue!25,bordercolor=blue,#1]{#2}}
\newcommandx{\info}[2][1=]{\todo[linecolor=OliveGreen,backgroundcolor=OliveGreen!25,bordercolor=OliveGreen,#1]{#2}}
\newcommandx{\improvement}[2][1=]{\todo[linecolor=purple,backgroundcolor=purple!25,bordercolor=purple,#1]{#2}}
\newcommandx{\thiswillnotshow}[2][1=]{\todo[disable,#1]{#2}}

\usepackage[pdf]{graphviz}
%In case you encounter the following error:
%Error 1010 The PDF file may be corrupt (unable to open PDF file) OR
%Error 1000 An error occurred while parsing a contents stream. Unable to analyze the PDF file.
%This is a known problem with pdfLaTeX conversion filter. The file cannot be opened with acrobat reader
%Please use one of the alternatives below to circumvent this error by uncommenting one or the other
%\pdfobjcompresslevel=0
%\pdfminorversion=4

% See the \addtolength command later in the file to balance the column lengths
% on the last page of the document

% The following packages can be found on http:\\www.ctan.org
%\usepackage{graphics} % for pdf, bitmapped graphics files
%\usepackage{epsfig} % for postscript graphics files
%\usepackage{mathptmx} % assumes new font selection scheme installed
%\usepackage{times} % assumes new font selection scheme installed
%\usepackage{amsmath} % assumes amsmath package installed
%\usepackage{amssymb}  % assumes amsmath package installed

\title{\LARGE \bf
Extension of TangentBug to Swarm Robotics
}


\author{Abraham Shultz% <-this % stops a space
%\thanks{*This work was not supported by any organization}% <-this % stops a space
%\thanks{$^{1}$Albert Author is with Faculty of Electrical Engineering, Mathematics and Computer Science,
%        University of Twente, 7500 AE Enschede, The Netherlands
%        {\tt\small albert.author@papercept.net}}%
%\thanks{$^{2}$Bernard D. Researcheris with the Department of Electrical Engineering, Wright State University,
%        Dayton, OH 45435, USA
%        {\tt\small b.d.researcher@ieee.org}}%
}


\begin{document}



\maketitle
\thispagestyle{empty}
\pagestyle{empty}


%%%%%%%%%%%%%%%%%%%%%%%%%%%%%%%%%%%%%%%%%%%%%%%%%%%%%%%%%%%%%%%%%%%%%%%%%%%%%%%%
\begin{abstract}

Swarm robots may have limited computation and communication. The need to complete assigned tasks under these constraints must be balanced with the need to have the behavior of the swarm be ledgible to the user, so that the user engages with the swarm an appropriate amount, not too little (neglect tolerance), nor too much (neglect benevolence). Bug algorithms present a possible mechanism for control of swarms with little or no communication which will be legible to end users. This work discusses an addition to TangentBug to perform path following and maximizing the density of robot clustering at a goal point. The limitations of bug algorithms for some common swarm tasks are also discussed. 

\end{abstract}


%%%%%%%%%%%%%%%%%%%%%%%%%%%%%%%%%%%%%%%%%%%%%%%%%%%%%%%%%%%%%%%%%%%%%%%%%%%%%%%%
\section{INTRODUCTION}

The purpose behind the development of swarm robots is to create swarms which can perform tasks for people. 
The control system that allows a human to operate a swarm can be based on a large array of possible metaphors and control methods \cite{kolling2016human}.
Whatever the underlying control system, the interface generally has to both accept commands from the user and provide feeback to the user about the state of at least some members of the swarm. 
In previous work in human-robot interfaces, a top-down map or overhead view for controlling robots has been proposed by multiple authors \cite{parasuraman2005flexible, bashyal2008human, Micire:2009:ANG:1731903.1731912,  diaz2017human}.

It is important to include the path specified by the user in the path planning for members of the swarm. 
By following the path laid out for it, the behavior of the swarm is made legible by the human user. 
Compliance with the requested path indicates to the user that the swarm is under their control, and allows them to assess the progress of the system towards the goal. 
If, instead, the system uses some other form of path planning, it will initially appear to not be doing what the user requested, even if the end state does eventually become what the user intended. 
What factors contribute to this legibility and how best to balance them with other operational constraints of the system is beyond the scope of this paper, but has been examined in the context of manipulation, navigation around people, and combined in mobile manipulation around people \cite{beetz2010generality, kruse2013human, dragan2015effects}.
The work on legible manipulation also distinguishes between predictable motion, which makes sense to observers based on a known goal, and legible motion, which allows observers to infer an unknown goal. 
Extension of legibility and the emotive content of motion to swarm robots has also been recently investigated, but not in relation to human control of the swarm \cite{Dietz:2017:HPS:3027063.3053220}.

The user input may also contain information that is not available to the swarm, but is available to the user. 
This information may be necessary for the successful completion of the task, and so is not to be discarded lightly. 
For example, the optimal path in terms of minimal travel distance may be blocked by some transient condition, especially in the case of disaster response, such as fire or flood, and so the user may direct the robots to take a longer route to the goal.
As such, the user's path selection can be viewed as an attempt to convey information about the desirability or utility of a given path to the robot, and so following the path given by the user is preferable to not following it.  
Discarding this information in favor of the shorter path could result in unit loss and mission failure.

\section{BUG ALGORITHMS}
There exist a family of algorithims, called ``bug algorithms'', which provide complete path planning in a priori unknown environments with seemingly minimal sensing, under reasonable bounds such as that the number of obstacles is finite and the goal is reachable. 
The bug family is large, and some of its members require sensing which may not be available in all conditions, such as location information or infinite-range distance sensing, but many do not. 

Generally, bug algorithms have two cases, the rules for motion in unobstructed space, and the rules for moving around obstacles. 
The bug algorithm transitions between the two states based on some set of conditions, which are normally based on local sensing and minimal memory for storage of environment information. 
In the initial bug algorithm paper, two bug algorithms are described, Bug1 and Bug2. 
Bug1 moves straight towards the goal if there is nothing in its way. 
When it detects an obstacle, the bug circumnavigates the obstacle until it returns to its starting point, called the ``hit point". 
During circumnavigation, the bug records the point of closest approach to the goal. 
After the circumnavigation is complete, the bug moves to the point on the obstacle perimeter closest to the goal and departs towards the goal \cite{lumelsky1987path}.
Bug2 circumnavigates the perimeter of the object as with Bug1, but Bug2 measures the angle to the goal. 
When the angle to the goal is the same as it was at the hit point, and the distance to the goal is less than it was at the hit point, Bug2 departs the obstacle perimeter and heads towards the goal. 

Bug1 and Bug2 are inefficient, in terms of the resulting path length compared to the optimal path in a given environment. By exending sensor range as in VisBug and TangentBug, the path can be greatly shortened \cite{lumelsky1990incorporating, kamon1996new}.
PointBug is competitive with TangentBug in some environments \cite{buniyamin2011simple}.
DistBug, RoverBug, and WedgeBug reduce the path length over Bug1 and Bug2, or have reduced sensing requirements \cite{kamon1997sensory, volpe2000enhanced}.
I-bug has possibly the most-reduced sensing, using only signal strength or a gradient emanating from the goal point \cite{taylor2009bug}.

Combining potential fields with bug algorithms yields controllers that can both move robots in formation and avoid obstacles \cite{weir2006potbug, mohamed2011improved}. 
Implementation of TangentBug on real robots indicates that there is some complexity swept under the rug by the use of theoretical point robots with holonomic drive.
Smooth controllers, whether derived from potential fields or not, allow bug-like algorithms to respect nonholonomic motion constraints, including articulated vehicles and car-style drive systems \cite{1461311, 6530864, nayl2015autonomous}. 
Allowing a robot to follow walls and move in straight lines requires some degree of continuous control supporting the logical functioning of the bug algorithm \cite{marin2010implementation}.

%The simplicity of bug algorithms allows them to be used to quickly find potential paths in a known map, rather than navigating an unknown map \cite{antich2009bug}. 
%The resulting plans are, as with the bug algorithms that generated them, correct, and the creation of multiple bug paths allows selection of the best one \cite{Bhanu2016multibug}.
%By combining bug algorithms with visual topological maps, robots can navigate without metric localization \cite{Maravall2015VisualEntropyBug}. \todo{This is a page of related work in a 6 page paper with no experiment yet. Cut WAY down.}

%In a pursuit-evasion task, a bug algorithm variant (PE-Bug) has the same ability to detect the evader as a system with a complete map and perfect navigation \cite{rajko2001pursuit}.

A performance comparison of 11 bug algorithms (Bug1, Bug2, Alg1, Alg2, DistBug, Class1, Rev1, Rev2, OneBug, LeaveBug, and TangentBug) is given in \cite{ng2007performance}.
TangentBug was found to achieve the shortest path in three of the four tested environments, but the geometry of the operating environment can impact the path length found by bug algorithms. 
Ng and Br{\"a}unl tested the bug algorithms without sensor noise. 
With noise, especially in odometry or localization, the bug algorithms can fail to perform correctly \cite{mcguire2018comparative}.

\subsection{Bugs with Moving Obstacles}

Bug algorithms are rarely used in swarm robotics, although their minimal sensing and memory requirements would seem to predispose them to use on inexpensive robots. 
The three primary problems with bug algorithms in practical situations are:

\begin{enumerate}
	\item Bug algorithms expect to control a point robot with holonomic drive. This is dealt with by hybrid controllers. 
	\item Bug algorithms expect unrealistic sensors (noiseless, infinite range). This is only true of some varieties of bug, and the impact of noise can be reduced. 
	\item Bug algorithms expect to control a single robot in a static environment. This is somewhat more problematic. 
\end{enumerate}

Ultimately, the presence of moving obstacles results in navigation becoming undecidable \todo{be damn sure this is correct} without knowledge of the future movement of the obstacles, as an obstacle can move to occupy the robot's goal. 
Unless it is known whether the obstacle will move off the goal in the future, it cannot be determined whether the goal is unreachable, or just not currently reachable. 

Simply applying a bug algorithm to the movement of each robot in the swarm may result in undesirable behavior in a number of ways. 
First, many bug algorithms rely on wall following to pass around obstacles. 
If the other obstacle is another robot, operating under the same algorithm, the robots may begin to circumnavigate each other. 
If the leave condition for the circumnavigation is never met, this behavior would persist indefinitely. 
Under the leave rule discussed for Bug1, if one of the robots is moving slightly faster than the other, the resulting path of the two robots would be a spiral moving in some direction in space. 
Since Bug1 uses return to the same location to detect circumnavigation, and the spiral would never return to the same location, the leave condition would never be satisfied.
For swarm robotics, there are expected to be multiple robots operating in an area, and for human-robot teams, urban deployments, and household robots, there may be humans in the area as well. 

By extending robots with the ability to detect the difference between other robots and obstacles, and the ability to assess the current motion of another robot, a controller that  \cite{lumelsky1997decentralized}. 
Disappointingly, a complete controller for non-communicating multiple robots is not feasible, although the incomplete controller by Lumelsky and Harinararyan has been determined to work well in situations that were not set up intentionally to confound it. 

If the robots recognize each other and are allowed to communicate, they can navigate around objects in opposite directions, thus reducing the amount of time required to find the leave point for even the inefficient Bug1 algorithm \cite{kandathil2018modified}.
Communicating robots moving using bug algorithms can also perform topological mapping of disconnected regions, so long as they can regularly communicate across the disconnect \cite{caccavale2017distributed}.

Distance Histogram Bug (DH-Bug) follows obstacles and has an additional mode for avoiding moving obstacles while still attempting to move towards the goal \cite{zhu2012new}. 
The creators of DH-Bug point out that in dynamic environments, the completeness of bug algorithms can be confounded by malicious actors. 
For example, fast-moving objects can attack the robot, overrunning its ability to navigate away from them. 

The Tangent-bug algorithm has been extended to handle moving obstacles, given a number of constraints \cite{tomita2009sensor}.
The main constraint that affects the use of this algorithm is that the obstacles are constrained to be moving at a velocity that is slower than that of the robot.
This constraint is required because if the robot is circumnavigating the obstacle, and the obstacle is moving faster than the robot, then in the time that the robot requires to circumnavigate the perimeter of the obstacle, the obstacle will have moved a distance greater than its own perimeter is long, and the circumnavigating robot will have moved with it.
As a consequence, the circumnavigating robot might not return to its own previous path and cross it, which is the condition that Tomita and Yamamoto use to determine that the robot should leave the obstacle. 

\section {PROPOSED ALGORITHM}
At first, this would appear to be a problem for swarm robots, because if the robots are the same, they will be moving at the same speed. 
If one moves in a straight line, and the other attempts to circumnavigate it, the circumnavigating robot will never cross its own path for the reason described above, and so never leave. 
However, if the robots are using the same bug algorithm, this trap will not be sprung, because each robot will attempt to circumnavigate the other. 
If they attempt to circumnavigate each other in opposite directions, they will spiral around each other, leading to at least one of the robots crossing its own previous path, and triggering the leave condition of the algorithm. 
If they attempt to circumnavigate each other in the same direction, they will come to a position side-by-side, as neither can outpace the other, but neither will pull away from the other because they are attempting to follow each other's perimeters. 
In this case, they are pointed towards the goal, because in the absence of an obstacle, Tomita and Yamamoto's modified Tangent-bug orients the robot towards the goal, and so before they encountered each other, the robots were oriented towards the goal.
The robots will then approach the goal, and one will arrive, while the other circumnavigates the first until it detects that it cannot arrive at the goal. 

Tomita and Yamamoto do not deal with the decidability of their modification of Tangent-bug because they constrain the goal point to be not within an obstacle, and so reachable by the robot. The original Tangent-bug will navigate the robot to the goal if it is reachable, or circumnavigate the obstacle, returning to its starting point, whereupon it detects that the point is unreachable \cite{kamon1998tangentbug}. Since Tomita and Yamamoto constrain the goal to not be within an obstacle, the original Tangent-bug will reach it. 

In the case of moving obstacles, if the goal is covered by an obstacle, the obstacle is either moving or not moving. 
If the obstacle is not moving, the arriving robot will circumnavigate the obstacle, return to the hit point, and stop, having detected that the goal cannot be reached.
If the obstacle is moving, the modified Tangent-bug will not return to its original hit point, which is either left behind or covered by the obstacle, but will eventually cross its own path, and leave the obstacle towards the goal.
If the obstacle is still covering the goal, this process will repeat until the object is not covering the goal anymore, and the robot will reach the goal. 

 \begin{figure*}
 	\centering
 	\digraph[scale=0.6]{TangentBugMod}{
 		
 		start -> obstacleToGoal;
 		obstacleToGoal -> aimToGoal [label="No"];
 		aimToGoal -> atGoal;
 		atGoal -> stop [label="Yes"]; 
 		atGoal -> obstacleToGoal [label="No"];
 		obstacleToGoal -> recordHit  [label="Yes"];
 		recordHit -> followEdge;
 		followEdge -> closerPoint;
 		closerPoint -> leaveObstacle [label="Yes"];
 		leaveObstacle -> obstacleToGoal;
 		closerPoint -> pathCross [label="No"];
 		pathCross -> returnToHit [label="No", color="blue"];
 		returnToHit -> followEdge [label="No", color="blue"];
 		returnToHit -> unreachable [label="Yes", color="blue"];
 		unreachable -> lastPoint [color="blue"];
 		lastPoint -> updateGoal  [label="No", color="blue"];
 		lastPoint -> stop  [label="Yes", color="blue"];
 		updateGoal -> obstacleToGoal [color="blue"];
 		pathCross -> obstacleToGoal3 [label="Yes"];		
		obstacleToGoal3 -> followEdge2 [label="Yes"];		
		obstacleToGoal3 -> aimToGoal[label="No"];		
		followEdge2 -> obstacleToGoal3;
		
 		obstacleToGoal [label="Obstacle in direction of goal?"];
 		aimToGoal [label="Orient towards and move to goal"];
 		atGoal [label="At goal?"];
 		leaveObstacle[label="Leave obstacle"];
 		recordHit [label="Record hit point"];
 		unreachable [label="Goal is unreachable", color="blue"];
 		updateGoal [label="Next point is new goal", color="blue"];
 		lastPoint [label="Is this the last point?", color="blue"];
 		followEdge [label="Follow obstacle edge"];
 		returnToHit [label="Returned to hit point?", color="blue"];
 		closerPoint [label="Closer to goal than hit point?"];
 		pathCross [label="Crossed own path?"];
 		obstacleToGoal3 [label="Obstacle in direction of goal?"];
 		followEdge2 [label="Follow obstacle edge"];
 	}
 	\caption{The proposed modifications (in blue) to the Tomita and Yamamoto Tangent-bug algorithm for user path following in cases with multiple robots, some of which may be treated as moving obstacles. This flow chart does not include the option for maximally dense packing at the goal described in the text. }
 \end{figure*}
 
In the case of swarm robots, some mechanism may be needed to determine that the goal is occupied, possibly by other swarm members, and to stop at a location as near the goal as possible. 
The stopping condition of the original Tangent-bug in the case where the goal is unreachable extends naturally to swarm robots. 

If a robot is the first to arrive at the goal, the goal is not occupied, so the robot occupies it and stops. 
If a robot is not first to arrive at the goal, the goal is occupied by a robot, which is stopped. 
The new arrival treats the stopped robot as an obstacle, circumnavigates it, returns to the original hit point and so detects that the goal is unreachable, and stops. 
If multiple new arrivals get to the stopped robot(s) at the same time, the conditions above hold, and so they eventually either cross their own paths while trying to circumnavigate another robot that is also circumnavigating an obstacle, and so leave and return (and so become later arrivals), or complete a circumnavigation and return to their own starting point and stop (becoming part of the obstacle).
Unfortunately, this method of handling late arrival will cause the robot cluster to grow in the direction which more robots are arriving from. 

If a maximally dense cluster of robots is desired, the unreachability check of Tangent-bug can be extended. 
Once the goal is determined to be unreachable, the robot performs another circumnavigation of the blocking obstacle to find the closest point to the goal, and returns to that point. 
That point has either been occupied by another robot, in which case the robot repeats this step, or occupies that point if it is free. 
Because every iteration of this step fills the point closest to the goal with a robot, the resulting cluster of robots is packed as close to the goal as possible. 


 
\section{EXPERIMENTAL EVALUATION}

Without user testing, this paper can't talk much about the legibility of the resulting paths. 

If I do a proof instead that e.g. adding path following doesn't break the completeness of the program, that may be of some interest, although it's a pretty trivial thing. 

Maybe handle which swarm algos are handlable by bugs and which are not, global/local sensing change, what about talkative bugs?

This is rapidly turning into a research project of its own. 


Hold off dealing with this until after PhD is complete.



\bibliography{../proposal/swarm.bib}
\bibliographystyle{IEEEtran}
\end{document}
