\documentclass[]{article}


\usepackage{xargs} 
\usepackage[colorinlistoftodos,prependcaption,textsize=tiny]{todonotes}
\newcommandx{\unsure}[2][1=]{\todo[linecolor=red,backgroundcolor=red!25,bordercolor=red,#1]{#2}}
\newcommandx{\change}[2][1=]{\todo[linecolor=blue,backgroundcolor=blue!25,bordercolor=blue,#1]{#2}}
\newcommandx{\info}[2][1=]{\todo[linecolor=OliveGreen,backgroundcolor=OliveGreen!25,bordercolor=OliveGreen,#1]{#2}}
\newcommandx{\improvement}[2][1=]{\todo[linecolor=purple,backgroundcolor=purple!25,bordercolor=purple,#1]{#2}}
\newcommandx{\thiswillnotshow}[2][1=]{\todo[disable,#1]{#2}}

\usepackage{graphicx}
\usepackage{auto-pst-pdf}
\usepackage{graphviz}
\usepackage{microtype}

%opening
\title{TinyRobo: An Affordable Tabletop Swarm Platform}
\author{Abraham Shultz}

\begin{document}

\maketitle

\begin{abstract}

\end{abstract}

\section{Motivation}

\section{Simulation and its Discontents}

Simulation of robots is clearly the lowest cost option for performing tests with large collections of robots. 
Once the simulation environment is created, adding additional robots does not incur any financial cost. 
Larger simulations may of course require more computer time, but by running at less than realtime, or improving the hardware the simulation runs on, this problem can be ameliorated. 

However, simulation brings with it a number of problems. 
Imperfections in reality may not be exposed by simulation.
As a teenager, the author had a collection of small robots that operated continuiously on an enclosed tabletop. 
The solar power supply for these robots was the SolarEngine, which charges a capacitor from a solar cell and then drains the capacitor through a motor when the capacitor is fully charged \cite{hasslacher1995living}.
One subset of the robots had no sensors and only one motor. 
These machines would tip when they collided with something, and because of the tipping, move in a new direction. 
The author observed that all of the tipping robots would gradually collect in one corner of the enclosed tabletop. 
The emergent gathering behavior was a result of a slight slant in the surface of the table. 
Motions that moved uphill consumed more power than motions that moved downhill, and so were shorter in duration. 
As a result, the average of the robots' motion was towards the lowest point on the table. 

While a simulation could be developed to include factors such as variable tilt in the surface the robots move over (and associated details, such as variable surface roughness and flexibility), doing so increases the complexity of the simulation significantly. 
Worse, in some situations, such complexities are not optional, and are difficult to capture correctly. 
The use of gentic algorithims is particularly prone to these problems, as the genetic approach frequently develops ``solutions'' which exploit deficiencies in the simulation software that do not obtain in reality. 
In one system, the fitness function for evolved simulated creature designs awarded high fitness for moving a long distance. 
This resulted in creatures that were very tall, and fell over as soon as the simulation started, resulting in a large and quick horizontal movement \cite{brooks2000artificial, sims1994evolving}. 
A bug in the conservation of momentum in the same simulation resulted in creatures that moved by hitting themselves on the back. 
As Sims notes, ``Any bugs that allow energy leaks from non-conservation, or even round-off errors, will inevitably be discovered and exploited by the
evolving creatures. Although this can be a lazy and often amusing approach for debugging a physical modeling system, it is not necessarily the most practical." \cite{sims1994evolving}.

\section{Related Work}

\subsection{Overview of Previous Swarm Hardware}

Swarm robots are generally small. 
The reason to keep swarm robots small is at least three-fold. 
First, larger robots consume more materials per unit, and so cost more money.
As a result, for a given number of swarm units, larger robots will result in a higher cost swarm. 
Second, each robot requires some amount of space to move around in. 
To keep the ratio of free space to robots constant, the area of space used by the robots grows as the robots do. 
If the ratio isn't kept constant, the robots will crowd each other, and so large robots will require either a very large space, or become overly crowded.
Finally, if the application domain of the robots is outside of research, the domain will dictate the size of the robots. 
For example, a swarm designed to monitor infrastructure such as a bridge or building should be small enough to enter all the areas that require monitoring. 
A search and rescue swarm might be composed of a variety of scales of robots, with smaller robots infiltrating a damaged building to collect telemetry that will guide robots large enough to move damaged material in the building. 

The challenge of construction of swarm robot hardware, then, is to put all of the same parts as non-swarm mobile robots: a mobility platform, a power supply, a processor, some sensors, and a communication system, into a small package.
Many impressive designs for small swarm robot platforms have been proposed and constructed as part of research in swarm robotics. 
However, most of these platforms are no longer easily commercially available, or never were. 

The robots used in most swarm work are of a sufficently small size that many of them can fit in a room. In addition to budgetary constraints, interaction with an environment built for humans places and upper bound the scale of the individual swarm members. 
The lower bound on swarm robots is generally dictated by fabrication technology, with smaller robots becoming increasingly difficult to assemble. 
As a result of these bounds, swarm robots are mostly between 1cm$^3$ and 0.3m$^3$. 
This scale range divides fairly evenly into robots that can operate in large swarms on a table, and those that can operate in swarms within a room, albeit possibly a large room. 

\subsubsection{Tabletop Swarms}

At the low end, in terms of size, the I-SWARM Project was intended to create a 2x2x1mm robot that moved by stick-slip locomotion actuated by piezo levers \cite{seyfried2005swarm}. 
Over the course of the project from 2004-2008, the hardware was developed and used in research, but was not converted to a commercial product.
Other techniques have been developed to use magnetic fields to apply force to small magnetic objects, resulting in controlled motion of the objects \cite{floyd2008untethered, pelrine2012diamagnetically}.
These systems are not amenable to decentralized control, because the moving components are not themselves robots. 
The moving parts are more accurately viewed as manipulators, with the instrumented environment, any sensors for feedback from that environment, and the manipulators themselves comprising a single robot. 

Alice, by Caprari et al. \cite{caprari1998autonomous} combined a PIC16F84 processor, motors, RF and IR networking, and enough battery power for 10 hours of autonomy into a robot measuring under 2.5cm$^3$. 
The processor used in Alice is relatively underpowered compared to modern processors at the same price point and power consumption. 
Alice robots are no longer available for purchase. 
The AmIR robot was similar to Alice in size and capability, but with a more modern processor \cite{arvin2009development}.
There is no evidence that AmIR was ever widely available.

The Jasmine swarm robots were possibly the closest thing to a commercially-available successor to Alice  \cite{kernbach2011swarmrobot}.
Jasmine measured 26x26x20mm, and included an ATMega processor, IR close range communication and obstacle detection, two motor skid steering, and li-po batteries.
Unfortunately, Jasmine units cost about 100 Euro (\$111 USD) each when they were available, and they are no longer available for purchase. 
The plans and information required to reproduce Jasmine units are available for free at Swarmrobot.org.
Assembling a Jasmine robot is not beyond the reach of competent electronics hobbyists, but it does require some unusual build processes, such as grinding down the cases of certain electronic parts and filling holes in the PCB with solder to prevent light leaks. 
The chassis of Jasmine is also a custom mechanical assembly, rather than a commercially available product. 

InsBot was a small robot, measuring 41mm x 30mm x 19mm, that was designed to interact with cockroaches \cite{colot2004insbot}.
It used two processors, one to run higher level behaviors and one to interface with a suite of sensors that included 12 IR sensors and a linear camera. 
InsBots were never commercially available, and each required approximately 6 hours of work to assemble by hand. 
However, the construction process appears to have been relatively straightforward. 

Combining a relatively modern microprocessor with a DSP allowed the creation of a system that could perform vision processing tasks on the robot \cite{haverinen2005miniature}. 
Again, it is not a platform that other researchers could buy.

Even when they are commercially available, most existing swarm robots are too expensive to build a large swarm, with the exception of the Harvard Kilobots  \cite{rubenstein2014kilobot}. 
Kilobots contain about \$15 worth of parts, but a 10-pack sells for 1100 Swiss francs, or about \$112 (US) per robot. 
The Kilobots are intended for research in a highly homogeneous environment, with all robots executing the same program. 
As a result, they are designed to be programmed in parallel using an IR interface. 
For small groups, individual Kilobots can be programmed differently, but any attempt to give each of a very large collection of robots an unique program will take a long time. 
The Kilobots also move by stick-slip motion, and so must operate on a smooth surface, such as a whiteboard. 

The Epuck from EFPL is approximately 800 Swiss francs (\$810 USD) per unit, so the cost of maintaining a large swarm can become daunting quickly. 
The high price of the Epuck is a result of its extensive suite of sensors, including a camera and 360$^{\circ}$ IR range sensor and communication system. 
Epucks also require a fairly smooth operating surface.
The Epuck uses differential drive, and allows the front or back edge of the robot to serve as a skid. 
Due to the relatively sharp corner of the lower edges of the robot, the Epuck can become stuck on 2-3mm high obstacles. 

The r-one research robot is cheaper than the Epuck, at approximately \$220 USD per unit \cite{mclurkin2013low}. 
The developers of the r-one position it as a more-featureful and less expensive alternative to the Epuck (\$810, cannot sense neighbors without additional hardware), Parallax's Scribbler (\$198, minimal sensors), the iRobot Create (\$220, requires additional hardware to be programmable), the K-team Khepera III (\$2000), and the Pololu 3pi (\$99, minimal sensors). The main advantage in sensing that the r-one has over these other platforms is neighbor sensors and ground truth position sensing, both of which are implemented on the r-one using infrared.
The design of the r-one is open source, but it does not appear to be commercially available as of this writing.   

One way to reduce the cost of swarm robots is to use commercial, off-the-shelf (COTS) hardware in the construction of the robot. 
Reusing existing hardware leverages the economies of scale that reduce the price of commercial hardware, as well as eliminating the need to design or build the COTS parts. 
Use of COTS parts in research robotics has led to at least two platforms referred to as COTSBots \cite{bergbreiter2003cotsbots, soule2011cotsbots}.
The first COTSBots used mote hardware for the communications link and sensing, plus a motor control add-on board \cite{bergbreiter2003cotsbots}. 
The mobility platform is a modified toy, in particular, a specific brand of high-quality micro RC car.
At the time of this writing, the car used in COTSBots is moderately expensive for a toy car, although quite cheap for a research robot, costing a little over \$100USD per unit. 
COTSBots use TinyOS, a modular and event-driven framework for developing node software \cite{levis2005tinyos}. 
TinyOS is written in a dialect of C called nesC. The motor and mote boards communicate using a messaging layer. 
The motor driver board is not commercially available, but can be custom-built by board fabrication companies, without the researcher having to assemble it by hand. 

\subsubsection{Room-sized Swarms}

One potential problem with extremely small swarms is that while the robots may scale down, the scale of obstacles they have to traverse may not scale with them. 
As previously mentioned, the Kilobots require a smooth surface, and the Epuck can be stopped by obstacles no more than a few millimeters tall. 
This sort of vulnerability prevents the smaller, tabletop swarm robots from operating well in human-scaled environments. 
In order to overcome this problem, larger swarm robots can be constructed.
 
The MarXbot swarm platform is capable of operating in unstructured human environments. 
MarXBots can also use their grippers to link themselves together and perform operations that an individual robot could not perform, such as bridging a gap larger than a single robot \cite{bonani2010marxbot}. 
The size and complexity of the MarXbots, as well as their powerful computer, renders the individual robots quite expensive. 

Swarmanoid extends the interlinking mechanism of MarXbot to a heterogeneous swarm with three different types of robots \cite{dorigo2013swarmanoid}.
The ``foot'' robots are MarXbots, and provide ground motion for ``hand'' robots. 
``Hand'' robots have grippers to manipulate objects, and can also climb.
The ``hand'' robots have an attachment point like the MarXbots, and so can be carried by ``foot'' robots. 
Flying ``eye'' robots provide overviews of the work area and networking.  

In order to reduce costs, another platform called COTSBots was developed \cite{soule2011cotsbots}.  
Instead of sensor motes on micro-scale RC cars, the newer COTSBots platform is composed of a laptop for processing and a modified RC car, tank, or similar toy as a mobility platform.
In order to interface between the laptop and motor drivers, a second microcontroller board, such as an Arduino or Phidget interface may be used. 
Due to the diversity of possible combinations of hardware that can be assembled into this configuration, it is still a very viable platform. 
However, the minimum size of this style of COTSBot is the size of a laptop, which is in turn dictated largely by the minimum size of a useful keyboard. 
The large size of these COTSBots demands a very large space if the density of robots in a large swarm is to be kept low. 
Additionally, each laptop has a screen, keyboard, and so forth that are not useful while the robot is operating. 
All of these parts add to the overall cost of the swarm. 

Beyond the scale of rooms, swarm research has been done with Amigobots and Roombas \cite{guo2007bio, tammet2008rfid}.
In theory, swarm research could be performed by robots of any size, but financial limitations would place it out of the reach of most academic organizations. 

\section{Hardware}

The previously described swarm robot platforms tend to fall into one of two groups, from a hardware perspective. 
The first group uses microcontrollers and very limited onboard computation, but is small and relatively cheap.
This group includes Alice, Jasmine, AmIR, and the other tabletop systems. 
Due to their limited computation, these systems do not generally support complex algorithms such as vision processing. 
The second group use more powerful computers, but at a significant cost in weight, power consumption, and financial outlay.

The robot described in this work is intended to occupy a theoretical ``sweet spot'' at the high end of the tabletop swarms or the low end of the room-scale swarms, depending on how large of a mobility platform is used. 
As a result, if it is configured for tabletop operation, the system can be used with a minimum of available space. 
If, on the other hand, it is configured for room-scale operations, the system can be tested in natural or naturalistic human environments. 

The robot swarm developed for this work consists of a hardware module for controlling two motors of a toy, such as a small RC car, for mobility. 
The reasons for choosing this hardware design are explained in more detail below, but the overall intent is to have an inexpensive platform available for swarm research, without having to rely on any particular group of swarm robotics researchers starting and maintaining a side business supporting and selling robots.
Duplication of software and other digital artifacts is trivial, so constructing a duplicate of the hardware becomes the primary difficulty. 
The use of toys for the mechanical components of the robots is intended to reduce the difficulty of constructing the hardware. 
If researchers are not to be expected to become entrepreneurs, they should also not be expected to become expert machine tool operators.
The hardware resulting from this work will be designed so that it can be duplicated by a researcher using common tools, and possessed of no more than hobby-level familiarity with electronic hardware.

In order to be both heterogeneous and inexpensive, the robots used for this work will be constructed by developing a Commercial Off-The-Shelf (COTS) modular control hardware platform that can be attached to children's toys. 
Modified toys are an adequate substitute for custom mechanical assemblies, and permit easy experimentation with heterogeneous swarms. 
The use of children's toys as mobility platforms may also avoid the sensitivity to the work surface exhibited by the Kilobots and, to a lesser extent, the Epucks.
The TinyRobo controller module was designed to be used as a replacement for the control electronics of children's toys, similar to the Spider-Bots developed by Laird, Price, and Raptis or Bergbreiter's COTSBots \cite{lairdspider, bergbreiter2003cotsbots}.
However, unlike the Spider-Bots and COTSBots, TinyRobo does not specify a particular toy chassis to use for mobility. 
Most children's toys use either one motor with a mechanical linkage to cause the toy to turn when the motor is reversed, or two motors.
Two-motor toys frequently use either differential steering or have one motor provide drive power and the other provide steering. 
All of these toys can be controlled by the hardware described in this work. 

TinyRobos are intended to be heterogeneous, partly because of the advantages of heterogeneity in a swarm, and partly because toy supplies are unreliable.
While toys in the general case are expected to remain available, a particular line of toys might be discontinued or a modified version released. 
The software framework in development to support TinyRobos is based on ROS, and so will allow modular replacement of the control algorithms used to convert desired motion of the robot into drive signals for the motors. 

The processor of the TinyRobo controller is an ESP-8266 wifi module.
The ESP-8266 costs approximately \$3-5, and contains both a wireless interface and a micro controller that can be programmed from a variety of programming environments and languages, including Lua and the Arduino variant of C/C++. \change{more info on processor core (clock speed, RAM, ROM size)}  
The ESP-8266 is available in several form factors, each designated by a different suffix. 
The version selected is the ESP-8266-03, which offers more GPIO pins than most other versions, and includes an internal antenna.

The TinyRobo control module also provides connections for a 3.7V lithium-ion battery pack, as well as charge control circuitry for the battery. 
The charge controller allows the TinyRobo to be charged from the same USB connection that is used to change the programming of the ESP-8266. 
Reset and entry into programming mode is controlled by a separate USB-to-serial adapter board, the Sparkfun BOB-11736.
Moving this functionality to the adapter board reduces the size and cost of the TinyRobo control module. 

Children's toys normally use inexpensive brushed DC motors in their construction. 
These motors have not been the subject of extensive study, as they are commodity parts. 
However, it is useful to quantify their behavior to some extent, to determine which kinds of toys can be used with the TinyRobo controller. 

Two common types of motors found in children's toys are the RE and FA series of motors produced by Mabuchi Motor, or imitations of these motors produced by other companies. 
These motors use simple metal brushes and are constructed to be inexpensive, rather than precise. 
The intended voltage range of the motors varies with different winding types, but according to datasheets available from Mabuchi Motor, the voltage ranges and current draws for motors in this range are as shown in table \ref{tab:properBrandedMotors}.

\begin{table}
	\begin{tabular}{l|l|l|l|l}
	Model & Voltage & No Load Current & Max Efficiency & Stall Current\\
	RE-140RA-2270 & 1.5-3 & 0.21 & 0.66 & 2.1 \\
	RE-140RA-18100 & 1.5-3 & 0.13 & 0.37 & 1.07 \\
	RE-140RA-12240 & 3-6 & 0.05 & 0.14 & 0.39 \\
	FA-130RA-2270 & 1.5-3 & 0.2 & 0.66 & 2.2\\
	FA-130RA-18100 & 1.5-3 & 0.15 & 0.56 & 2.1\\
	FA-130RA-14150 & 1.5-4.5 & 0.11 & 0.31 & 0.9\\
	\end{tabular}
	\caption{Current draw for Mabuchi-branded motors.}
	\label{tab:properBrandedMotors}
\end{table}


\change{More detail on the motors used in children's toys. Quantify the behavior of a bunch of motors.}

For tables \ref{tab:FA-series}, \ref{tab:RE-series}, and \ref{tab:coreless}, the motor current at stall was estimated by applying a regulated 3.3v supply to the motor and measuring the voltage drop across the motor. 

This was done to avoid damage to the motors. 
In toys, the current to the motors is frequently limited by the available current from the batteries, and stalling the motor is prevented by simple slip clutches or play in the motor drive train. 

\change{Add photos of motors, with cm scale for visual clairity}
\change{Two main types of motors, ~2cm$^3$ brushed DC and >1cm$^3$ coreless motors}

\begin{table}
	\begin{tabular}{l|l|l}
	Motor number & No Load Current & Stall Current (estimated)\\
	2 & 0.08 &
	3 & 0.35 &
	4 & 0.02 &
	5 & 0.11 &
	6 & 0.20 &
	7 & 0.02 &
	8 & 0.33 &
	\end{tabular}
	\caption{No load and stall current for Mabuchi FA series and similar motors. Measurements were performed at 3.3V supply voltage. Motor 1 was defective.}
	\label{tab:FA-series}
\end{table}


\begin{table}
	\begin{tabular}{l|l|l}
	Motor number & No Load Current & Stall Current (estimated)\\
	\end{tabular}
	\caption{No load and stall current for Mabuchi RE series and similar motors. Measurements were performed at 3V supply voltage.}
	\label{tab:RE-series}
\end{table}

\begin{table}
	\begin{tabular}{l|l|l}
	Motor number & No Load Current & Stall Current (estimated)\\
	\end{tabular}
	\caption{No load and stall current for coreless DC micromotors. Measurements were performed at 3V supply voltage.}
	\label{tab:coreless}
\end{table}

The motor controller in the TinyRobo boards is the DRV8830 IC by Texas Instruments. 
The DRV8830 provides 1A of drive current, and is controlled by the ESP-8266 over an I2C serial interface. 
Experimental tests with 8 different toys indicate that small toys draw well under 1A while moving freely, and peak around 2A when the motors are stalled. 
The tested toys include 3 insect-styled walkers, 3 wheeled vehicles (2 differential drive, 1 Ackerman steering), 1 toy helicopter, and 1 toy quadcopter.
The DRV8830 provides overcurrent limiting, so a stall condition or short circuit of the motor leads will disable the motor drive, but not damage the DRV8830. 

\subsection{Potential for Expansion}

The current design for TinyRobo does not include sensors as a cost-saving decision. 
However, the communication between the ESP-8266 and the motor drivers uses the industry standard I2C bus serial interface. 
Due to the non-proprietary nature of this interface standard, it has been widely adopted, and many sensors are available to connect to an I2C bus. 
For example, Vishay Semiconductor makes the VCNL3020, which is an infrared proximity sensor with a 20mm range. 
If greater range is required, The ST Microelectronics VL53L0X Time-of-flight (ToF) laser ranger and gesture sensor provides a 2M range and 1D gesture sensing in a 4.4mm x 2.4mm package. 
As of this writing, the VCNL3020 is \$3.44 and the VL53L0X costs \$6.28 in single quantities.
These prices are reduced significantly when buying components in bulk, but because they increase the cost, size, and power draw of the hardware, they have not yet been integrated with the TinyRobo platform. 
Numerous multichannel ADC ICs with I2C interfaces are also available, which permits the addition of analog sensors to the TinyRobo platform. 

\change{Use a URL shortener to point to latest revision?}

\section {Firmware}

Adaptation to underlying toy hardware. 

Should have software calibration \& hardware reset option (e.g. a IoT-style connection dialog that also includes a "motor calibrate" and "clear calibration" options.)

Motor speed translation from twist message should be brought into ROS, robots don't get a Twist. Board can't tell how motor speeds affect position, ROS/Camera system can.

\section{Software}

The individual robots being developed for this research do not have very powerful processors. 
The majority of the processing will be performed on a host computer running the ROS software framework. 
Each robot's processor will mostly be concerned with interfacing with the robot's sensors, if any, and controlling the motors of the robot. 
The structure of the software framework is such that as available processing power on each individual robot increases, more of the processing can be handled locally, without changing the overall behavior of the system.

In order to locate the robots within the experiment area, an overhead camera will be used to detect machine-readable symbols on each robot. 
The symbols provide location, heading, and a unique identifier for each robot. 
Because the system has an omniscient-view camera, other objects in the robot arena can also be tracked. 
For example, obstacles can be created by drawing lines on the floor of the robot enclosure. 
Different colors could represent different types of obstacles, or qualities of the obstacles that are relevant to the software under test.

Since the robots will all be reporting to a central server, and the central server will also be receiving the video from the overhead camera, it may appear that this is a highly centralized system. 
However, the central computer will provide a framework for implementing a decentralized control scheme on the individual robots. 
Rather than controlling each robot, the central computer will maintain a separate process for each robot in the swarm. 
Each of these robot processes will only have access to the information that would be available to that robot, and so will act as a local control program for the robot, but will have the full processing resources of the host computer. 
As a result, the individual robots can be small, lightweight, and consume relatively little electrical power, but the system as a whole will endow them with significant computing power. 
As a result, the overall system will behave as though the individual robots each had a powerful computer. 
As more powerful and lower power consumption processors become available, more of the processing can be moved from the virtualized robot processors and onto the actual robots, enabling a smooth transition from a simulated decentralized system to a real decentralized system. 

Similarly, it should be stressed that while the central computer can localize the robots, both relative to each other and by absolute position within the arena, this information may be withheld from the individual robots, or given to them if required. 
The code virtually operating on the robot may be neither aware of its own position in the world, or the location of other robots, if the experiment calls for such a lack of information. 
However, the central computer can use the location information to create ``virtual sensors'' for each robot. 
The sensor precepts from virtual sensors would be simulated, but their magnitude or direction may be based on the location of a real robot relative to a real object in the experiment area. 
For example, collision avoidance between two could be implemented by a virtual sensor on each robot that indicates the direction and heading of the nearest robot. 
Since the central computer knows the location of each robot, the relevant information can be sent to each robot's control process as if it were coming from a sensor on the robot. 
The virtual sensors can also be configured to emulate error conditions such as noisy sensors, failed sensors, degraded localization, and so forth.
Virtual parameter tweaking will allow fine-grained testing of the behavior of algorithms under imperfect conditions, and the response of human users to unreliability in the swarm. 

If the robots are required to communicate directly with each other, the communication will pass through a virtual network. 
From the point of view of the robots, messages sent into the virtual network will be delivered to other robots as if the messages were sent directly from one robot to another. 
However, all the communication will be taking place between processes running on the central computer.
By changing how the messages are delivered by the central system, we can implement full connectivity, range-limited mesh networking, directional beacons, or other forms of networking. 
We can also vary the reliability of the network, by dropping some messages or reducing parameters based on elements of the virtual environments. 
For example, signals that pass through a virtual wall may have a reduced virtual signal strength and range, or may not arrive. 
Having communication mediated by software on the host will allow for simple experimentation with variable network bandwidth and reliability. 

%Graphic of the software environment as a whole
\begin{figure}[h]
	\centering
	\digraph[scale=0.6]{Framework}{
	
	graph[nodesep=0.5];
	
	subgraph clusterRobot1 {
		motor[shape=box; label="Motor Driver"];
		robotCode[label=<Robot <br/> Firmware>];
		robotCode -> motor;
		label="Robot 1";
		shape=box;
	}
	
	subgraph clusterRobot2 {
		motor2[shape=box; label="Motor Driver"];
		robotCode2[label=<Robot <br/> Firmware>];
		robotCode2 -> motor2;
		label="Robot 2";
		shape=box;
	}
	
	subgraph clusterRobot3 {
		motor3[shape=box; label="Motor Driver"];
		robotCode3[label=<Robot <br/> Firmware>];
		robotCode3 -> motor3;
		label="Robot 3";
		shape=box;
	}
	
	subgraph clusterRobotN {
		motorN[shape=box; label="Motor Driver"];
		robotCodeN[label=<Robot <br/> Firmware>];
		robotCodeN -> motorN;
		label="Robot N";
		shape=box;
	}
	
	subgraph clusterComp {
		concentrate=true;
		label="Control Computer";
		{rank=source;
			vrSense [label="Virtual Sensors"];
			vrNet [label="Virtual Network"];
			worldModel [label="World Model"];
			worldModel -> vrSense;
			worldModel -> vrNet;			
		}
		rp1 [label=<Robot <br/> Process 1>];
		rp2 [label=<Robot <br/> Process 2>];
		rp3 [label=<Robot <br/> Process 3>];
		rpN [label=<Robot <br/> Process N>];
		vrNet -> {rp1, rp2, rp3, rpN} [dir="both"];
		vrSense -> {rp1, rp2, rp3, rpN};	
	}
	
	camera[label=<Overhead<br/>Camera>;shape=box;]
	camera->worldModel;
	
	rp1 -> robotCode [label="WiFi", dir="both"];
	rp2 -> robotCode2 [label="WiFi", dir="both"];
	rp3 -> robotCode3 [label="WiFi", dir="both"];
	rpN -> robotCodeN [label="WiFi", dir="both"];
	}
	\caption{Overview of the planned framework. Rectangular nodes are hardware, oval nodes are software.}
\end{figure}

Not bound to TinyRobo hardware, so may be useful to other users of tabletop swarms. 

Automatic setup of robots on the network (need to figure out a faster scan method)

\bibliography{../proposal/swarm.bib}
\bibliographystyle{apalike}

\end{document}
