\chapter{Related Work}

\section{Overview of Previous Swarm Hardware}

Swarm robots are generally small. 
The reason to keep swarm robots small is related to both the cost of making them and the cost of using them. 
Larger robots consume more materials per unit, and so cost more money.
As a result, for a given number of swarm units, larger robots will result in a higher cost swarm. 
Also, each robot requires some amount of space to move around in. 
To keep the ratio of free space to robots constant, the area of space used by the robots grows as the robots do. 
If the ratio isn't kept constant, the robots will crowd each other, and so large robots will require either a very large space, or become overly crowded.
Finally, larger robots are more cumbersome to deal with. 
They require larger storage areas, possibly teamwork to lift or repair, and so forth. 
All of these efforts are also multiplied by the number of robots in the swarm. 

The robots used in most swarm work are of a sufficiently small size that many of them can fit in a room. 
In addition to budgetary constraints, interaction with an environment built for humans places an upper bound the scale of the individual swarm members. 
For example, typical indoor doorways are around thirty inches wide, so a robot would have to be less than thirty inches wide to fit through them. 
The lower bound on swarm robots is generally dictated by fabrication technology, with smaller robots becoming increasingly difficult to assemble. 
As a result of these bounds, swarm robots are mostly between 1cm$^3$ and 0.3m$^3$. 
This scale range divides fairly evenly into robots that can operate in large swarms on a table, and those that can operate in swarms within a room, albeit possibly a large room. 
The challenge of construction of swarm robot hardware, then, is to put all of the same parts as non-swarm mobile robots: a mobility platform, a power supply, a processor, some sensors, and a communication system, into a small package.
Many impressive designs for small swarm robot platforms have been proposed and constructed as part of research in swarm robotics. 
However, most of these platforms are no longer easily commercially available, or never were. 

\subsection{Tabletop Swarms}

At the low end, in terms of size, the I-SWARM Project was intended to create a 2x2x1mm robot that moved by stick-slip locomotion actuated by piezo levers \citep{seyfried2005swarm}. 
Over the course of the project from 2004-2008, the hardware was developed and used in research, but was not converted to a commercial product.
Other techniques have been developed to use magnetic fields to apply force to small magnetic objects, resulting in controlled motion of the objects \citep{floyd2008untethered, pelrine2012diamagnetically}.
These systems are not amenable to decentralized control, because the moving components are not themselves robots. 
The moving parts are more accurately viewed as manipulators, with the instrumented environment, any sensors for feedback from that environment, and the manipulators themselves comprising a single robot. 

Alice, by Caprari \emph{et al}. \citep{caprari1998autonomous} combined a PIC16F84 processor, motors, RF and IR networking, and enough battery power for 10 hours of autonomy into a robot measuring under 2.5cm$^3$. 
The processor used in Alice is relatively underpowered compared to modern processors at the same price point and power consumption. 
Alice robots are no longer available for purchase. 
The AmIR robot was similar to Alice in size and capability, but with a more modern processor \citep{arvin2009development}.
There is no evidence that AmIR was ever widely available.

The Jasmine swarm robots were possibly the closest thing to a commercially-available successor to Alice  \citep{kernbach2011swarmrobot}.
Jasmine measured 26x26x20mm, and included an ATMega processor, IR close range communication and obstacle detection, two motor skid steering, and li-po batteries.
Unfortunately, Jasmine units cost about 100 Euro (\$111 USD) each when they were available, and they are no longer available for purchase. 
The plans and information required to reproduce Jasmine units are available for free at Swarmrobot.org.
Assembling a Jasmine robot is not beyond the reach of competent electronics hobbyists, but it does require some unusual build processes, such as grinding down the cases of certain electronic parts and filling holes in the PCB with solder to prevent light leaks. 
The chassis of Jasmine is also a custom mechanical assembly, rather than a commercially available product. 

InsBot was a small robot, measuring 41mm x 30mm x 19mm, that was designed to interact with cockroaches \citep{colot2004insbot}.
It used two processors, one to run higher level behaviors and one to interface with a suite of sensors that included 12 IR sensors and a linear camera. 
InsBots were never commercially available, and each required approximately 6 hours of work to assemble by hand. 
However, the construction process appears to have been relatively straightforward. 

Even when they are commercially available, most existing swarm robots are too expensive to build a large swarm.
The E-puck from EFPL is approximately 800 Swiss francs (\$810 USD) per unit, so the cost of maintaining a large swarm can become daunting quickly. 
The high price of the E-puck is a result of its extensive suite of sensors, including a camera and 360$^{\circ}$ IR range sensor and communication system. 
E-pucks also require a fairly smooth operating surface.
The E-puck uses differential drive, and allows the front or back edge of the robot to serve as a skid. 
Due to the relatively sharp corner of the lower edges of the robot, the E-puck can become stuck on 2-3mm high obstacles. 

The r-one research robot is cheaper than the E-puck, at approximately \$220 USD per unit \citep{mclurkin2013low}. 
The developers of the r-one position it as a more-featureful and less expensive alternative to the E-puck (\$810, cannot sense neighbors without additional hardware), Parallax's Scribbler (\$198, minimal sensors), the iRobot Create (\$220, requires additional hardware to be programmable), the K-team Khepera III (\$2000), and the Pololu 3pi (\$99, minimal sensors). 
The main advantage in sensing that the r-one has over these other platforms is neighbor sensors and ground truth position sensing, both of which are implemented on the r-one using infrared.
The design of the r-one is open source, but it does not appear to be commercially available as of this writing.   

The Harvard Kilobots are a more recent entry to inexpensive swarms, and have been produced in large quantities \citep{rubenstein2014kilobot}. 
Kilobots contain about \$15 worth of parts, but a 10-pack sells for 1100 Swiss francs, or about \$112 (US) per robot. 
The Kilobots are intended for research in a highly homogeneous environment, with most or all of the robots executing the same program. 
As a result, they are designed to be programmed in parallel using an IR interface. 
For small groups, individual Kilobots can be programmed differently, but any attempt to give each of a very large collection of robots an unique program will take a long time. 
The Kilobots also move by stick-slip motion, and so must operate on a smooth surface, such as a whiteboard. 

\todo{add Colias: An Autonomous Micro Robot for Swarm Robotic Applications, paper has table of robot costs}

One way to reduce the cost of swarm robots is to use commercial, off-the-shelf (COTS) hardware in the construction of the robot. 
Reusing existing hardware leverages the economies of scale that reduce the price of commercial hardware, as well as eliminating the need to design or build the COTS parts. 
Use of COTS parts in research robotics has led to at least two platforms referred to as COTSBots \citep{bergbreiter2003cotsbots, soule2011cotsbots}.
The first COTSBots used mote hardware for the communications link and sensing, plus a motor control add-on board \citep{bergbreiter2003cotsbots}. 
The mobility platform is a modified toy, in particular, a specific brand of high-quality micro RC car.
At the time of this writing, the car used in COTSBots is moderately expensive for a toy car, although quite cheap for a research robot, costing a little over \$100USD per unit. 
COTSBots use TinyOS, a modular and event-driven framework for developing node software \citep{levis2005tinyos}. 
TinyOS is written in a dialect of C called nesC. 
The motor and mote boards communicate using a messaging layer. 
The motor driver board is not commercially available, but can be custom-built by board fabrication companies, without the researcher having to assemble it by hand. 
The second COTSBots is larger, and will be discussed in the next section. 

\subsection{Room-sized Swarms}

One potential problem with extremely small swarms is that while the robots may scale down, the scale of obstacles they have to traverse may not scale with them. 
As previously mentioned, the Kilobots require a smooth surface, and the E-puck can be stopped by obstacles no more than a few millimetres tall. 
This sort of vulnerability prevents the smaller, tabletop swarm robots from operating well in human-scaled environments. 
In order to overcome this problem, larger swarm robots can be constructed.
 
The MarXbot swarm platform is capable of operating in unstructured human environments. 
MarXBots can also use their grippers to link themselves together and perform operations that an individual robot could not perform, such as bridging a gap larger than a single robot \citep{bonani2010marxbot}. 
The size and complexity of the MarXbots, as well as their powerful computer, renders the individual robots quite expensive. 

Swarmanoid extends the interlinking mechanism of MarXbot to a heterogeneous swarm with three different types of robots \citep{dorigo2013swarmanoid}.
The ``foot'' robots are MarXbots, and provide ground motion for ``hand'' robots. 
``Hand'' robots have grippers to manipulate objects, and can also climb.
The ``hand'' robots have an attachment point like the MarXbots, and so can be carried by ``foot'' robots. 
Flying ``eye'' robots provide overviews of the work area and networking.  

In order to reduce costs, another platform called COTSBots was developed \citep{soule2011cotsbots}.  
Instead of sensor motes on micro-scale RC cars, the newer COTSBots platform is composed of a laptop for processing and a modified RC car, tank, or similar toy as a mobility platform.
In order to interface between the laptop and motor drivers, a second micro-controller board, such as an Arduino or Phidget interface, may be used. 
Due to the diversity of possible combinations of hardware that can be assembled into this configuration, it is still a very viable platform. 
However, the minimum size of this style of COTSBot is the size of a laptop, which is in turn dictated largely by the minimum size of a useful keyboard. 
The large size of these COTSBots demands a very large space if the density of robots in a large swarm is to be kept low. 
Additionally, each laptop has a screen, keyboard, and so forth that are not useful while the robot is operating. 
All of these parts add to the overall cost of the swarm. 

Beyond the scale of rooms, swarm research has been done with Amigobots and Roombas, as well as larger custom platforms for outdoor multi-robot work \citep{guo2007bio, tammet2008rfid, olson2013cacm}.
In theory, swarm research could be performed using robots of any size, but financial limitations would place it out of the reach of most academic organizations. 

\section{UI Designs}

The user interface to a swarm has two functions. 
The first is to allow the user to provide input to the swarm, so that the user can direct the swarm to perform tasks. 
For the purposes of this research, the user interface is a multitouch surface that displays representations of the area the swarm is in and of the individual swarm robots \citep{micire2009multi}. 
The second function of a swarm user interface is to display information about the swarm, or to display information gathered by the swarm to the user. 
By providing an overview of the activities of the swarm, the user interface can give the user feedback on the progress of the task as it proceeds, as well as allowing the user to detect problems. 

Multitouch interfaces have been determined to improve on WIMP or voice interfaces for multi-robot control in a sequence of command and control tasks, including commanding the swarm to a location, performing reconnaissance, and having the swarm cross a dangerous area \citep{hayes2010multi}.
The interface displayed the locations of the robots on a directly manipulatable map, and used movable or semi-transparent user interface widgets, in order to minimize occlusion of the map. 
Areas were selected with with drawing gestures, and paths with fluid strokes, rather than e.g. selection of vertices bounding an area.
The use of multi-touch interaction is desirable because one-at-a-time selection doesn't scale beyond a very limited number of robots.
In order to interact with large groups of robots, the user must be able to perform operations on areas and groupings, rather than on the single point available with a traditional pointer-based interface. 

One approach to getting feedback from a swarm was the development of the Swarmish sound and light system \citep{mclurkin2006speaking}. 
Swarmish provides an ambient means of determining the overall state of the swarm, as well as some information about individual robots. 
The swarm that used Swarmish had autonomous charging, and so the individual robots had long runtimes, and minimal one-on-one interaction with humans. 
The ``ambient'' aspect of the interaction is that the information is continuously available, and the human user ``tunes in'' to it when needed. 
Swarmish uses a set of colored lights and sounds, produced by each robot, to provide feedback. 
The lights were in three colors, and had a total of 108 different combinations of colors and blink sequences, as a visual indicator of the state of each robot. 
In addition to the lights, each robot could produce MIDI notes over its audio system. 
Each note could vary in instrument, pitch, duration, and volume, in addition to having tempos and rhythms as the code executes. 
The designers of Swarmish indicate that the sum of the audio output of the swarm could provide a overall idea of the status of the swarm, but that as a musical instrument, it is difficult to play well. 
Further, the use of lights as signaling mechanisms assumes that you can look at the robots. 

If we accept the assumption that the robots are visible to the user, the robots can carry some form of display that provides local information to the user. 
This information can then be displayed as an overlay in the real world, with the display of the information conterminous with its presence \citep{Daily:2003:WEI:820752.821587}. 
Local display of local information works if the user is part of a hybrid human/robot team, and so is in the same location as the users. 
However, there are many situations where the robot is not in the same location as the user. 
A common example is urban search and rescue, where buildings may be known to be unsafe, or of unknown stability, but it is desirable to search them for trapped people. 
In such a situation, the human user would rather be located elsewhere, and receive information from the robots. 

For situations where the user is not located in the same area as the robots, one possible approach is a ``call center'', where robots can request human attention when required \citep{chen2011supervisory}. 
The human in the call center, however, is faced with having to answer potentially multiple calls with no awareness of the robot's situation. 
The theoretical basis for call center UI is Supervisory Control. 
Supervisory control has the human act as the planner and monitor of the systems being supervised, but allowing the systems to operate on their own.
Automation is frequently broken down into ten levels of automation, with level 10 being a fully automatic system with no humans involved, and level 1 having no automation, such as a bicycle \citep{parasuraman2000model}. 

It would be expected that reducing the number of times the human is required to interact with the robot will permit the user to operate more robots.
At level 1, the user has to interact constantly, and so could not be expected to operate more than one robot. 
By increasing the level of autonomy of the robot, the time required for the user to operate the robot decreases.
Instead of continuous interaction, the user can specify actions for the robot to undertake, and then ignore the robot while it performs the actions.
It is expected that the robot's effectiveness will decline over time since the last user interaction. 
This time that the robot operates without interaction before its effectiveness declines to a fixed minimum is called ``neglect time"\citep{olsen2003metrics}.
With increasing autonomy, neglect time increases.
The fifth level of the autonomy scale is a sort of operation by consent model, where the computer chooses a route and executes it if the human permits it. 
The ninth level is the inverse of the fifth.
Rather than asking the user for consent to act, the robot acts and informs the human only in exceptional cases. 
At the higher levels of the autonomy scale, the robot's neglect time far outweighs the time the user is expected to operate it, and so the user could reasonably be expected to operate other robots during the neglect time. 

Increasing neglect time may allow the user to operate more swarm robots, but it comes at a cost. 
The longer a user goes without learning about the state of one of the robots, the less idea they will have of the situation when they are called upon to operate that robot. 
The problem of automation decreasing the situational awareness (SA) of the user has been described in cockpit automation for aircraft \citep{wiener1980flight}, and generalized well to other systems that combine automation with human control \citep{kaber1997out}. 
If the user takes a long time to relearn the situation, the efficiency of the system will drop. 
Worse, the user may make errors because of an incorrect understanding of the system when they begin operations after a long neglect time \citep{cummings2008predicting}. 
One possible approach to maintain a constant and manageable workload on the user is adapting the level of automation to the workload. 
When the load is low, the user is more directly engaged, but when load is high, there is more automated assistance. 
By varying the level of automation, the workload for the user is kept constant. 
A constant workload is desirable because the user remains engaged with the work, and so has an ongoing understanding of the situation as it develops. 
The user is not suddenly called into a situation after remaining disengaged for some time. 
However, the workload must also be manageable. 
If the user is overloaded, their attention will become subject to triage, and they will begin to miss elements of the task. 
Adaptation does not have to be based on measured load, but could instead be based on perceived load or physiological markers in the user. 
%\citep{goodrich2010maximizing}

However, in situations with even moderate numbers of robots, even relatively high levels of automation may overwhelm the user \citep{lewis200617}. 
Level five, operation by consent, is a fairly high level of autonomy, but with a large number of robots checking in, even this level may generate too many events for the human to deal with. 
Increasing the autonomy to level nine, so that the robots are only checking in with the operator when an exceptional situation occurs, may still overwhelm the operator if enough robots are active.
Increasing the use of automation may also create new difficulties by leaving operator out of practice, or encouraging mis-placed trust in the automation's ability \citep{lee2004trust}. 

In fact, any kind of multitasking may prove insufficient for large swarms. 
For teleoperation, the best case is uncrewed aerial vehicles (UAVs), which require relatively little oversight. 
Uncrewed ground vehicles (UGVs) require more oversight than UAVs, due to the higher complexity of the ground environment. 
Estimates place the limits on the number of robots under control at 12 or 13 for UAVs and 3-9 for UGVs \citep{WangSearchScale}.  
There is some latitude, at least in UGVs, to increase multitasking by increasing automation, as shown by the relatively wide range in the interaction limits, but even 9 robots per operator is nowhere near the scale of kilo-robot swarms  \citep{Olsen:2004:FMH:985692.985722}.
Failure generally takes the form of task effectiveness no longer increasing as more robots are added.
Instead, the amount of time the user spends interacting with the robots begins to outweigh the neglect time, and so the robots spend increasing amounts of time waiting for interactions \citep{cummings2008predicting}. 

Ecological interface design (EID) presents a possible guide to the architecture of user interfaces for swarm robotics, and has been used in interfaces with mixed human-robot teams \citep{vicente1992ecological, gancet2010user}. 
In EID, a user's abilities that enable them to interact with a system is separated into a taxonomy of skills, rules, and knowledge. 
The user has skills, which are rote, simple activities that form the basis of the normal operation of the system. 
The user also knows a set of rules about the system. 
Rules allow the user to handle exceptions or unusual cases that have come up before. 
Rules do not require the user to understand the system, just to know that when certain situations are recognized, certain actions must be performed in response. 
Beyond rules and skills, the user also has knowledge of the system. 
Knowledge allows the user to handle novel exceptions. 
Knowledge gives the user an understanding of how the system works, and can apply that understanding to react to situations that the user has not experienced or been told about before. 
Events are also broken into three levels: routine, which uses skills; foreseen exceptions, which use rules; and unforeseen exceptions, which use knowledge. 
All levels should be supported by the interface, but the user should not be forced to operate at a higher level than is required. 
The abstraction of the process maps onto the hierarchy of ecological design, with the highest level being the function of the process and the lowest level being how the function is accomplished. 
At each level, there are constraints on the process that are used to define the normal operation of the process.
Detection of exceptions requires the display of all constraints, because exception is the breaking of constraints, and undisplayed constraints cannot be assessed to determine if they have been broken.

The user should be able to extract meaning from the information display quickly, as in the case of Swarmish and the robot-as-pixel UI designs.
By using the lights in Swarmish, the user can assess the state of individual robots, but by listening to the overall sound of the swarm, the user can also assess the behavior of the system as a whole.
The state and status lights of an individual robot is the low level in EID, watching how an individual action of the overall process is progressing. 
The ``tune'' of the entire swarm, produced by the sum of their MIDI notes, provides the high level overview, where a user can tell if the system is progressing well or developing problems. 
As the system changes, the changes and predictions should be highlighted so that the user understands consequences of their actions. 
In Swarmish, sudden changes in the tone or tempo of the swarm tune indicate transitions in its behavior, without the user having to observe the actions of the robots closely. 

EID is well-positioned to deal with emergent behavior, because the emergent behavior of the entire system is present at the functional level, but is composed of actions at the physical level.  
The control of swarm robots can be viewed as a hierarchy of increasing abstraction. 
At the least abstract, base level are the individual interactions of the swarm robots with each other and their environment, as dictated by their explicit programs. 
Above that level is the implicit, emergent behavior of the swarm as a whole. 
Finally, the most abstract level is the user intent, as expressed in the interface through their gestures. 
This hierarchy corresponds well to the abstraction of process in EID, with discrete physical actions at the lowest level and the overall results of the process at the highest level. 
Consequently, the user is permitted to issue commands in the most abstract domain, and the system can propagate them ``downwards'' into the concrete actions of the robots in the world, while also propagating information from individual robots ``upwards'' into the global view. 

Automation in EID allows the user to operate primarily with rules and knowledge, dealing with exceptions \citep{vicente2002ecological}.
The interface should allow direct manipulation of perceptual forms that map directly onto work-domain constraints and represent all of the information identified by the abstraction hierarchy. 
In a swarm context, this means displaying functional information in such a way that the user can move across the hierarchy from individual swarm bots to high-level swarm-wide tasks, and interact at all levels to control the swarm. 
More practically, this means that the information displayed must be integrated in such a way that the mapping from one unit of information to another is made apparent in the interface, rather than offloaded to the user to compute in their head \citep{yanco2004beyond}. 
For example, if a robot can send video and range information, the video and range information can be projected into a 3D space around the robot, rather than being displayed in separate UI windows.
Such a projection allows the user to easily relate visual and range information, and relate that information to the ongoing robot control task, which in turn increases task performance \citep{ricks2004ecological}.
Previous work in multi-touch interfaces directly satisfies these requirements of EID by providing both an omniscient camera view for direct manipulation of the high-level, functional actions of the entire swarm, and the ability to move down the hierarchy to control individual swarm members \citep{Micire:2009:ANG:1731903.1731912}.
The ability to display information about individual robots along side or on top of the interface representation of the robot is an important method of providing feedback to the user \citep{Kato:2009:MIC:1520340.1520500}. 

% TODO: Maybe use some of this stuff, but I don't have a good idea where
%Emergent behaviors arise from the interactions of actors with each other and the world around them. 
%In the face of uncertainty in the world, the behaviors will also become uncertain. \change{citep trust work}
%Programs synthesized to guide the swarm should be designed to be robust against failure or degradation of swarm members. 
%The heterogeneity of the swarm may also be leveraged to increase its robustness against failures of individual nodes or alterations of the environment. 
%However, because heterogeneity increases the dimensionality of the solution space for program synthesis, it may adversely affect the performance of the program synthesis and the swarm's runtime convergence to the desired state.
%
%A concept of the swarm as a whole as a programmable entity runs into trouble with reliability. 
%In conventional compilation, assuming the compiler is correct and the computer is correct, the compiled binary does what the source code says. 
%Robots interact with the real world, which is much less likely to be ``correct'' in the same sense a compiler can be asserted to be. 
%Programs for swarms are only going to be functional within some probabilistic grounds and assumed conditions. 
%This requirement indicates that the situation has to be at least somewhat known ahead of time, so that the robots will all receive programs that allow them to perform the task.
%In the ideal case, the emergent action of all of the robots interacting with the environment causes them to perform the task. 
%
%Swarms have more uncertainty, because the reliability of individual robots is low; and higher attentional demands because there are many robots. 
%It may be that above some threshold, the attentional demand will drop again, as the group is no longer treated as a large number of individuals, but as a single group. 
%%A lot of Dr. Adams' work in HSI was under ONR Award N00014-12-1-0987
%
%If the user is unconcerned with the functioning of individual swarm members, so long as the swarm as a whole remains functional, the UI may simply drop malfunctioning individuals from view. 
%This handling of error conditions on individual swarm units fits with the assumption that the swarm as a whole achieves robustness through redundant expendable units, while also allowing the human user to have a rough idea of how the situation is developing by watching the cloud shrink. 
%Do long as progress appears to be being made on the mission, the user might let underperforming units slide. 
%The supervisory system might not even announce when units are lost, until it starts to affect performance.  

\section{Swarm Software Development Methods}

Because the conversion of the specification of desired behavior for the swarm into individual programs for the swarm member robots is still an open question, it is necessary to understand the current methods used in the development of programs for swarm robots. 
Much of swarm robotic development follows the usual model of software development. 
Starting from a desired functionality, the developer writes a program that they think will provide that functionality.
The program is then tested, in simulation or on real robots, and its behavior is observed. 
The programmer then modifies their program to account for any observed difference between the desired function and the system's behavior. 
This loop of coding, testing, and coding again is repeated until the system behaves as expected, or the programmer graduates. 

Because the normal software development model is time-consuming, and outside of the abilities of many people who might want to use swarm robots, it is desirable to automate the development of software controllers for robots. 
One approach to the conversion of the command language to programs for the robots is to define a transformation from the command language to executable code that can be codified into a compiler. 
As a consequence, input in the command language defines a program which is compiled and loaded onto the robots. 
Another possibility is the composition of preexisting behaviors that each satisfy part of the user's desired behavior. 
Pheromone robotics provides one possible method of controlling this composition dynamically. 
Still another approach is to allow the user to specify a desired behavior and evolving controllers to match it. 

\subsection{Amorphous Computing}

Amorphous computing (AC), also called spatial computing, is computation using locally-linked and interacting, asynchronous, unreliable computing elements dispersed on a surface or throughout a volume \citep{abelson2000amorphous}. 
The motivation for AC is that while it may be possibly to produce arbitrary quantities of ``smart dust'', it is not possible to ensure that it all works well and is precisely located, especially in real-world applications.
The goal of AC is to get useful work out of such materials, despite uncertainty as to their reliability and location. 
Smart dusts are also the limit-case, in terms of scale, for swarm robotics, and if AC promises to get useful work out of smart dust, then it also has some applicability to larger swarm robots.

There are several languages intended to program amorphous computers. 
``Proto" is a language for a continuous plane spatial computer \citep{correll2009ad}.
Because the devices are distributed over a plane, the difficulty in communicating between any two devices is a function of the distance between them, much as with RF or other radiative communications.
In Proto, the behavior of regions of space is described by the programmer, and the description is transformed into local actions for the network of devices. 
Because devices have a size in the real world, and space between them, the devices cannot not have a one-to-one mapping with the space, but instead perform an approximation of the desired behavior. 
Swarm robots are mobile, so some swarm algorithms can be implemented as a description of constraints on the robot's state, such as ``the robot must have communications links to no more than 2 and no less than 1 other robots", and a command to move randomly unless the constraint is satisfied. 
Within a bounded environment, such an algorithm can be shown to converge to satisfy its constraints \citep{correll2009ad}. 

Proto also has considerable appeal as a programming language for swarm control development because of the layering in its structure. 
Proto is designed to map from behavior of regions at the global level to programs for discrete points at the level of individual devices \citep{beal2006infrastructure}. 
If user interface interactions can be interpreted as indications of desired behaviors displayed over spatial regions, then conversion of those behaviors into programs in Proto may be amenable to automation. 
Proto's layered structure also has a clear relationship to the hierarchical structure of EID, with the programming language serving as a user interface at the highest abstraction level of the interface design, but providing a smooth transition to the lower abstraction levels.  

Origami Shape Language (OSL) uses the abstraction of a foldable sheet to form shapes, inspired by both origami and the folding of epithelial cells during the development of biological organisms \citep{nagpal2004engineering, nagpal2001programmable}.
Regions and edges on the sheet can be defined by propagation of morphogens, and folds along the edges result in the development of the final form.
Because of the use of morphogens and local communication between the agents on the sheet, there is no need for a global controller to dictate the development of the final form. 
Further, because the high-level description of the desired form does not involve abstractions of the underlying modules, OSL could operate on interlocking modular robots, actuated flexible materials, swarms, or other kinds of computational media. 
In fact, the flexible sheet could be assumed to be virtual, and the resulting motions of the sheet could be translated into motor commands to configure swarm robots into specific arrangements in space. 

Growing Point Language (GPL) allows the specification of topological patterns in an amorphous computer, and so can also be used to specify the distribution of swarm robots, or behaviors of the swarm robots, in a space \citep{nagpal2004engineering}. 
GPL is inspired by the morphogenic controls present in biological organisms, which use gradients of chemicals called morphogens to dictate the development of cells \citep{turing1952chemical}.
The name GPL arises from one of the language's main abstractions, the growing point. 
The growing point is the location of activity within the amorphous medium, at which local agents are changing their state. 
Growing points move through the medium, affecting the state of the computational points they pass, and emitting pheromones into the medium which control the motion of growing points.

Importantly, GPL does not make any prior assumptions on the location of the particles in the system, or robots in the swarm, aside from that they are sufficiently dense in the medium. 
For swarm robotics, this is an important quality, as precise localization may not be available. 
Initially, all agents have the same state and program, with a few exceptions that serve as seeds for the growth to begin. 
If the pattern is not required to be fixed at a particular location, even the seeds could be undetermined initially, and elect themselves via a method such as lateral inhibition. 
During the execution of the GPL program, each agent chooses its state based on the presence of pheromones, which are morphogens with limited range. 
Range limitation on morphogens propagating between robots is set using a TTL (Time To Live) counter that propagates with the morphogen, and is decremented with each hop in the communication network. 
When the TTL hits zero, the morphogen message is no longer propagated. 
By controlling the production or propagation of morphogens within the amorphous medium, complex patterns can be developed. 

\subsection{Pheromone Approaches}

The abstractions of GPL bear a very strong relation to pheromone robotics. 
Pheromone robotics refers to a metaphor for developing control software for swarm robots. 
Some social animals, especially insects, use chemical signals called pheromones to communicate with each other. 
For example, wasps inside their nest react to the scent of wasp venom by travelling to the outer surface of the nest and attacking nearby moving targets  \citep{jeanne1981alarm}.
Ants leave trails of pheromones for other ants to follow to food sources. 
Each individual ant's contribution to the trail can be modulated by the quality of the food source, which allows the reaction of the other ants to the trail to cause an emergent distribution of the foraging work force that favors higher-quality food sources \citep{sumpter2003nonlinearity}.
Because pheromones are chemicals with spatial locations, it would be possible to combine the use of pheromones with reaction diffusion equations to structure activity within a space or to converge to patterns of activity over time  \citep{turing1952chemical}. 
Assuming even diffusion of the robots in space, the global map of the pheromone concentrations is represented over the network by the locally-computed concentrations computed by each robot.

In pheromone robotics, the pheromones are usually simulated or ``virtual'' pheromones, rather than real chemicals which are detected by chemical sensors (for an exception, see \citep{hayes2001swarm}). 
Each pheromone can have properties such as diffusion and evaporation rates that result in the pheromone spreading in space or gradually disappearing. 
In addition to its properties, the pheromones may have other characteristics which robots can sense. 
For example, a robot may emit a pheromone which diffuses into the environment and evaporates quickly, so distance from the robot can be determined by the strength of the pheromone, and approaching or avoiding the robot may be accomplished by moving up or down the gradient of pheromone strength. 
If the swarm is engaged in a search, each searching robot may emit a ``search marker'' pheromone that lingers in the area after the robot leaves. 
Other robots, on entering the area, would detect the pheromone and know that searching this area again would be fruitless. 
If the object of the search can move, the marker pheromone could diminish as a function of time, so areas that have not been searched for a long time become unmarked and may be searched again. 
Once the target is found, the robot may stop and emit a ``discovery pheromone'', which diffuses into the environment, attracts other robots, and causes them to also emit a discovery pheromone. 
As a result, once any robot discovers the target, all of the robots quickly converge on its location. 

The addition of directional communication for the messages that convey virtual pheromone information allows easy determination of the direction of pheremone gradients \citep{payton2001pheromone}.
Rather than directly diffusing in the space as a chemical would, hop counts in the network of robots simulate diffusion. 
Because routes may be of different length, the message with the lowest hop count is assumed to be the truest indication of minimal distance within the network. 
Rather than modelling the world based on the incoming messages, the content of the pheromone messages and the network behavior as a whole serves as a model of the world, mapped 1:1 onto the real environment. 
While it is possible to build a set of behavioral primitives out of pheromone signalling and associated behaviors, controlling the swarm to perform a task with these primitives is still done by hand \citep{payton2003compound}.

In all of these examples, the sensing of the pheromones is assumed to be local to the robot, at least metaphorically. 
To actually maintain pheromones in the environment without robots being present to transmit them requires, again, a global representation of the task space which the robots can refer to when needed. 
Use of pheromones to guide swarm robots for simulated search and patrol tasks has been demonstrated, with the assumption that there is a central controller maintaining the concentration of pheromones on the map, and informing the swarm members \citep{coppin2012controlling}. 
In a real implementation, some robots could remain stationary and only act as transponders, computing and transmitting the local pheromone information for a given area. 
However, even if the robots are limited to only the pheromones they can directly perceive and emit at the present instant, some emergent behaviors are still possible. 

It has been demonstrated that a swarm can perform construction tasks using only local sensing and no communication \citep{wawerla2002collective, bowyer2000automated}.
However, the addition of communication between systems and memory of the state of the world will improve the efficiency of the system.
The system under discussion was developed to have the task implied by the behaviors available for the agents, rather than generating the program from a higher-level specification, such as the form of the structure to be built.

Pheromone approaches can guide the construction of objects, even if the individual swarm members have no memory and only local perception \citep{mason2003programming}. 
The agents engaged in the construction move at random, and take actions governed by their individual perception of environment at present time. 
The agents can release and react to pheromones in the environment, and so there is an implicit communication via stigmurgy, but no explicit agent-to-agent communication. 
The set of rules that govern the mapping of sensor precepts to actions must be such that no point in the construction of the building can be mistaken for another, as that could result in loops or skipping parts of the building sequence. 
The TERMES project created a compiler that translates desired final structures into rules to guide the construction of those structures by cooperating agents \citep{werfel2014designing}.

Both global vector fields and the global and local blending of vector fields in co-fields can be viewed as subsets of pheromone robotics that use a global spatial representation. 
One approach to a control UI for a remotely-located swarm is a multi-touch interface for specifying a vector field \citep{Kato:2009:MIC:1520340.1520500}.
Because the user interface design focuses on the vector field rather than individual robots, the same control interface can scale to an arbitrarily large collection of robots. 
Vector field paths can have loops, which do not exist in waypoint-based paths. 
Waypoint paths have explicit ends, unless an additional command is added to join beginning and ending points. 

Vector field paths have substantial limitations. 
Because the vectors are bound to a 2-D plane, the paths they create cannot cross each other. 
Instead, they flow together. 
A 2-D vector field is also not a useful metaphor for controlling UAVs.
The vector field could be possibly extended into three dimensions, to control UAVs as well as ground vehicles, but there would have to be some form of discontinuity in the field to prevent assignment of UAVs to ground vehicle paths and vice versa. 
The vector field can be viewed as an abstraction of pheromone control, or even implemented in terms of the presence or absence of virtual pheromones, but it has some limitations that pure pheromone control does not have.
For instance, pheromones permit the presence of multiple pheromones at one point, with multiple meanings, but the vector field has one value for each point. 
The vector field is also not very intuitive to users. 
Kato \emph{et al}. indicates that in order to use the vector field well, the users had to anticipate and project the future motions of the robots. 
Interface changes, such as showing particles on the vector field, could improve usability, but these approaches would have the same scaling problems that the robot representation does. 
When the view is zoomed out very far, individual whirls and eddies in the field may not be visible to the user. 

This vector field interface does not directly map to programs on the robots. 
Instead, the central computer maintains the vector field representation and commands the individual robots.
Vector fields also do not allow the assignment of tasks to robots, but allows the user to directly control the motion of the robots. 
In order to convert from a task-based user interface to a vector field representation, the task would have to be converted into a series of changes to the field.
Since the robots may not have accurate localization within the task space, it may not be possible to guide the robots by relating their position to a global vector field. 

The use of co-fields may provide a way to move the vector field representation from the central computer to the swarm, or allow the swarm to act for some time without constant updates from a central controller \citep{mamei2003co}.
Co-fields distribute the data within a space, which may be physical or may be abstract. 
Agents react to gradients in field, and spread their own fields over local communication networks. 
The overall vector space created by the user (the UI vector space) could be propagated to the robots periodically, and combined with their own internal vector fields to generate movement based on both the user's desires and the local rules operating on each robot. 
As with general vector fields, knowing which areas of the UI vector space are relevant to each robot may require global localization, and so only be available for swarms operating in conditions that permit global localization. 

\subsection{Compositional Approaches}

Rather than developing a novel control program for each robot automatically, it may be possible to compose programs from behavioral primitives, such that some combination of the primitives results in the emergence of the desired behavior. 
A compositional approach to program generation requires the definition of primitives out of which programs can be composed, and some degree of assurance that these primitives can cover the space of possible tasks required from the robot. 
One possible list of primitives is disperse (no other nodes within distance d), general disperse (no more than n nodes within distance d), clump/cluster, attract to location, swarm in a direction, and scan area \citep{evans2000programming}.
Another proposed catalog of behaviors for swarm control bases the simple behaviors on pheromones or chemical sensing in single cells \citep{nagpal2004catalog}. 
The proposed behaviors are the use of gradient sensing for position and direction information, local inhibition and competition, lateral inhibition for spatial information, local monitoring, quorum sensing for timing and counting, checkpoint and consensus sensing, and random exploration. 
The first five are common in amorphous computing as well, but the last three are not. %TODO what are their uses?. 
While these behaviors are themselves expressed in terms of pheromones, the composition of the primitives into complete programs is not dictated by a pheromone-based system.
Furthermore, compostional approaches have been proposed in control-theoretic terms as well as pheremone-based terms, so the process of composition of primitives can be viewed as a metastrategy for the creation of programs, rather than a process specific to pheremone robotics \citep{belta2007symbolic}.

The pheremone based approaches to swarm programming are sufficient for relatively complex behaviors. 
Quorum sensing is used to detect whether the local agent count is sufficient for a task. 
By detecting the presence of sufficient robots to perform a task, the robots can allocate themselves to tasks in a just-in-time manner, rather than being pre-allocated when the task is designed. 
Decentralizing the selection of robots, in turn, may be more robust against failures of individual robots, as it uses the robots that are in the right place at the right time, rather than waiting for specially assigned robots. 
However, under sufficiently bad conditions, a sufficient quorum may never arrive, deadlocking the task. 
In combination with domino timing, where completion of each phase triggers the next, locking at any step could then deadlock the entire process unless another mechanism detects and corrects it.

Individual robots can cooperate without communication to push an object to a beacon based on simple behaviors \citep{chen2015occlusion}. 
Each robot has two simple behaviors.
If the view of the beacon is blocked, and the robot is next to an object, the robot pushes on the object.
If the robots can see the beacon, it wanders and avoids obstacles. 
The sum of the two behaviors results in a net pushing force on the side of the object opposite the beacon, which moves the object to the beacon. 
There do exist certain pathological shapes which the system cannot move towards the beacon, but it is demonstrated to work for all convex shapes. 

Another compositional method for programming robots proposes that the behaviors can be separated into classes, such as motion, orientation, and so forth \citep{mclurkin2004stupid}. 
Among these behaviors are ``primitives'' such as several forms of clustering, which other, later works have treated as an emergent behavior itself, arising from more primitive primitives. 
The variable granularity of the primitives available to compose swarm control programs seems to point to a hierarchy of control elements, with perhaps single motor operations at the bottom, and an increasing composition of elements to create more and more complex behaviors.
Swarm control programs would then call multiple primitive behaviors, providing them with parameters such as degrees of bearing and centimeters of proximity. 
The behaviors ideally run concurrently, and some of them respond to sensor inputs. 
The output of behaviors is whether they are running, translational and rotational velocity for the robot, and LED configuration. 
Because multiple behaviors might specify differing outputs, subsumption and summation are used to arbitrate between behaviors of differing priorities. 

These emergent approaches do not have the robots perform all of their available actions all of the time. 
Instead, it is assumed that the behavior of each robot is controlled by its reaction to the environment around it, and possibly to signals from other robots, so that actions are only performed when they are required. 
As a result, user programs compiled from a higher-level representation could be a table consisting of possible values for the sensors, and the actions to undertake when those values are met.
Guarded Command Programming with Rates (GCPR) provides a formal framework for the analysis of this type of compositional program \citep{napp2011compositional}. 
Robots are assumed to only have local sensing.
The guards of GCPR are conditions on the environment.
When a condition is met, the robot performs actions at a given rate. 
In the concurrent case, this is modeled as each action happening one at a time, but in random order. 
On a real swarm, the actions would take place in parallel, but the concurrent model is more amenable to analysis. 
To determine if a set of actions will be successful, it is required to ensure that for all orderings of all actions, the final state space of the swarm is the desired final state. 
Correct programs are those that reach the target state with probability one, even when composed with bounded failures. 
Once the target state is reached, the program is assumed to halt, so while the final state may be reached very slowly, once it is reached, it is not left. 
In the GCPR models, the time to execution of an action is stochastic, but in the real-world case of noisy or imperfect sensors, the variable time to execution of a guarded behavior would be caused by the imperfection of the robot's ability to detect that the guard was satisfied. 

\subsection{Evolutionary Composition}

Determining how the behaviors should be composed for an individual swarm robot's controller is difficult. 
Unfortunately, much of the process of composing of programs for swarm robots consists of iterating between composing sets of primitives and observing behavior of the system in an ad hoc process, just as with the creation of control programs by coding \citep{palmer2005behavioral}. 
Rather than removing iterative software design from the process, it has simply been moved up one layer of abstraction, from writing code for the robots to composing that code from behavior primitives. 

One possibility is to permit composition to behave like the programming environment Tierra \citep{ray1991approach}.
In Tierra, there is no such thing as an invalid program. 
All sequences of the existent symbols are regarded as executable programs, although some are more useful than others. 
The possibility that programs can be ranked by some function creates the possibility that genetic algorithms (GA) can direct the automated composition of behavioral primitives into programs \citep{palmer2005emergence}.
A GA expresses the robot program as a genome, which is translated into the actual program and run on the robot. 
The result of running each program is assessed using a fitness function and then the genomes for the best programs are combined to produce a new generation of genomes. 
This cycle of combining and assessing genomes continues until a certain level of quality is reached, as judged by the fitness function.

Unfortunately, given the time required to iterate over multiple generations of controllers, genetic approaches are unlikely to be fast enough for interactive control of robots by a human user. 
However, it is still useful to examine the possibility of developing a fitness function as a way of investigating methods for automatically assessing the behavior of a swarm.
Without some way of determining the ``goodness'' of a swarm's behavior, it is useless to say that one algorithm or design paradigm is better or worse than another. 

In order to determine the quality of the behavior of the swarm, its behavior must be measured.
Harriott \emph{et al}. propose that metrics for measuring the interaction of humans and swarms differs significantly from the interaction of humans and individual robots, and can be broken down into 9 classes \citep{harriott2014biologically}. 
\begin{itemize}[noitemsep]
\item Human attributes - Interaction, trust, intervention frequency 
\item Task performance - Ability to accomplish task, speed, accuracy, cost
\item Timing - Command diffusion lag, behavior convergence
\item Status - Battery life, number of functioning members, stragglers
\item Leadership - Interaction between special members of the swarm and others
\item Decisions - Action selection, likelihood that the correct action is chosen
\item Communication - Speed, range, network efficiency
\item Micromovements - Relative motion of individual swarm elements
\item Macromovement - Overall swarm motion, flocking, elongation, shape 
\end{itemize}

Task performance is an especially interesting metric, but it is difficult to automatically extract from the observed behavior of the system an overall understanding of the progress it is making on the task, and so a value for the output of the fitness function. 
Worse, without a time bound on solving a problem or a way to calculate progress, it is impossible to tell if a program has failed, or has merely not yet succeeded.
For example, assume a program's intended purpose is to gather all of the units of a resource at a goal. 
If the program merely moves the units stochastically, sometimes they will enter the goal, creating an appearance of progress. 
However, it may be vanishingly unlikely that all the units will randomly happen to be in the goal at once. 
Counting the units moved to the goal, then, cannot distinguish between a program that cannot find the goal, and so will never put any units in it, and a perfect resource-gathering program that just hasn't moved any units to the goal yet.
 
Ideally, it would be possible to recognize and evaluate performance on sub-problems. 
It has been proposed that the interactions and emergent behavior of the system are observable, while the reactions of the agents in the system are programmable, and so by observing the interactions and emergent behavior, the developer can receive feedback on how the system is progressing \citep{palmer2005behavioral}. 
However, all of the proposed observation and hierarchy of reaction and emergence is intended as a design process, not an automation process. 
In other words, while the observation and hierarchical structure may guide the development of an emergent system, the system is still developed by programmers writing code and then running it on the robots.

In the limit, the swarm could be treated as a gas, and for tasks such as diffusion over an area, the performance of the swarm would be compared to the behavior of an ideal gas \citep{jantz1997kinetics}.
The addition of sensors and computation would then allow the robots to outperform a gas at tasks, and so achieve higher scores on a task-oriented metric than a gas could attain. 
Unfortunately, this metric is constrained to tasks that an ideal gas could perform, which are largely restricted to diffusion and alteration of density in response to temperature. 

Controllers have been evolved to allow robots to move into formation from random starting positions \citep{quinn2003evolving}. 
These controllers use local interactions and minimal sensing to achieve their goals. 
One point the authors make, which is not frequently mentioned in other work, is that while flocking or shoaling behavior is a relatively simple behavior to have emerge from robots who can detect the distance, position, and velocity of the other nearby robots, implementing that perception on real robots is quite difficult.
Because a specific behavior was desired, the fitness function used to evolve it was specified in terms of metrics related to the behavior. 
Task-specific fitness functions are also found in later work on evolution of swarm robot behavior, which seems to indicate that evolution of behaviors in swarm robots may only be a time-complexity trade-off. 

Interestingly, some of the work in evolvable controllers leads to inter-robot communication as one of the emergent properties of the evolved controller \citep{quinn2001evolving}.
In order to move as a formation, one of the robots must be the leader, but there is nothing in the fitness function or any of the other code that designates roles for the robots. 
Instead, the selection of the leader arises from the evolutionary development of the controllers, and is present in the controller as a response to a particular series of stimuli. 
Genomes that did not encode such a symmetry-breaking reaction never developed a leader-follower distinction, and so failed to move in formation, and so received low fitness scores. 
For the follow-the-leader task, genetic variation among the robots increased fitness more readily than having all robots share the same genome \citep{quinn2001comparison}.
The condition where all robots shared the same genes was called ``clonal", while each robot having its own genome was ``aclonal".
Oddly, while one would expect that the aclonal condition would result in a specialization, with each robot developing a genome that performed either the leader or follower role well, the aclonal condition developed robots which could perform both roles. 
It was hypothesized that while the clonal condition had to evolve roles and an allocation mechanism simultaneously, the aclonal condition could specialize the roles during early evolution, and then develop an arbitration mechanism to select roles.

Genetic algorithims have also been used to develop aggregation behavior in swarm robots \citep{bahgecci2005evolving, dorigo2004evolving}.  
Aggregation was chosen because it is a preliminary behavior primitive, which the swarm might engage in prior to doing some other task, such as moving an object, attacking together, etc.
The resulting controllers only controls aggregation behavior, so each behavioral primitive would require its own evolutionary development. 
Solutions discovered by genetic algorithms are also prone to overfitting. 
The swarms described in Dorigo \emph{et al}. decreased in performance when the number of robots involved in the swarm was changed from the values used to evolve the solutions, and when a more accurate physical model was used in the simulations.

The common hope of users of genetic algorithms is that they can reduce the complexity of directly specifying the task to the (hopefully lower) complexity of describing the results in the fitness function.
Rather than describing how to solve a problem, one simply describes what the solution would look like. 
Unfortunately, the reduction in hands-on time spent programming frequently comes at the expense of time spent waiting for the system to converge, or determining why it converged on a problematic solution. 
One early version of the evolved aggregation controllers allowed the evolved motion strategy to acquire a high fitness by spinning in place, and so the fitness function required modification to prevent robots from simply spinning. 
The ad hoc iterative process of creating emergent behaviors is replaced by an ad hoc iterative process of creating fitness functions.
As a result, developing novel behavior in the field by converting user specifications of the behavior of the swarm into a fitness function for a genetic algorithm is unlikely to yield results in a timely manner. 
However, individually evolved primitives could be saved in a library of primitives for use by a higher-level compositional approach. 
Such a library could take advantage of the possible overfitting of GA evolution by storing primitives intentionally overfitted to specific situations and robots, and using the best matches. 
For example, a controller that aggregates small-scale UAVs outdoors is likely quite different from one that aggregates medium-scale wheeled robots indoors, even though they are both aggregation controllers. 

\section{Domain-Specific Languages for Swarms}

Proto and other programming languages for amorphous computers provide abstractions for compuation performed on homogeneous spatially-distributed computating nodes, but do not generally support motion of nodes within the space. 
Other versions of tuple-space based amorphous computation include motion of the agents, but do not explicitly support heterogeneity \citep{viroli2012linda}.

The Voltron programming language provides what its authors describe as ``team-level programming'' for autonomous drones \citep{mottola2014team}.
This level of programming is distinct from drone-level programming, where specific instructions are provided to each drone, and swarm programming, where each drone has the same instructions and operates without communication with other drones.
Voltron programs consist of sensing tasks that are subject to space and time constraints, so the language does not permit the user to specify direct interactions between drones. 
This leaves out activities beyond sensing that may be useful for swarm robots, such as patrolling an area or collaboratively moving an object.
Additionally, Voltron is based on the assumptions that drones have global localization, synchronous clocks, and reliable inter-unit communication.   

Karma provides a programming framework for micro-aerial vehicles with minimal localization and no communication in the field \citep{dantu2011programming}.
The framework allows the composition of behaviors described at the level of individual robots. 
Rather than each robot performing series of behaviors in response to input from its sensors, each robot is tasked by a central ``hive'' to perform a single behavior, such as performing a sensor sweep of an area. 
The hive collects data from robots as they return to the hive, and updates a central data store, which includes both the sensor information from the individual robots and spatial information about the sensor information. 
The hive then assigns activities to robots based on rules that use the central data store to determine which activities should be performed. 
As a result, while the individual robots are autonomous and not in communication with the hive while operating, the hive maintains a central data store that is used to guide the future behaviors of the swarm. 
This model does permit a form of interaction between swarm members, in that information from one swarm member can inform the behavior of another member, but it does not permit dynamic collaboration between swarm members while they are operating away from the hive. 

Meld is a programming language for robot ensembles, which are composed of individual modular robots \citep{ashley2007meld}. 
However, many of the problems facing an ensemble of modules are the same as those facing a swarm, such as determining the overall goal, moving to proper positions, and detecting when the goal has been acheived. 
Meld is written in terms of facts and rules. 
Facts describe things such as adjacency between robots and location of robots. 
Rules are applied to facts, resulting in the generation of new facts. 
By including rules that generate facts which describe motor actions, the application of rules to the known state of the system can create a ``to-do list'' of actions for individual members of the system to perform. 
Rules are said to ``prove'' facts, and when no further facts can be proven, the system has arrived in a final state. 
Since facts that alter the state of the world can make previous facts false, the set of facts available must be periodically purged of facts that no longer hold. 
The authors of Meld point out that logic programming, of which Meld is an example, is poor at representing state beyond what can be computed as a consequence of the base facts. 
However, it is also claimed that the use of aggregates, which compute results based on the provable facts, can be used to store state about the system, avoiding this restriction. 

Buzz is a programming language and a virtual machine (VM) to run it on that is designed for programming swarm robots \citep{PinciroliLB15}. 
Each robot is assumed to be running the same bytecode on the Buzz VM (BVM). 
Buzz is also based on the assumption that robots exchange information in a situated manner, with any robot that receives a communication also being able to estimate the relative location of the source of that communication.
In order to support programming for swarms, Buzz treats the swarm and virtual stigmergy as first-class objects in the language. 
Swarms in Buzz provide an abstraction for a group of robots that allows the programmer to have every robot in the group execute a function, as well as dynamically create and disband groups. 
Virtual stigmergy provides a global, distributed data store for the swarm. 
The implementation of the virtual stigmergy structure allows the robots to maintain relatively up-to-date versions of the values stored in the data store, and to refresh them after recovering from failures of network connectivity.
Buzz also provides a convenient abstraction of the neighbours of the robot executing the program, which usually consists of all robots within communication range. 
Using the neighbours abstraction, the robot can, for instance, query all its neighbours about the value of a sensor precept at their location, in order to build a local map of the intensity of the sensed quantity. 

\subsection{Overview of Previous Swarm Hardware}

% Swarm robots are generally small. 
% The reason to keep swarm robots small is related to both the cost of making them and the cost of using them. 
% Larger robots consume more materials per unit, and so cost more money.
% As a result, for a given number of swarm units, larger robots will result in a higher cost swarm. 
% Also, each robot requires some amount of space to move around in. 
% To keep the ratio of free space to robots constant, the area of space used by the robots grows as the robots do. 
% If the ratio is not kept constant, the robots will crowd each other, and so large robots will require either a very large space, or become overly crowded.
% Finally, larger robots are more cumbersome to deal with. 
% They require larger storage areas, possibly teamwork to lift or repair, and so forth. 
% All of these efforts are also multiplied by the number of robots in the swarm. 

% The robots used in most swarm work are of a sufficiently small size that many of them can fit in a room. In addition to budgetary constraints, interaction with an environment built for humans places an upper bound the scale of the individual swarm members. 
% For example, typical indoor doorways are around thirty inches wide, so a robot would have to be less than thirty inches wide to fit through them. 
% The lower bound on swarm robots is generally dictated by fabrication technology, with smaller robots becoming increasingly difficult to assemble. 
% As a result of these bounds, swarm robots are mostly between 1cm$^3$ and 0.3m$^3$. 
% This scale range divides fairly evenly into robots that can operate in large swarms on a table, and those that can operate in swarms within a room, albeit possibly a large room. 
% The challenge of construction of swarm robot hardware, then, is to put all of the same parts as non-swarm mobile robots: a mobility platform, a power supply, a processor, some sensors, and a communication system, into a small package.
% Many impressive designs for small swarm robot platforms have been proposed and constructed as part of research in swarm robotics. 
% However, most of these platforms are no longer easily commercially available, or never were. 

\subsubsection{Tabletop Swarms}

At the low end, in terms of size, the I-SWARM Project was intended to create a 2x2x1mm robot that moved by stick-slip locomotion actuated by piezo levers \cite{seyfried2005swarm}. 
Over the course of the project from 2004-2008, the hardware was developed and used in research, but was not converted to a commercial product.
Other techniques have been developed to use magnetic fields to apply force to small magnetic objects, resulting in controlled motion of the objects \cite{floyd2008untethered, pelrine2012diamagnetically}.
These systems are not amenable to decentralized control, because the moving components are not themselves robots. 
The moving parts are more accurately viewed as manipulators, with the instrumented environment, any sensors for feedback from that environment, and the manipulators themselves comprising a single robot. 

Alice combined a PIC16F84 processor, motors, RF and IR networking, and enough battery power for 10 hours of autonomy into a robot measuring under 2.5cm$^3$ \cite{caprari1998autonomous}. 
The processor used in Alice is relatively underpowered compared to modern processors at the same price point and power consumption. 
Alice robots are no longer available for purchase. 
%The AmIR robot was similar to Alice in size and capability, but with a more modern processor \cite{arvin2009development}.
%There is no evidence that AmIR was ever widely available.

The Jasmine swarm robots were possibly the closest thing to a commercially-available successor to Alice  \cite{kernbach2011swarmrobot}.
Jasmine measured 26x26x20mm, and included an ATMega processor, IR close range communication and obstacle detection, two motor skid steering, and li-po batteries.
Unfortunately, Jasmine units cost about 100 Euro (\$111 USD) each when they were available, and they are no longer available for purchase. 
The plans and information required to reproduce Jasmine units are available for free at Swarmrobot.org.
Assembling a Jasmine robot is not beyond the reach of competent electronics hobbyists, but it does require some unusual build processes, such as grinding down the cases of certain electronic parts and filling holes in the PCB with solder to prevent light leaks. 
The chassis of Jasmine is also a custom mechanical assembly, rather than a commercially available product. 

InsBot was a small robot, measuring 41mm x 30mm x 19mm, that was designed to interact with cockroaches \cite{colot2004insbot}.
It used two processors, one to run higher level behaviors and one to interface with a suite of sensors that included 12 IR sensors and a linear camera. 
InsBots were never commercially available, and each required approximately 6 hours of work to assemble by hand. 
However, the construction process appears to have been relatively straightforward. 

Colias is a dual-microprocessor system similar to Jasmine in functionality \citep{arvin2014colias}. 
It is built out of two PCBs, with the upper PCB and processor handling IR collision avoidance and communication, and the lower processor controlling the motors, power management, and a few other sensors.
Of particular note, the Colias lower processor is responsible for detecting light on the underside of the robot. 
Because of this downward facing light sensor, Colias robots can operate on a translucent table illuminated from beneath with a projector, for experiments in pheromone robotics. 
The light patterns displayed by the projector act as pheromones, which the robot can detect based on its position on the table. 
Colias robots cost \pounds25, or \$35US. 

Even when they are commercially available, most existing swarm robots are too expensive to build a large swarm.
The E-puck from EFPL is approximately 800 Swiss francs (\$810 USD) per unit, so the cost of maintaining a large swarm can become daunting quickly. 
The high price of the E-puck is a result of its extensive suite of sensors, including a camera and 360$^{\circ}$ IR range sensor and communication system. 
E-pucks also require a fairly smooth operating surface.
The E-puck uses differential drive, and allows the front or back edge of the robot to serve as a skid. 
Due to the relatively sharp corner of the lower edges of the robot, the E-puck can become stuck on 2-3mm high obstacles. 

The r-one research robot is cheaper than the E-puck, at approximately \$220 USD per unit \cite{mclurkin2013low}. 
The developers of the r-one position it as a more-featureful and less expensive alternative to the E-puck (\$810, cannot sense neighbors without additional hardware), Parallax's Scribbler (\$198, minimal sensors), the iRobot Create (\$220, requires additional hardware to be programmable), the K-team Khepera III (\$2000), and the Pololu 3pi (\$99, minimal sensors). 
The main advantage in sensing that the r-one has over these other platforms is neighbor sensors and ground truth position sensing, both of which are implemented on the r-one using infrared.
The design of the r-one is open source, but it does not appear to be commercially available as of this writing.   

The Harvard Kilobots are a more recent entry to inexpensive swarms, and have been produced in large quantities \cite{rubenstein2014kilobot}. 
Kilobots contain about \$15 worth of parts, but a 10-pack sells for 1100 Swiss francs, or about \$112 (US) per robot. 
The Kilobots are intended for research in a highly homogeneous environment, with most or all of the robots executing the same program. 
As a result, they are designed to be programmed in parallel using an IR interface. 
For small groups, individual Kilobots can be programmed differently, but any attempt to give each of a very large collection of robots an unique program will take a long time. 
The Kilobots also move by stick-slip motion, and so must operate on a smooth surface, such as a whiteboard. 

One way to reduce the cost of swarm robots is to use commercial, off-the-shelf (COTS) hardware in the construction of the robot. 
Reusing existing hardware leverages the economies of scale that reduce the price of commercial hardware, as well as eliminating the need to design or build the COTS parts. 
Use of COTS parts in research robotics has led to at least two platforms referred to as COTSBots \cite{bergbreiter2003cotsbots, soule2011cotsbots}.
The first COTSBots used mote hardware for the communications link and sensing, plus a motor control add-on board \cite{bergbreiter2003cotsbots}. 
The mobility platform is a modified toy, in particular, a specific brand of high-quality micro RC car.
At the time of this writing, the car used in COTSBots is moderately expensive for a toy car, although quite cheap for a research robot, costing a little over \$100USD per unit. 
COTSBots use TinyOS, a modular and event-driven framework for developing node software \cite{levis2005tinyos}. 
TinyOS is written in a dialect of C called nesC. 
The motor and mote boards communicate using a messaging layer. 
The motor driver board is not commercially available, but can be custom-built by board fabrication companies, without the researcher having to assemble it by hand. 
The second COTSBots is larger, and will be discussed in the next section. 

\subsubsection{Room-sized Swarms}

One potential problem with extremely small swarms is that while the robots may scale down, the scale of obstacles they have to traverse may not scale with them. 
As previously mentioned, the Kilobots require a smooth surface, and the E-puck can be stopped by obstacles no more than a few millimetres tall. 
This sort of vulnerability prevents the smaller, tabletop swarm robots from operating well in human-scaled environments. 
In order to overcome this problem, larger swarm robots can be constructed.
 
The MarXbot swarm platform is capable of operating in unstructured human environments. 
MarXBots can also use their grippers to link themselves together and perform operations that an individual robot could not perform, such as bridging a gap larger than a single robot \cite{bonani2010marxbot}. 
The size and complexity of the MarXbots, as well as their powerful computer, renders the individual robots quite expensive. 

Swarmanoid extends the interlinking mechanism of MarXbot to a heterogeneous swarm with three different types of robots \cite{dorigo2013swarmanoid}.
The ``foot'' robots are MarXbots, and provide ground motion for ``hand'' robots. 
``Hand'' robots have grippers to manipulate objects, and can also climb.
The ``hand'' robots have an attachment point like the MarXbots, and so can be carried by ``foot'' robots. 
Flying ``eye'' robots provide overviews of the work area and networking.  

In order to reduce costs, another platform called COTSBots was developed \cite{soule2011cotsbots}.  
Instead of sensor motes on micro-scale RC cars, the newer COTSBots platform is composed of a laptop for processing and a modified RC car, tank, or similar toy as a mobility platform.
In order to interface between the laptop and motor drivers, a second micro-controller board, such as an Arduino or Phidget interface, may be used. 
Due to the diversity of possible combinations of hardware that can be assembled into this configuration, it is still a very viable platform. 
However, the minimum size of this style of COTSBot is the size of a laptop, which is in turn dictated largely by the minimum size of a useful keyboard. 
The large size of these COTSBots demands a very large space if the density of robots in a large swarm is to be kept low. 
Additionally, each laptop has a screen, keyboard, and so forth that are not useful while the robot is operating. 
All of these parts add to the overall cost of the swarm. 

Beyond the scale of rooms, swarm research has been done with Amigobots and Roombas, as well as larger custom platforms for outdoor multi-robot work \cite{guo2007bio, tammet2008rfid, olson2013cacm}.
In theory, swarm research could be performed using robots of any size, but financial limitations would place it out of the reach of most academic organizations. 

A Minimalist Flocking Algorithm for Swarm Robots
Christoph Moeslinger, Thomas Schmickl, and Karl Crailsheim
	Simulated flocking 
	Can work with heterogeneous robots as long as they use the same sensing method (IR rangers)
		Doesn't actually justify how they know this, though
		Robots heading/motion is abstract, so could be mobility-heterogeneous

Self-Organized Flocking with a Heterogeneous Mobile Robot Swarm
Alessandro Stranieri, Vito Trianni, Eliseo Ferrante, Carlo Pinciroli, Marco Dorigo, Ali Emre Turgut, Mauro Birattari
	Heterogeneous
	Aligning robots can agree on a common heading
	non-aligning robots cannot
	Behavioral heterogeneity
	Boids - seperate, cohere, align
		This work, some robots don't align
	As proportion of aligning robots drops, performance degrades

Mixed Species Flocking for Heterogeneous Robotic Swarms
Sifat Momen, Bala P. Amavasai, Nazmul H. Siddique 
	Defines flocking efficiency as percentage mean number of agents participating in a flock
	Two types of agents, can tell each other apart
	Variable heterospecific attraction (tendency to join each other's flocks)
		Increasing this increases efficency
	Doesn't really indicate why heterogeneity is cool
		Birds apparently form mixed flocks for predator defence

Flocking for Heterogeneous Robot Swarms: A Military Convoy Scenario
Christopher JR. McCook and Joel M. Esposito
	2 unit types, supply unit and defender unit
	Control based on potential field derived from stated goals of the units

Swarmanoid: A Novel Concept for the Study of Heterogeneous Robotic Swarms
By Marco Dorigo, Dario Floreano, Luca Maria Gambardella, Francesco Mondada, Stefano Nolfi, Tarek Baaboura, Mauro Birattari, Michael Bonani, Manuele Brambilla, Arne Brutschy, Daniel Burnier, Alexandre Campo, Anders Lyhne Christensen, Antal Decugnière, Gianni Di Caro, Frederick Ducatelle, Eliseo Ferrante, Alexander Förster, Jerome Guzzi, Valentin Longchamp, Stéphane Magnenat, Javier Martinez Gonzales, Nithin Mathews,
Marco Montes de Oca, Rehan O’Grady, Carlo Pinciroli, Giovanni Pini, Philippe Rétornaz, James Roberts, Valerio Sperati, Timothy Stirling, Alessandro Stranieri, Thomas Stützle, Vito Trianni, Elio Tuci, Ali Emre Turgut, and Florian Vaussard
(37 authors, but it was a big project...)
	Accuses swarm systems of being homogeneous as an oversimpification of natural systems, e.g. all worker ants
	Footbots (wheeled, with omnicams)
	Handbots (climbers and manipulators)
	Eyebots (flyers, with cameras for overhead views)
	No centralized control
	Swarm + Humanoid, intended to refute the idea humans are the best form to operate in human-designed spaces
	Interchangalbe parts, gives robustness
	Local ad-hoc groups and structures to perform tasks
	Being able to physically interact adds complexity to design
		Manipulation and sensing methods must be compatible to allow interaction and communication
		Robots may have very different perceptions and want to communicate about them
	Indirect relationship between what individual robot is doing and what swarm as a whole is doing
		Heterogenieity increases complexity
	All the robots do IR comms and range/bearing sensing
	Eyebots collaborate to localize off each other
		Other robots can also localize off of them

\subsection{UI Designs}

The user interface to a swarm has two functions. 
The first is to allow the user to provide input to the swarm, so that the user can direct the swarm to perform tasks. 
The second function of a swarm user interface is to display information about the swarm, or to display information gathered by the swarm to the user. 
By providing an overview of the activities of the swarm, the user interface can give the user feedback on the progress of the task as it proceeds, as well as allowing the user to detect problems. 

For the purposes of this research, the user interface is a multitouch surface that displays representations of the area the swarm is in and of the individual swarm robots \cite{micire2009multi}. 
Multitouch interfaces have been determined to improve on WIMP or voice interfaces for multi-robot control in a sequence of command and control tasks, including commanding the swarm to a location, performing reconnaissance, and having the swarm cross a dangerous area \cite{hayes2010multi}.
The interface displayed the locations of the robots on a directly manipulatable map, and used movable or semi-transparent user interface widgets, in order to minimize occlusion of the map. 
Areas were selected with with drawing gestures, and paths with fluid strokes, rather than e.g. selection of vertices bounding an area.
The use of multi-touch interaction is desirable because one-at-a-time selection doesn't scale beyond a very limited number of robots.
In order to interact with large groups of robots, the user must be able to perform operations on areas and groupings, rather than on the single point available with a traditional pointer-based interface. 

One approach to getting feedback from a swarm was the development of the Swarmish sound and light system \cite{mclurkin2006speaking}. 
Swarmish provides an ambient means of determining the overall state of the swarm, as well as some information about individual robots. 
The swarm that used Swarmish had autonomous charging, and so the individual robots had long runtimes, and minimal one-on-one interaction with humans. 
As a result, most of the interactions were remote.
The ``ambient'' aspect of the interaction is that the information is continuously available, and the human user ``tunes in'' to it when needed. 
Swarmish uses a set of colored lights and sounds, produced by each robot, to provide feedback. 
The lights were in three colors, and had a total of 108 different combinations of colors and blink sequences, as a visual indicator of the state of each robot. 
In addition to the lights, each robot could produce MIDI notes over its audio system. 
Each note can vary in instrument, pitch, duration, and volume, in addition to having tempos and rhythms as the code executes. 
The designers of Swarmish indicate that the sum of the audio output of the swarm could provide a overall idea of the status of the swarm, but that as a musical instrument, it is difficult to play well. 
Further, the use of lights as signaling mechanisms assumes that you can look at the robots. 

If we accept the assumption that the robots are visible to the user, the robots can carry some form of display that provides local information to the user. 
This information can then be displayed as an overlay in the real world, with the display of the information conterminous with its presence \cite{Daily:2003:WEI:820752.821587}. 
Local display of local information works if the user is part of a hybrid human/robot team, and so is in the same location as the users. 
However, there are many situations where the robot is not in the same location as the user. 
A common example is urban search and rescue, where buildings may be known to be unsafe, or of unknown stability, but it is desirable to search them for trapped people. 
In such a situation, the human user would rather be located elsewhere, and receive information from the robots. 

For situations where the user is not located in the same area as the robots, one possible approach is a ``call center'', where robots can request human attention when required \cite{chen2011supervisory}. 
The human in the call center, however, is faced with having to answer potentially multiple calls with no awareness of the robot's situation. 
The theoretical basis for call center UI is Supervisory Control. 
Supervisory control has the human act as the planner and monitor of the systems being supervised, but allowing the systems to operate on their own.
Automation is frequently broken down into ten levels of automation, with level ten being a fully automatic system with no humans involved, and level one having no automation, such as a bicycle \cite{parasuraman2000model}. 

It would be expected that reducing the number of times the human is required to interact with the robot will permit the user to operate more robots.
With level one automation, the user has to interact constantly, and so could not be expected to operate more than one robot. 
By increasing the level of autonomy of the robot, the time required for the user to operate the robot decreases.
Instead of continuous interaction, the user can specify actions for the robot to undertake, and then ignore the robot while it performs the actions.
It is expected that the robot's effectiveness will decline over time since the last user interaction. 
This time that the robot operates without interaction before its effectiveness declines to a fixed minimum is called ``neglect time"\cite{olsen2003metrics}.

With increasing autonomy, neglect time increases.
The fifth level of the autonomy scale is operation by consent, where the computer chooses a route and executes it if the human permits it. 
The ninth level is the inverse of the fifth.
Rather than asking the user for consent to act, the robot acts autonomously and informs the human only in exceptional cases. 
At the higher levels of the autonomy scale, the robot's neglect time far outweighs the time the user is expected to operate it, and so the user could reasonably be expected to operate other robots during the neglect time. 

Increasing neglect time may allow the user to operate more swarm robots, but it comes at a cost. 
The longer a user goes without learning about the state of one of the robots, the less idea they will have of the situation when they are called upon to operate that robot. 
The problem of automation decreasing the situational awareness (SA) of the user has been described in cockpit automation for aircraft \cite{wiener1980flight}, and generalized well to other systems that combine automation with human control \cite{kaber1997out}. 
If the user takes a long time to relearn the situation, the efficiency of the system will drop. 
Worse, the user may make errors because of an incorrect understanding of the system when they begin operations after a long neglect time \cite{cummings2008predicting}. 
One possible approach to maintain a constant and manageable workload on the user is adapting the level of automation to the workload. 
When the load is low, the user is more directly engaged, but when load is high, there is more automated assistance. 
By varying the level of automation, the workload for the user is kept constant. 
A constant workload is desirable because the user remains engaged with the work, and so has an ongoing understanding of the situation as it develops. 
The user is not suddenly called into a situation after remaining disengaged for some time. 
However, the workload must also be manageable. 
If the user is overloaded, their attention will become subject to triage, and they will begin to miss elements of the task. 
Adaptation does not have to be based on measured load, but could instead be based on perceived load or physiological markers in the user. 
%\cite{goodrich2010maximizing}

However, in situations with even moderate numbers of robots, even relatively high levels of automation may overwhelm the user \cite{lewis200617}. 
Level five, operation by consent, is a fairly high level of autonomy, but with a large number of robots checking in, even this level may generate too many events for the human to deal with. 
Increasing the autonomy to level nine, so that the robots are only checking in with the operator when an exceptional situation occurs, may still overwhelm the operator if enough robots are active.
Increasing the use of automation may also create new difficulties by leaving operator out of practice, or encouraging mis-placed trust in the automation's ability \cite{lee2004trust}. 


How Search and its Subtasks Scale in N Robots
Huadong Wang, Michael Lewis, Prasanna Velagapudi, Paul Scerri, Katia Sycara
	Full task - users directed robots and looked for victims
	Explore - users just directed robots
	Perceptual search - Users just looked for victims
	Workload increases monotonically with robot count
	Task performance on full task increased up to four robots, then decreased
	Increasing autonomy decreases overload
		Depends on task characteristics as well
		Fan-out is higher with simpler tasks

In fact, any kind of multitasking may prove insufficient for large swarms. 
For teleoperation, the best case is uncrewed aerial vehicles (UAVs), which require relatively little oversight. 
Uncrewed ground vehicles (UGVs) require more oversight than UAVs, due to the higher complexity of the ground environment. 
Estimates place the limits on the number of robots under control at 12 or 13 for UAVs and 3-9 for UGVs \cite{WangSearchScale}.  
There is some latitude, at least in UGVs, to increase multitasking by increasing automation, as shown by the relatively wide range in the interaction limits, but even 9 robots per operator is nowhere near the scale of kilo-robot swarms  \cite{Olsen:2004:FMH:985692.985722}.
Failure generally takes the form of task effectiveness no longer increasing as more robots are added.
Instead, the amount of time the user spends interacting with the robots begins to outweigh the neglect time, and so the robots spend increasing amounts of time waiting for interactions \cite{cummings2008predicting}. 

Ecological interface design (EID) presents a possible guide to the architecture of user interfaces for swarm robotics, and has been used in interfaces with mixed human-robot teams \cite{vicente1992ecological, gancet2010user}. 
In EID, a user's abilities that enable them to interact with a system is separated into a taxonomy of skills, rules, and knowledge. 
The user has skills, which are rote, simple activities that form the basis of the normal operation of the system. 
The user also knows a set of rules about the system. 
Rules allow the user to handle exceptions or unusual cases that have come up before. 
Rules do not require the user to understand the system, just to know that when certain situations are recognized, certain actions must be performed in response. 
Beyond rules and skills, the user also has knowledge of the system. 
Knowledge allows the user to handle novel exceptions. 
Knowledge gives the user an understanding of how the system works, and can apply that understanding to react to situations that the user has not experienced or been told about before. 
Events are also broken into three levels: routine, which uses skills; foreseen exceptions, which use rules; and unforeseen exceptions, which use knowledge. 
All levels should be supported by the interface, but the user should not be forced to operate at a higher level than is required. 
The abstraction of the process maps onto the hierarchy of ecological design, with the highest level being the function of the process and the lowest level being how the function is accomplished. 
At each level, there are constraints on the process that are used to define the normal operation of the process.
Detection of exceptions requires the display of all constraints, because exception is the breaking of constraints, and undisplayed constraints cannot be assessed to determine if they have been broken.

The user should be able to extract meaning from the information display quickly, as in the case of Swarmish and the robot-as-pixel UI designs.
By using the lights in Swarmish, the user can assess the state of individual robots, but by listening to the overall sound of the swarm, the user can also assess the behavior of the system as a whole.
The state and status lights of an individual robot is the low level in EID, watching how an individual action of the overall process is progressing. 
The ``tune'' of the entire swarm, produced by the sum of their MIDI notes, provides the high level overview, where a user can tell if the system is progressing well or developing problems. 
As the system changes, the changes and predictions should be highlighted so that the user understands consequences of their actions. 
In Swarmish, sudden changes in the tone or tempo of the swarm tune indicate transitions in its behavior, without the user having to observe the actions of the robots closely. 

EID is well-positioned to deal with emergent behavior, because the emergent behavior of the entire system is present at the functional level, but is composed of actions at the physical level.  
The control of swarm robots can be viewed as a hierarchy of increasing abstraction. 
At the least abstract, base level are the individual interactions of the swarm robots with each other and their environment, as dictated by their explicit programs. 
Above that level is the implicit, emergent behavior of the swarm as a whole. 
Finally, the most abstract level is the user intent, as expressed in the interface through their gestures. 
This hierarchy corresponds well to the abstraction of process in EID, with discrete physical actions at the lowest level and the overall results of the process at the highest level. 
Consequently, the user is permitted to issue commands in the most abstract domain, and the system can propagate them ``downwards'' into the concrete actions of the robots in the world, while also propagating information from individual robots ``upwards'' into the global view. 

Automation in EID allows the user to operate primarily with rules and knowledge, dealing with exceptions \cite{vicente2002ecological}.
The interface should allow direct manipulation of perceptual forms that map directly onto work-domain constraints and represent all of the information identified by the abstraction hierarchy. 
In a swarm context, this means displaying functional information in such a way that the user can move across the hierarchy from individual swarm bots to high-level swarm-wide tasks, and interact at all levels to control the swarm. 
More practically, this means that the information displayed must be integrated in such a way that the mapping from one unit of information to another is made apparent in the interface, rather than offloaded to the user to compute in their head \cite{yanco2004beyond}. 
For example, if a robot can send video and range information, the video and range information can be projected into a 3D space around the robot, rather than being displayed in separate UI windows.
Such a projection allows the user to easily relate visual and range information, and relate that information to the ongoing robot control task, which in turn increases task performance \cite{ricks2004ecological}.
Previous work in multi-touch interfaces directly satisfies these requirements of EID by providing both an omniscient camera view for direct manipulation of the high-level, functional actions of the entire swarm, and the ability to move down the hierarchy to control individual swarm members \cite{Micire:2009:ANG:1731903.1731912}.
The ability to display information about individual robots along side or on top of the interface representation of the robot is an important method of providing feedback to the user \cite{Kato:2009:MIC:1520340.1520500}. 

\subsection{Existing UI Designs}


Affordances for designing natural user interfaces for 3D modelling
Zoltán Rusák, Ismail Cimen, Imre Horváth and Aadjan van der Helm
	Physical and anthropomorphic aspects of hand gestures
	Discussion of various matching methods
	Have to balance what can be detected by the technology with what the user finds appropriate
	Users asked to do gestures for various modeling actions
		motions, rotations of object
		Selection of object
		motion/rotation of camera/zoom
		extrude
		undo
		etc.
	Validation by doing the tasks
		Qualitative accuracy assessment
	Error types
		1 - hand motion isn't part of the UI, so system doesn't recognize it
		2 - system fails to recognize an action that is part of the UI
		3 - errors due to limitations of the tracking technology
	Mouse and keyboard took less time, but more actions
		It would be nice to know how an "action" is defined...
	Extrusion and delete tasks had most of the errors
		Mostly type 2 \& 3, so the gesture chosen by the designer was bad

Semiotic analysis of multi-touch interface design: The MuTable case study
Jan Derboven, Dries De Roeck, Mathijs Verstraete
	Semiotic analysis (meaning of signs and sign systems)
	User interface is a message from designers to users
	early NUI, content serves as the interface
	Users interact in a direct, unmediated way
		In theory, no icons, no metaphors
			But anything on a screen is a metaphor
			
Guiding and Interacting with Virtual Crowds in Real-Time
Soraia Raupp Musse, Fabien Garat and Daniel Thalmann
	Virtual crowds for simulated worlds
	Programmed, autonomous, or guided/interactive control
	UI allows people to type in messages to be sent to the crowds, text boxes and a big "SEND" button
	Guided crowd has some autonomy, can be told to go do specific actions or react to triggered events

IntGUItive: Developing a Natural, Intuitive Graphical User Interface for Mobile Devices
Ammaar Amin Mufti
	Ha ha ha "Natural, Intuitive"
	Depth as a guide
		highlighting and shading
		static light in upper l of screen
		buttons on top of screen plane, screen controls below it
		non-interactive elements should stay flat/have no perspective
	Transition from skeuomorphism to flat/minimal
	Zooming-based UI


Arch-Explore: A Natural User Interface for Immersive Architectural Walkthroughs
Gerd Bruder, Frank Steinicke, Klaus H. Hinrichs
	Cave-based architectural walkthrough software
	Based on actual walking around, but not related (v. clever, though)

Post-WIMP User Interfaces
Andries van Dam
	Eventually wants a butler-style interface
		Not useful for creative tasks, but I guess that's not what he wants to do
	Reasonably precient for 1997, though. 

A comparison of human-computer user interface methods: The effectiveness of touch interface compared to mouse
Muncey, Andrew
	Touch better than mouse for selection, dragging, gestures
	Analysis of some UIs, not relevant at the moment

Skeu the Evolution: Skeuomorphs, Style, and the Material of Tangible Interactions
Shad Gross, Jeffrey Bardzell, Shaowen Bardzell
	Interesting points about skueomorphs kept around as kitsch (deliberately) vs kept around because no one thought to do it a different way

Intel shooting star drones (from Intel-Shooting-Star-Tech-Fact-Sheet-073117-1.pdf)
	"proprietary algorithms"
	Automated creation of image, calculation of drones needed to form it
	Controlled by a single computer (what happens if it goes down?)
	Max range is 1.5km (is that drone travel or wireles range?)

	From video screenshots, appears to be a 3D rendering suite similar to Blender
	GPS + barometer (that's not enough resoultion, they're doing something else)
		Ascending Technologies (AscTec) Trinity Autopilot, triple-redundant IMU
		http://www.asctec.de/en/asctec-trinity/ claims reproducable 3d flight "within gps accuracy"
		ASCTec also had a program called Navigator for GPS flight control

	Can apparently import some form of GIS data to place drones relative to ground level, avoid hills
	Has tabs for "Curves/Surfaces", "Poly modeling" and "Sculpting" in UI
	Also "Rigging", "Animation", FX \& FX Caching, HEALTH, and some stuff I can't make out
	Seems like a 3d CGI suite, but rendering to drones rather than rendering to an STL file

	Coachella video has touch interaction with 3d display of drones in a volume of space

	UI in 500 drone video seems to be earlier version? Still has GIS import
	Also shows single point scrolling and rotation of swarm with touch commands
	Talks about changing thinking "from many individual drones to one single fleet of drones"

	Automated health monitoring and drone selection for better deployment
	Red/yellow/green dot interface for at-a-glance health monitor, probably battery status

Design of a Drone Lead-Follow Control System
Alexander Lanteigne, Elias Kibru, Sabreena Azam, Suliman Al Shammary
	Lead/follow control law
	AscTec drone (hummingbird)
	Also has some stuff on profitability of shows?
		Weird change in focus for the paper
	Models risk of drone loss

Gesture Based Human - Multi-Robot Swarm Interaction
and its Application to an Interactive Display
J. Alonso-Mora, S. Haegeli Lohaus, P. Leemann, R. Siegwart, and P. Beardsley
	Gesture control of robots acting as mobile pixels
	Co-locaed with robots, had to point at the robots
	Useful description of interface
	Swarm still centralized, though
	System operating as an "intractive display", so kind of an electronic toy/amusement device


Human Influence of Robotic Swarms with Bandwidth and Localization Issues
S. Nunnally, P. Walker, A. Kolling, N. Chakraborty, M. Lewis, K. Sycara, and M. Goodrich
	Cites a vector-field based interface for battlefield conditions
		M. Fields, E. Haas, S. Hill, C. Stachowiak, and L. Barnes, “Effective robot team control methodologies for battlefield applications,”
	Other relevant papers
		A. Naghsh, J. Gancet, A. Tanoto, and C. Roast, “Analysis and design of human-robot swarm interaction in firefighting,”
		M. Goodrich, B. Pendleton, P. Sujit, and J. Pinto, “Toward human interaction with bio-inspired robot teams,”
		G. Coppin and F. Legras, “Autonomy spectrum and performance perception issues in swarm supervisory control,”
		Z. Kira and M. Potter, “Exerting human control over decentralized robot swarms,”
	Interface has two commands, "stop" and "head towards"
		Mouse clicks, plus zoomable/pannable viewpoint
	Operators adapted to swarm conditions and stayed able to perform tasks

Two Invariants of Human-Swarm Interaction
Daniel S. Brown, Michael A. Goodrich, Shin-Young Jung, and Sean Kerman
	Claims two invariants:
		(1) collective state is the fundamental percept associated with a bio-inspired swarm
		(2) a human’s ability to influence and understand the collective state of a swarm is determined by the balance between the span and persistence.
	Attractors abstract from individual to group behavior, so humans percieve and try to manage attractors
	"Span" is how many units the controller interacts with
	"Persistence" is how long each interaction takes
		If you have low span, you need long persistence to make a change
	Citation of mulitple fan-out papers
		Need to manage group, not individuals
		"Other work has estimated fan-out using large simulation studies involving several sizes of robot teams (Olsen \& Wood, 2004; Olsen, Wood, \& Turner, 2004)."
		"Pourmehr, Monajjemi, Vaughan, and Mori (2013) examine group-level abstracted multi-robot control to form dynamically-sized teams to increase fan-out."
	Swarm model similar to boids, can transition between torus and flock
		... doesn't really talk about e.g. moving an object or obstacle avoidance
	Shape torus using mediators with attraction/repulsion

Toward Human Interaction with Bio-Inspired Robot Teams
Michael A. Goodrich and Brian Pendleton, P. B. Sujit and Jose Pinto
	Information foraging problem
		Abstract resources at unknown locations in a space
		Agents have to locate the resources and stay at them for a period of time
		Resources are depleted by agents at them
		New resources can appear at any time
		Human input is helpful
	Other appproaches
		Playbook, human calls plays for robots
		Human controls leader, other robots follow leader
		Potential fields
	Lead-by-attraction (leader) vs. lead-by-repulsion (predator)
		lead-by-repulsion makes having persistent interactions hard, and keeps span low over time
	Simulation, no real robots
		Leader models are better than predator models


Human-Swarm Interactions Based on Managing Attractors
Daniel S. Brown, Sean C. Kerman, Michael A. Goodrich
	V. Similar to "Two Invariants", which I think recaps/references it
	Raidus used for orienting affects probability of flock or torus
	So it's controllable, can switch under user control
	Good for e.g. overwatch/loiter \& move-to-area
		Which is to say, flying ISR applications or search \& destroy
	This seems a lot like that military system (Perdix)
		Perdix UI appears to have point-orbit, and move-to-point functions
		Overhead view, but each unit seems to be individualy taskable, but tasking all at once is also supported
			Playbook interface, operator calls plays, drones run them
			https://www.defense.gov/Portals/1/Documents/pubs/Perdix%20Fact%20Sheet.pdf	
	Has study of reliability of quorum sensing in situation with multiple robot failures


Online Feature Extraction for the Incremental Learning of Gestures in Human-Swarm Interaction
Jawad Nagi, Alessandro Giusti, Farrukh Nagi, Luca M. Gambardella, Gianni A. Di Caro
	Learning gestures from a person co-located with the robots
	Can't provide thousands of examples
	Cites work on gaze detecton, again assuming the robots are co-located
	Variant of next paper, not using an SVM anymore, doing online training

Human-Swarm Interaction through Distributed Cooperative Gesture Recognition
Alessandro Giusti, Jawad Nagi, Luca M. Gambardella, Stéphane Bonardi, Gianni A. Di Caro
	Uses swarmanoid footbots
	Bots look at human hand gestures as commands
	Bots also move around to try to figure out what the gesture is
		distributed consensus protocol
	Video session, so paper is one page teaser

Integrating human swarm interaction in a distributed robotic control system
Cristian Vasile and Ana Pavel and C ̆at ̆alin Buiu
	Chidori - 1k birds in Japanese
	Cites Particle swarm optimization methods, physicomimetics, gravity points method
	User creates attraction points and repulsion points
	Network heavy, all done in sim

Bearing-Compass Formation Control: A Human-Swarm Interaction Perspective
Eric Schoof, Airlie Chapman, and Mehran Mesbahi
	Using only local information aids robustness: agent can act alone
	Version in this paper uses bearing only
	Also uses a compass to get rotation of robot in global frame
	Deals with predictability of the motion from an HRI view
		Example is preservation of scale and centroid when changing formation shape
	Control permits combination with human influence to allow scaling and translation of formation
	Very control-theoretic approach

Human-Swarm Interaction Using Spatial Gestures
Jawad Nagi, Alessandro Giusti, Luca M. Gambardella, Gianni A. Di Caro
	Most of the same crew that were throwing gang signs at footbots
	4 selection gestures involving wearing some garish gloves and an orange shirt
	Selection, bit not commands beyond that (in future work)

Human-Swarm Interaction: An Experimental Study of Two Types of Interaction with Foraging Swarms
Andreas Kolling, Katia Sycara, Steven Nunnally, Michael Lewis
	"Currently, multi-robot approaches generally scale to at most ten’s of robots per operator even when using state of the art mapping, path planning, target detection, and coordination algorithms to alleviate the load on the operator (H. Wang et al., 2011; J. Wang \& Lewis, 2007)."
	2 kinds of interaction
		Intermittent
			User tells swarm when to change from one behavior to another
			Selection - active selection of robots
				Scales better to larger swarms
				Outperforms beacon
					Broader span and persistence
		Environmental
			User manipulates environment to cause swarm response
			Beacon - operates on robots within beacon range
	Paper also lists
		Persistent
			User provides a constant control input
				Predator/leader systems, single-joystick systems
		Parameter setting
			Attractor management
	UI appears to be very single-click mouse oriented

User Interfaces for Human Robot Interactions with a Swarm of Robots in Support to Firefighters
Jeremi Gancet, Elvina Motard, Amir Naghsh, Chris Roast, Miguel Munoz Arancon and Lino Marques
	"Sahin [2] describes the swarm robotics as a (i) a large number, of (ii) homogeneous, (iii) autonomous, (iv) relatively incapable or inefficient robots with (v) local sensing and communication capabilities" - Could be the other Sahin definition I was looking for
	Remote v. Prximate interaction
	Early deployment to gather knowledge vs. synchronous deployemnt as helpers
	Details reasons firefighting is hard and how they constrain the interface design
	Light array visor
		LEDs on inside of visor as a HUD
		...I guess that's a good idea? Looks like it could be blinding.
	Paper cites EID, but I'm not sure the LAV is properly EID based on my understanding of it
	Paper mostly about LAV and data on tests with it
		Remote HSI evaluation was inerview with 4 firefighters
		Swarm seems to have been very small (4 robots?) and not very swarm-like

Exerting Human Control Over Decentralized Robot Swarms
Zsolt Kira, Mitchell A. Potter
	Focus on real-time control
	Top down control
		Global swarm characteristics defined, agents optimized to acheive those characteristics
	Bottom up control
		Virtual agents exert forces on the real agents as if they were present in the space
	Physicomimetic control scheme
	Virtual particle parameters evolved in simulation, and then used by the user to perform split and follow operations on swarm

Assessing the Scalability of a Multiple Robot Interface
Curtis M. Humphrey, Christopher Henk, George Sewell, Brian W. Williams, Julie A. Adams
	Halo display that shows were other robts are with respect to selected robot
	Supervisory control usuaally top-down, assumes known map (which USAR may not have)
	Halo kind of like HUD/radar displays in video games
	Tapping halo robot switches to that view
		So this is a many-individuals kind of interface, not a swarm interface
	Increase in workload with bounded time diminishes success
		Adding robots increases workload
	SA didn't appear to drop with more robots
		6 robots vs. 9 robots, which is still in the human-handlable fan-out that other papers have seen
	Experiment proposed halo display, but doesn't appear to have a non-halo test case?
		So they were testing the display in 6 vs. 9 robot cases, but not halo vs. non-halo case
		So the experiment doesn't say anything about the halo case
			Except in comparision to other studies, which they don't compare it to...

A Flexible Delegation-Type Interface Enhances System Performance in Human Supervision of Multiple Robots: Empirical Studies With RoboFlag
Raja Parasuraman, Scott Galster, Peter Squire, Hiroshi Furukawa, and Christopher Miller
	Addresses whether encreasing autonomy can help
	Playbook control, waypoint control, or both
	Delegation decreases mission time and increases successes
	Only 8 robots
	Cost of developing automation tradeoff for gains in efficency
	Simulated red vs. blue teams playing capture the flag
	Second experiment, users don't use automated plays as often as they could have
		And some plays were prefered over others
		And manual control reduced mission time (automation wasn't optimal)
		Having manual and automation is best control, but higher workload
	Seems a lot like the precursor to Perdix, and they mention control of military drones at the end of the paper "Other ongoing work [49], [50] on
the implementation and application of such interfaces provides
further support for their use in complex human–robot systems.
This work has been conducted in a high-fidelity simulation em-
ulating multiple small UAVs (fixed and rotary wing) operating
in an urban environment to perform useful support services for
small units of infantry soldiers."

Neglect Benevolence in Human Control of Swarms in the Presence of Latency
Phillip Walker, Steven Nunnally, and Mike Lewis, Andreas Kolling, Nilanjan Chakraborty, and Katia Sycara 
	Claims to introduce neglect benevolence, first paper on HSI under latency
		Published...?
	Leader-follower control mentioned
	Prev user studies had no delay, or had continuious input
		which is a problem if the user is doing anything else
	Swarm takes time to stabilize
		Cites a lot of HSI papers
		Neglect benevolence is leaving the swarm alone to stabilize
	3 commands: Stop swarm, send heading to swarm, apply boids-style constraints
	Latency adversly impacts finding targets
	Adding a predictive display mitigates it
	Paper never really shows their UI

Virtual Synergy: A Human-Robot Interface for Urban Search and Rescue
Sheila Tejada, Andrew Cristina, Priscilla Goodwyne, Eric Normand, Ryan O’Hara, and Shahrukh Tarapore
	Robots build 3d VR map (Aibos + 2 blimps)
	Virtual world acts as "memory" of the system, stores the things it has seen
	System is based on Tekkotsu
	Humans can interact with robots in VR
	Student paper, probably undergrads
	Not a lot of detail on UI evaluation

\subsubsection{The Interfaceless Interface}	

Some swarm robots are designed to be used with no control interface at all.
By altering the shape of simple stick-slip locomotion toys with laser-cut foam ``hats'' that change their outer perimeter, and changing the outline of the laser-cut foam forms they interact with, the random motion of the toys can create stable structures \citep{andreen2016emergent}).
The development of the shapes relies on human interaction, but specifying the shapes of the hats and interactive components seems amenable to solution with genetic algorithms. 
Because the toys used in this work were about \$4-8 USD, and everything else was composed of laser-cut foam, the platform is extremely inexpensive. 

\todo{GUARDIAN interface or lack thereof}
%The possibility of robots being designed so that their behavior is controlled by their physical form points towards a possible convergence of the physical design of swarm robots and physicomimetic swarm controls with design of macromolecules. 
%Rather than implementing physicomimetic controls in software, the resulting system would use the existing laws of physics as an element of its control program.
%These systems would construct structures larger than themselves as a consequence of how they react, chemically and physically, with their environment. 
%Each assembly step would be done by a vast number of robots so cheap and so fast to build that it is easier to synthesize new ones for each construction step than to build a general-purpose reprogrammable assembler. 


\subsection{Video Game UI Design}

\todo{Mobile video game design similar to top-down interaction with the 3M Multitouch monitor.}

\todo{Mark's work on video games, RTS gamers want menus}

Games with a similar top-down interface for control of multiple units generally appear in the genres of Real Time Strategy (RTS) games, Turn-based strategy games, and Multiplayer Online Battle Arena (MOBA). 

Realtime strategy games generally have the player playing against multiple other players, some or all of which may be controlled by the computer. 
The goal of the game is typically to acquire some form of resources, which are used to develop military units or units which accelerate the acquisition of further resources. 
The military units are then used to attack the other players, with the goal of the game being to be the only remaining player. 

Turn-based strategy games are similar to RTSs, but instead of all players acting simultaneously, each players actions are taken on their turn. 

MOBAs have a smaller focus and more rapid play than RTS or turn-based strategy games. 
Typically, there are multiple players, but each player is on one of two sides battling over some set of objectives on the world map. 
Rather than controlling multiple units, each player controls a single, powerful unit, and may be able to influence weaker, semi-autonomous computer-controlled units.  There may be resources as in RTS games, but typically they are far more limited in number.   

MOBAs, RTSs, and turn-based strategy games all have a similar user interface. 
The ``world'' or area of play is depicted as a map, typically viewed either top-down or an axonometric perspective. 
The map is where the main action of the game takes place, and displays the relative position of the characters and other elements of the game. 
Around the map are typically displays and interactive elements such as menus or buttons to interact with the game. 

The control of units in the game is typically done using the mouse to select units and send commands via mouse clicks, while one hand operates keys, typically on the left side of the keyboard, for other functions. 
How the keyboard is mapped depends on the pace of gameplay. 
For extremely rapid games, such as Defense of the Ancients (DoTA, a MOBA), there are a small number of key commands that can all be operated without removing the fingers of the left hand from the keyboard. 
DotA in particular uses mouse clicking, combined with the control, shift, and alt keys in various combinations, to perform all of the actions of gameplay \todo{cite https://dota2.gamepedia.com/Controls}. 

Age of Empires is a RTS. 
It is less fast-paced than DotA, but still takes place in realtime. 
Most of the keys of the keyboard are mapped to various commands, such as locating and selecting different types of units or buildings, or commanding those units and buildings. 
Commands to buildings typically cause the construction of various units, while consuming resources. 
For example, pressing `D' would find a Dock, and then pressing `A' would command the Dock to start work on a Fishing boat \todo{cite http://ageofempiresonline.wikia.com/wiki/List\_of\_Hotkeys }. 

Europa Universalis 4 uses the entire top row of letters on the keyboard to switch between different map modes, and every other key to activate some feature of the game, as well as mouse controls for interactions with individual units and locations \todo{cite https://eu4.paradoxwikis.com/Controls}. 
Europa Universalis gameplay does not take place in realtime, and the player may speed or slow the game clock, as well as pausing the game. 

Another common feature of MOBA, RTS, and turn-based strategy games is the ``fog of war". 
The entire map is normally hidden from the player until they command their units to enter an area, whereupon the area is revealed. 
In gameplay, the intent of fog of war is to prevent players from knowing about the opposing player's location without sending sorties. 
This lack of information is almost identical to that experienced by a teleoperator of a robot exploring an area with a SLAM algorithm, as the map is revealed to the operator as the robot moves into the area. 

\todo{graphic here to depict typical game UI}

Previous work with top-down multitouch user interfaces has found that users who play video games express a desire for more UI ``widget'' interactions, that is, interacting with buttons, menus, or specialized user interface components \cite{Micire:2009:ANG:1731903.1731912}. 
Gamers who play RTS games also used fewer pinch gestures in a multitouch context, possibly because RTS games are typically played with a mouse and keyboard, which does not have a pinch-like gesture. 

% TODO: Maybe use some of this stuff, but I don't have a good idea where
%Emergent behaviors arise from the interactions of actors with each other and the world around them. 
%In the face of uncertainty in the world, the behaviors will also become uncertain. \change{cite trust work}
%Programs synthesized to guide the swarm should be designed to be robust against failure or degradation of swarm members. 
%The heterogeneity of the swarm may also be leveraged to increase its robustness against failures of individual nodes or alterations of the environment. 
%However, because heterogeneity increases the dimensionality of the solution space for program synthesis, it may adversely affect the performance of the program synthesis and the swarm's runtime convergence to the desired state.
%
%A concept of the swarm as a whole as a programmable entity runs into trouble with reliability. 
%In conventional compilation, assuming the compiler is correct and the computer is correct, the compiled binary does what the source code says. 
%Robots interact with the real world, which is much less likely to be ``correct'' in the same sense a compiler can be asserted to be. 
%Programs for swarms are only going to be functional within some probabilistic grounds and assumed conditions. 
%This requirement indicates that the situation has to be at least somewhat known ahead of time, so that the robots will all receive programs that allow them to perform the task.
%In the ideal case, the emergent action of all of the robots interacting with the environment causes them to perform the task. 
%
%Swarms have more uncertainty, because the reliability of individual robots is low; and higher attentional demands because there are many robots. 
%It may be that above some threshold, the attentional demand will drop again, as the group is no longer treated as a large number of individuals, but as a single group. 
%%A lot of Dr. Adams' work in HSI was under ONR Award N00014-12-1-0987
%
%If the user is unconcerned with the functioning of individual swarm members, so long as the swarm as a whole remains functional, the UI may simply drop malfunctioning individuals from view. 
%This handling of error conditions on individual swarm units fits with the assumption that the swarm as a whole achieves robustness through redundant expendable units, while also allowing the human user to have a rough idea of how the situation is developing by watching the cloud shrink. 
%Do long as progress appears to be being made on the mission, the user might let underperforming units slide. 
%The supervisory system might not even announce when units are lost, until it starts to affect performance.  

\subsection{Swarm Software Development Methods}

Because the conversion of the specification of desired behavior for the swarm into individual programs for the swarm member robots is still an open question, it is necessary to understand the current methods used in the development of programs for swarm robots. 
Much of swarm robotic development follows the usual model of software development. 
Starting from a desired functionality, the developer writes a program that they think will provide that functionality.
The program is then tested, in simulation or on real robots, and its behavior is observed. 
The programmer then modifies their program to account for any observed difference between the desired function and the system's behavior. 
This loop of coding, testing, and coding again is repeated until the system behaves as expected, or the programmer graduates \citep{cham2010graduate}. 

Because the normal software development model is time-consuming, and outside of the abilities of many people who might want to use swarm robots, it is desirable to automate the development of software controllers for robots. 
One approach to the conversion of the command language to programs for the robots is to define a transformation from the command language to executable code. 
The transformation operation can be codified as a sort of ``compiler'', or more accurately, a code generator, which creates programs for the robots. 
The possibility of coding for the swarm as an entire system, rather than writing each robot's program independently, has given rise to several programming paradigms and domain-specific languages for robotic swarms. 

Another possibility is the composition of preexisting behaviors that each satisfy part of the user's desired behavior. 
Given some library of primitives, the user could select a sequence of behaviors, or conditions for the execution of behaviors, to build a complete program. 
Pheromone robotics provides one possible method of controlling this composition dynamically.
 
Still another approach is to allow the user to specify a desired behavior and evolving controllers to match it. 
It would be hoped that evolving the controllers would reduce the complexity of the development task to the definition of a suitable means of determining the fitness of the resulting program. 
Unfortunately, even this reduction does not permit an escape from iterative development. 

\subsubsection{Amorphous Computing}

Amorphous computing (AC), also called spatial computing, is computation using locally-linked and interacting, asynchronous, unreliable computing elements dispersed on a surface or throughout a volume \cite{abelson2000amorphous}. 
The motivation for AC is that while it may be possible to produce arbitrary quantities of ``smart dust'', it is not possible to ensure that it all works well and is precisely located, especially in real-world applications.
The goal of AC is to get useful work out of such materials, despite uncertainty as to their reliability and location. 
Smart dusts are also the limit case, in terms of scale, for swarm robotics. 
Indeed, most of the volume of existing swarm robots is motors and batteries, with the computational components taking far less space. 
If AC promises to get useful work out of smart dust, then it also has some applicability to larger swarm robots.

There are several languages intended to program amorphous computers. 
``Proto" is a language for a continuous plane spatial computer \cite{correll2009ad}.
Because the devices are distributed over a plane, the difficulty in communicating between any two devices is a function of the distance between them, much as with RF or other radiative communications.
In Proto, the behavior of regions of space is described by the programmer, and the description is transformed into local actions for the network of devices. 
Because devices have a size in the real world, and space between them, the devices cannot not have a one-to-one mapping with the space, but instead perform an approximation of the desired behavior. 
Swarm robots are mobile, so some swarm algorithms can be implemented as a description of constraints on the robot's state, such as ``the robot must have communications links to no more than 2 and no less than 1 other robots", and a command to move randomly unless the constraint is satisfied. 
Within a bounded environment, such an algorithm can be shown to converge to satisfy its constraints \cite{correll2009ad}. 

Proto also has considerable appeal as a programming language for swarm control development because of the layering in its structure. 
Proto is designed to map from behavior of regions at the global level to programs for discrete points at the level of individual devices \cite{beal2006infrastructure}. 
If user interface interactions can be interpreted as indications of desired behaviors displayed over spatial regions, then conversion of those behaviors into programs in Proto may be amenable to automation. 
Proto's layered structure also has a clear relationship to the hierarchical structure of EID, with the programming language serving as a user interface at the highest abstraction level of the interface design, but providing a smooth transition to the lower abstraction levels.  

Origami Shape Language (OSL) uses the abstraction of a foldable sheet to form shapes, inspired by both origami and the folding of epithelial cells during the development of biological organisms \cite{nagpal2004engineering, nagpal2001programmable}.
Regions and edges on the sheet can be defined by propagation of morphogens, and folds along the edges result in the development of the final form.
Because of the use of morphogens and local communication between the agents on the sheet, there is no need for a global controller to dictate the development of the final form. 
Further, because the high-level description of the desired form does not involve abstractions of the underlying modules, OSL could operate on interlocking modular robots, actuated flexible materials, swarms, or other kinds of computational media. 
In fact, the flexible sheet could be assumed to be virtual, and the resulting motions of the sheet could be translated into motor commands to configure swarm robots into specific arrangements in space. 

Growing Point Language (GPL) allows the specification of topological patterns in an amorphous computer, and so can also be used to specify the distribution of swarm robots, or behaviors of the swarm robots, in a space \cite{nagpal2004engineering}. 
GPL is inspired by the morphogenic controls present in biological organisms, which use gradients of chemicals called morphogens to dictate the development of cells \cite{turing1952chemical}.
The name GPL arises from one of the language's main abstractions, the growing point. 
The growing point is the location of activity within the amorphous medium, at which local agents are changing their state. 
Growing points move through the medium, affecting the state of the computational points they pass, and emitting pheromones into the medium which control the motion of growing points.

Importantly, GPL does not make any prior assumptions on the location of the particles in the system, or robots in the swarm, aside from that they are sufficiently dense in the medium. 
For swarm robotics, this is an important quality, as precise localization may not be available. 
Initially, all agents have the same state and program, with a few exceptions that serve as seeds for the growth to begin. 
If the pattern is not required to be fixed at a particular location, even the seeds could be undetermined initially, and elect themselves via a method such as lateral inhibition. 
During the execution of the GPL program, each agent chooses its state based on the presence of pheromones, which are morphogens with limited range. 
Range limitation on morphogens propagating between robots is set using a TTL (Time To Live) counter that propagates with the morphogen, and is decremented with each hop in the communication network. 
When the TTL hits zero, the morphogen message is no longer propagated. 
By controlling the production or propagation of morphogens within the amorphous medium, complex patterns can be developed. 

\subsubsection{Pheromone Approaches}

The abstractions of GPL bear a very strong relation to pheromone robotics. 
Pheromone robotics is a metaphor used for developing control software for swarm robots. 
Some social animals, especially insects, use chemical signals called pheromones to communicate with each other. 
For example, wasps inside their nest react to the scent of wasp venom by travelling to the outer surface of the nest and attacking nearby moving targets  \cite{jeanne1981alarm}.
Ants leave trails of pheromones for other ants to follow to food sources. 
Each individual ant's contribution to the trail can be modulated by the quality of the food source, which allows the reaction of the other ants to the trail to cause an emergent distribution of the foraging work force that favors higher-quality food sources \cite{sumpter2003nonlinearity}.
Because pheromones are chemicals with spatial locations, it would be possible to combine the use of pheromones with reaction diffusion equations to structure activity within a space or to converge to patterns of activity over time  \cite{turing1952chemical}. 
Assuming even diffusion of the robots in space, the global map of the pheromone concentrations is represented over the network by the locally-computed concentrations computed by each robot.

In pheromone robotics, the pheromones are usually simulated or ``virtual'' pheromones, rather than real chemicals which are detected by chemical sensors (for an exception, see \cite{hayes2001swarm}). 
Each pheromone can have properties such as diffusion and evaporation rates that result in the pheromone spreading in space or gradually disappearing. 
In addition to its properties, the pheromones may have other characteristics which robots can sense. 
For example, a robot may emit a pheromone which diffuses into the environment and evaporates quickly, so distance from the robot can be determined by the strength of the pheromone, and approaching or avoiding the robot may be accomplished by moving up or down the gradient of pheromone strength. 
If the swarm is engaged in a search, each searching robot may emit a ``search marker'' pheromone that lingers in the area after the robot leaves. 
Other robots, on entering the area, would detect the pheromone and know that searching this area again would be fruitless. 
If the object of the search can move, the marker pheromone could diminish as a function of time, so areas that have not been searched for a long time become unmarked and may be searched again. 
Once the target is found, the robot may stop and emit a ``discovery pheromone'', which diffuses into the environment, attracts other robots, and causes them to also emit a discovery pheromone. 
As a result, once any robot discovers the target, all of the robots quickly converge on its location. 

The addition of directional communication for the messages that convey virtual pheromone information allows easy determination of the direction of pheremone gradients \cite{payton2001pheromone}.
Rather than directly diffusing in the space as a chemical would, hop counts in the network of robots simulate diffusion. 
Because routes may be of different length, the message with the lowest hop count is assumed to be the truest indication of minimal distance within the network. 
Rather than modelling the world based on the incoming messages, the content of the pheromone messages and the network behavior as a whole serves as a model of the world, mapped 1:1 onto the real environment. 
While it is possible to build a set of behavioral primitives out of pheromone signalling and associated behaviors, controlling the swarm to perform a task with these primitives is still done by hand \cite{payton2003compound}.

In all of these examples, the sensing of the pheromones is assumed to be local to the robot, at least metaphorically. 
To actually maintain pheromones in the environment without robots being present to transmit them requires, again, a global representation of the task space which the robots can refer to when needed. 
Use of pheromones to guide swarm robots for simulated search and patrol tasks has been demonstrated, with the assumption that there is a central controller maintaining the concentration of pheromones on the map, and informing the swarm members \cite{coppin2012controlling}. 
In a real implementation, some robots could remain stationary and only act as transponders, computing and transmitting the local pheromone information for a given area. 
However, even if the robots are limited to only the pheromones they can directly perceive and emit at the present instant, some emergent behaviors are still possible. 
Another possible approach to enable pheromones in space is to operate the robots in an area which is illuminated by a projector. 
The projector can then color the area with light that the robots can sense, and modulate the intensity of the light to indicate the intensity of the pheremone \citep{arvin2015cosvarphi,diaz2017human}. 

It has been demonstrated that a swarm can perform construction tasks using only local sensing and no communication \cite{wawerla2002collective, bowyer2000automated}.
However, the addition of communication between systems and memory of the state of the world will improve the efficiency of the system.
The system under discussion was developed to have the task implied by the behaviors available for the agents, rather than generating the program from a higher-level specification, such as the form of the structure to be built.

Pheromone approaches can guide the construction of objects, even if the individual swarm members have no memory and only local perception \cite{mason2003programming}. 
The agents engaged in the construction move at random, and take actions governed by their individual perception of environment at present time. 
The agents can release and react to pheromones in the environment, and so there is an implicit communication via stigmurgy, but no explicit agent-to-agent communication. 
The set of rules that govern the mapping of sensor precepts to actions must be such that no point in the construction of the building can be mistaken for another, as that could result in loops or skipping parts of the building sequence. 
The TERMES project created a compiler that translates desired final structures into rules to guide the construction of those structures by cooperating agents \cite{werfel2014designing}.

Both global vector fields and the global and local blending of vector fields in co-fields can be viewed as subsets of pheromone robotics that use a global spatial representation. 
One approach to a control UI for a remotely-located swarm is a multi-touch interface for specifying a vector field \cite{Kato:2009:MIC:1520340.1520500}.
Because the user interface design focuses on the vector field rather than individual robots, the same control interface can scale to an arbitrarily large collection of robots. 
Vector field paths can have loops, which do not exist in waypoint-based paths. 
Waypoint paths have explicit ends, unless an additional command is added to join beginning and ending points. 

Vector field paths have substantial limitations. 
Because the vectors are bound to a 2-D plane, the paths they create cannot cross each other. 
Instead, they flow together. 
A 2-D vector field is also not a useful metaphor for controlling UAVs.
The vector field could be possibly extended into three dimensions, to control UAVs as well as ground vehicles, but there would have to be some form of discontinuity in the field to prevent assignment of UAVs to ground vehicle paths and vice versa. 
The vector field can be viewed as an abstraction of pheromone control, or even implemented in terms of the presence or absence of virtual pheromones, but it has some limitations that pure pheromone control does not have.
For instance, pheromones permit the presence of multiple pheromones at one point, with multiple meanings, but the vector field has one value for each point. 
Attempting to solve this problem by proposing multiple fields raises the question of how to combine them, and if a universal combination function applies at all points in the field, it raises the question of why they were not combined already. 

The vector field is also not very intuitive to users. 
Kato \emph{et al}. indicates that in order to use the vector field well, the users had to anticipate and project the future motions of the robots. 
Interface changes, such as showing particles on the vector field, could improve usability, but these approaches would have the same scaling problems that the robot representation does. 
When the view is zoomed out very far, individual whirls and eddies in the field may not be visible to the user. 

This vector field interface does not directly map to programs on the robots. 
Instead, the central computer maintains the vector field representation and commands the individual robots.
Vector fields also do not allow the assignment of tasks to robots, but allows the user to directly control the motion of the robots. 
In order to convert from a task-based user interface to a vector field representation, the task would have to be converted into a series of changes to the field.
Since the robots may not have accurate localization within the task space, it may not be possible to guide the robots by relating their position to a global vector field. 

The use of co-fields may provide a way to move the vector field representation from the central computer to the swarm, or allow the swarm to act for some time without constant updates from a central controller \cite{mamei2003co}.
Co-fields distribute the data within a space, which may be physical or may be abstract. 
Agents react to gradients in field, and spread their own fields over local communication networks. 
The overall vector space created by the user (the UI vector space) could be propagated to the robots periodically, and combined with their own internal vector fields to generate movement based on both the user's desires and the local rules operating on each robot. 
As with general vector fields, knowing which areas of the UI vector space are relevant to each robot may require global localization, and so only be available for swarms operating in conditions that permit global localization. 

\subsubsection{Compositional Approaches}

Rather than developing a novel control program for each robot automatically, it may be possible to compose programs from behavioral primitives, such that some combination of the primitives results in the emergence of the desired behavior. 
A compositional approach to program generation requires the definition of primitives out of which programs can be composed, and some degree of assurance that these primitives can cover the space of possible tasks required from the robot. 
One possible list of primitives is disperse (no other nodes within distance d), general disperse (no more than n nodes within distance d), clump/cluster, attract to location, swarm in a direction, and scan area \cite{evans2000programming}.
Another proposed catalog of behaviors for swarm control bases the simple behaviors on pheromones or chemical sensing in single cells \cite{nagpal2004catalog}. 
The proposed behaviors are the use of gradient sensing for position and direction information, local inhibition and competition, lateral inhibition for spatial information, local monitoring, quorum sensing for timing and counting, checkpoint and consensus sensing, and random exploration. 
The first five are common in amorphous computing as well, but the last three are not. %TODO what are their uses?. 
While these behaviors are themselves expressed in terms of pheromones, the composition of the primitives into complete programs is not dictated by a pheromone-based system.
Furthermore, compositional approaches have been proposed in control-theoretic terms as well as pheromone-based terms, so the process of composition of primitives can be viewed as a metastrategy for the creation of programs, rather than a process specific to pheromone robotics \cite{belta2007symbolic}.

The pheromone based approaches to swarm programming are sufficient for relatively complex behaviors. 
Quorum sensing is used to detect whether the local agent count is sufficient for a task. 
By detecting the presence of sufficient robots to perform a task, the robots can allocate themselves to tasks in a just-in-time manner, rather than being pre-allocated when the task is designed. 
Decentralizing the selection of robots, in turn, may be more robust against failures of individual robots, as it uses the robots that are in the right place at the right time, rather than waiting for specially assigned robots. 
However, under sufficiently bad conditions, a sufficient quorum may never arrive, deadlocking the task. 
In combination with domino timing, where completion of each phase triggers the next, locking at any step could then deadlock the entire process unless another mechanism detects and corrects it.

Individual robots can cooperate without communication to push an object to a beacon based on simple behaviors \cite{chen2015occlusion}. 
Each robot has two simple behaviors.
If the view of the beacon is blocked, and the robot is next to an object, the robot pushes on the object.
If the robots can see the beacon, it wanders and avoids obstacles. 
The sum of the two behaviors results in a net pushing force on the side of the object opposite the beacon, which moves the object to the beacon. 
There do exist certain pathological shapes which the system cannot move towards the beacon, but it is demonstrated to work for all convex shapes. 

Another compositional method for programming robots proposes that the behaviors can be separated into classes, such as motion, orientation, and so forth \cite{mclurkin2004stupid}. 
Among these behaviors are ``primitives'' such as several forms of clustering, which other, later works have treated as an emergent behavior itself, arising from more primitive primitives. 
The variable granularity of the primitives available to compose swarm control programs seems to point to a hierarchy of control elements, with perhaps single motor operations at the bottom, and an increasing composition of elements to create more and more complex behaviors.
Swarm control programs would then call multiple primitive behaviors, providing them with parameters such as degrees of bearing and centimeters of proximity. 
The behaviors ideally run concurrently, and some of them respond to sensor inputs. 
The output of behaviors is whether they are running, translational and rotational velocity for the robot, and LED configuration. 
Because multiple behaviors might specify differing outputs, subsumption and summation are used to arbitrate between behaviors of differing priorities. 

These emergent approaches do not have the robots perform all of their available actions all of the time. 
Instead, it is assumed that the behavior of each robot is controlled by its reaction to the environment around it, and possibly to signals from other robots, so that actions are only performed when they are required. 
As a result, user programs compiled from a higher-level representation could be a table consisting of possible values for the sensors, and the actions to undertake when those values are met.
Guarded Command Programming with Rates (GCPR) provides a formal framework for the analysis of this type of compositional program \cite{napp2011compositional}. 
Robots are assumed to only have local sensing.
The guards of GCPR are conditions on the environment.
When a condition is met, the robot performs actions at a given rate. 
In the concurrent case, this is modeled as each action happening one at a time, but in random order. 
On a real swarm, the actions would take place in parallel, but the concurrent model is more amenable to analysis. 
To determine if a set of actions will be successful, it is required to ensure that for all orderings of all actions, the final state space of the swarm is the desired final state. 
Correct programs are those that reach the target state with probability one, even when composed with bounded failures. 
Once the target state is reached, the program is assumed to halt, so while the final state may be reached very slowly, once it is reached, it is not left. 
In the GCPR models, the time to execution of an action is stochastic, but in the real-world case of noisy or imperfect sensors, the variable time to execution of a guarded behavior would be caused by the imperfection of the robot's ability to detect that the guard was satisfied. 

\subsubsection{Evolutionary Composition}

Determining how the behaviors should be composed for an individual swarm robot's controller is difficult. 
Unfortunately, much of the process of composing programs for swarm robots consists of iterating between composing sets of primitives and observing behavior of the system in an ad hoc process, just as with the creation of control programs by coding \cite{palmer2005behavioral}. 
Rather than removing iterative software design from the process, it has simply been moved up one layer of abstraction, from writing code for the robots to composing that code from behavior primitives. 

One possibility is to permit composition to behave like the programming environment Tierra \cite{ray1991approach}.
In Tierra, there is no such thing as an invalid program. 
All sequences of the existent symbols are regarded as executable programs, although some are more useful than others. 
The possibility that programs can be ranked by some function creates the possibility that genetic algorithms (GA) can direct the automated composition of behavioral primitives into programs \cite{palmer2005emergence}.
A GA expresses the robot program as a genome, which is translated into the actual program and run on the robot. 
The result of running each program is assessed using a fitness function and then the genomes for the best programs are combined to produce a new generation of genomes. 
This cycle of combining and assessing genomes continues until a certain level of quality is reached, as judged by the fitness function.

Unfortunately, given the time required to iterate over multiple generations of controllers, genetic approaches are unlikely to be fast enough for interactive control of robots by a human user. 
However, it is still useful to examine the possibility of developing a fitness function as a way of investigating methods for automatically assessing the behavior of a swarm.
Without some way of determining the ``goodness'' of a swarm's behavior, it is useless to say that one algorithm or design paradigm is better or worse than another. 

In order to determine the quality of the behavior of the swarm, its behavior must be measured.
Harriott \emph{et al}. propose that metrics for measuring the interaction of humans and swarms differs significantly from the interaction of humans and individual robots, and can be broken down into 9 classes \cite{harriott2014biologically}. 
\begin{itemize}[noitemsep]
\item Human attributes - Interaction, trust, intervention frequency 
\item Task performance - Ability to accomplish task, speed, accuracy, cost
\item Timing - Command diffusion lag, behavior convergence
\item Status - Battery life, number of functioning members, stragglers
\item Leadership - Interaction between special members of the swarm and others
\item Decisions - Action selection, likelihood that the correct action is chosen
\item Communication - Speed, range, network efficiency
\item Micromovements - Relative motion of individual swarm elements
\item Macromovement - Overall swarm motion, flocking, elongation, shape 
\end{itemize}

Task performance is an especially interesting metric, but it is difficult to automatically extract from the observed behavior of the system an overall understanding of the progress it is making on the task, and so a value for the output of the fitness function. 
Worse, without a time bound on solving a problem or a way to calculate progress, it is impossible to tell if a program has failed, or has merely not yet succeeded.
For example, assume a program's intended purpose is to gather all of the units of a resource at a goal. 
If the program merely moves the units stochastically, sometimes they will enter the goal, creating an appearance of progress. 
However, it may be vanishingly unlikely that all the units will randomly happen to be in the goal at once. 
Counting the units moved to the goal, then, cannot distinguish between a program that cannot find the goal, and so will never put any units in it, and a perfect resource-gathering program that just has not moved any units to the goal yet.
 
Ideally, it would be possible to recognize and evaluate performance on sub-problems. 
It has been proposed that the interactions and emergent behavior of the system are observable, while the reactions of the agents in the system are programmable, and so by observing the interactions and emergent behavior, the developer can receive feedback on how the system is progressing \cite{palmer2005behavioral}. 
However, all of the proposed observation and hierarchy of reaction and emergence is intended as a design process, not an automation process. 
In other words, while the observation and hierarchical structure may guide the development of an emergent system, the system is still developed by programmers writing code and then running it on the robots.

In the limit, the swarm could be treated as a gas, and for tasks such as diffusion over an area, the performance of the swarm would be compared to the behavior of an ideal gas \cite{jantz1997kinetics}.
The addition of sensors and computation would then allow the robots to outperform a gas at tasks, and so achieve higher scores on a task-oriented metric than a gas could attain. 
Unfortunately, this metric is constrained to tasks that an ideal gas could perform, which are largely restricted to diffusion and alteration of density in response to temperature. 

Controllers have been evolved to allow robots to move into formation from random starting positions \cite{quinn2003evolving}. 
These controllers use local interactions and minimal sensing to achieve their goals. 
One point the authors make, which is not frequently mentioned in other work, is that while flocking or shoaling behavior is a relatively simple behavior to have emerge from robots who can detect the distance, position, and velocity of the other nearby robots, implementing that perception on real robots is quite difficult.
Because a specific behavior was desired, the fitness function used to evolve it was specified in terms of metrics related to the behavior. 
Task-specific fitness functions are also found in later work on evolution of swarm robot behavior, which seems to indicate that evolution of behaviors in swarm robots may only be a time-complexity trade-off. 

Interestingly, some of the work in evolvable controllers leads to inter-robot communication as one of the emergent properties of the evolved controller \cite{quinn2001evolving}.
In order to move as a formation, one of the robots must be the leader, but there is nothing in the fitness function or any of the other code that designates roles for the robots. 
Instead, the selection of the leader arises from the evolutionary development of the controllers, and is present in the controller as a response to a particular series of stimuli. 
Genomes that did not encode such a symmetry-breaking reaction never developed a leader-follower distinction, and so failed to move in formation, and so received low fitness scores. 
For the follow-the-leader task, genetic variation among the robots increased fitness more readily than having all robots share the same genome \cite{quinn2001comparison}.
The condition where all robots shared the same genes was called ``clonal", while each robot having its own genome was ``aclonal".
Oddly, while one would expect that the aclonal condition would result in a specialization, with each robot developing a genome that performed either the leader or follower role well, the aclonal condition developed robots which could perform both roles. 
It was hypothesized that while the clonal condition had to evolve roles and an allocation mechanism simultaneously, the aclonal condition could specialize the roles during early evolution, and then develop an arbitration mechanism to select roles.

Genetic algorithims have also been used to develop aggregation behavior in swarm robots \cite{bahgecci2005evolving, dorigo2004evolving}.  
Aggregation was chosen because it is a preliminary behavior primitive, which the swarm might engage in prior to doing some other task, such as moving an object or attacking together.
The resulting controllers only controls aggregation behavior, so each behavioral primitive would require its own evolutionary development. 

Solutions discovered by genetic algorithms are also prone to overfitting. 
The swarms described in Dorigo \emph{et al}. decreased in performance when the number of robots involved in the swarm was changed from the values used to evolve the solutions, and when a more accurate physical model was used in the simulations.

The common hope of users of genetic algorithms is that they can reduce the complexity of directly specifying the task to the (hopefully lower) complexity of describing the results in the fitness function.
Rather than describing how to solve a problem, one simply describes what the solution would look like. 
Unfortunately, the reduction in hands-on time spent programming frequently comes at the expense of time spent waiting for the system to converge, or determining why it converged on a problematic solution. 
One attempt to evolve aggregation controllers had a fitness function which allowed the evolved motion strategy to acquire a high fitness by spinning in place.
The ad hoc iterative process of creating emergent behaviors is replaced by an ad hoc iterative process of creating fitness functions.
As a result, developing novel behavior in the field by converting user specifications of the behavior of the swarm into a fitness function for a genetic algorithm is unlikely to yield results in a timely manner. 
However, individually evolved primitives could be saved in a library of primitives for use by a higher-level compositional approach. 
Such a library could take advantage of the possible overfitting of GA evolution by storing primitives intentionally overfitted to specific situations and robots, and using the best matches. 
For example, a controller that aggregates small-scale UAVs outdoors is likely quite different from one that aggregates medium-scale wheeled robots indoors, even though they are both aggregation controllers. 

\subsubsection{Domain-Specific Languages for Swarm Robotics}

Proto and other programming languages for amorphous computers provide abstractions for computation performed on homogeneous spatially-distributed computing nodes, but do not generally support motion of nodes within the space. 
Other versions of tuple-space based amorphous computation include motion of the agents, but do not explicitly support heterogeneity \cite{viroli2012linda}.

The Voltron programming language provides what its authors describe as ``team-level programming'' for autonomous drones \cite{mottola2014team}.
This level of programming is distinct from drone-level programming, where specific instructions are provided to each drone, and swarm programming, where each drone has the same instructions and operates without communication with other drones.
Voltron programs consist of sensing tasks that are subject to space and time constraints, so the language does not permit the user to specify direct interactions between drones. 
This leaves out activities beyond sensing that may be useful for swarm robots, such as patrolling an area or collaboratively moving an object.
Additionally, Voltron is based on the assumptions that drones have global localization, synchronous clocks, and reliable inter-unit communication.   

Karma provides a programming framework for micro-aerial vehicles with minimal localization and no communication in the field \cite{dantu2011programming}.
The framework allows the composition of behaviors described at the level of individual robots. 
Rather than each robot performing a set of behaviors in response to input from its sensors, each robot is tasked by a central ``hive'' to perform a single behavior, such as performing a sensor sweep of an area. 
The hive collects data from robots as they return to the hive, and updates a central data store, which includes both the sensor information from the individual robots and spatial information about the sensor information. 
The hive then assigns activities to robots based on rules that use the central data store to determine which activities should be performed. 
As a result, while the individual robots are autonomous and not in communication with the hive while operating, the hive maintains a central data store that is used to guide the future behaviors of the swarm. 
This model does permit a form of interaction between swarm members, in that information from one swarm member can inform the behavior of another member, but it does not permit dynamic collaboration between swarm members while they are operating away from the hive. 

Meld is a programming language for robot ensembles, which are composed of individual modular robots \cite{ashley2007meld}. 
However, many of the problems facing an ensemble of modules are the same as those facing a swarm, such as determining the overall goal, moving to proper positions, and detecting when the goal has been acheived. 
Meld is written in terms of facts and rules. 
Facts describe things such as adjacency between robots and location of robots. 
Rules are applied to facts, resulting in the generation of new facts. 
By including rules that generate facts which describe motor actions, the application of rules to the known state of the system can create a ``to-do list'' of actions for individual members of the system to perform. 
Rules are said to ``prove'' facts, and when no further facts can be proven, the system has arrived in a final state. 
Since facts that alter the state of the world can make previous facts false, the set of facts available must be periodically purged of facts that no longer hold. 
The authors of Meld point out that logic programming, of which Meld is an example, is poor at representing state beyond what can be computed as a consequence of the base facts. 
However, it is also claimed that the use of aggregates, which compute results based on the provable facts, can be used to store state about the system, avoiding this restriction. 

Buzz is a programming language and a virtual machine (VM) to run it on that is designed for programming swarm robots \cite{PinciroliLB15}. 
Each robot is assumed to be running the same bytecode on the Buzz VM (BVM). 
Buzz is also based on the assumption that robots exchange information in a situated manner, with any robot that receives a communication also being able to estimate the relative location of the source of that communication.
In order to support programming for swarms, Buzz treats the swarm and virtual stigmergy as first-class objects in the language. 
Swarms in Buzz provide an abstraction for a group of robots that allows the programmer to have every robot in the group execute a function, as well as dynamically create and disband groups. 
Virtual stigmergy provides a global, distributed data store for the swarm. 
The implementation of the virtual stigmergy structure allows the robots to maintain relatively up-to-date versions of the values stored in the data store, and to refresh them after recovering from failures of network connectivity.
Buzz also provides a convenient abstraction of the neighbours of the robot executing the program, which usually consists of all robots within communication range. 
Using the neighbours abstraction, the robot can, for instance, query all its neighbours about the value of a sensor precept at their location, in order to build a local map of the intensity of the sensed quantity. 

\subsection{Program Generation}
Automatic Synthesis of Controllers for Distributed Assembly and Formation Forming
Eric Klavins
	Controlled self-assembly of intelligent parts
	Controllers syntehsized from some sort of description of the higher-level form
	Forms forms in space, but has no notion of position within that space
		So can't really go to a goal location
	Focused on tree-graph structures
	So, very close to being scooped maybe, but different enough 
		He does call the controller synthesizer a compiler, though

Translating Temporal Logic to Controller Specifications
Georgios E. Fainekos, Savvas G. Loizou and George J. Pappas
	LTL -> Hybrid automaton -> control spec
	Control specification is probably for movement, continuious moves despite discrete states
	Event-based semantics for LTL can't distingish events that must hold at an instant from those that must hold over a time period
	Combines continuious flows (vector field) with discrete control transitions
	I think this is making some assumptions of a pure holonomic robot on the ideal plane
	
Robot Creation from Functional Specifications 
Ankur M. Mehta, Joseph DelPreto, Kai Weng Wong, Scott Hamill, Hadas Kress-Gazit, and Daniela Rus
	3D prints the entire robot from the specification, as well as synthesizing the controller for it
	Still has binary sensors in the form of prepositions such as "seeObject\_d" where d is an object
		This blows up because the number of possible states is 2$^cardnality(objects)$
	Maps propositions to components, so the actuations the robot can take and the senses it needs get added to the robot
	This is actually a pretty cool argument for heterogeneity, in that the software can design the robot for you and add it to the swarm

An End-to-End System for Accomplishing Tasks with Modular Robots
Gangyuan Jing, Tarik Tosun, Mark Yim, Hadas Kress-Gazit
	V similar to "An Integrated System for Perception-Driven Autonomy with Modular Robots"
	LTL controller isn't specified in terms of forms, but in terms of desired behaviors
	Properties of robot action and of environment dictate selected forms
	
From High-level Task Specification to Robot Operating System (ROS) Implementation
Kai Weng Wong and Hadas Kress-Gazit
	Automatically suggests possible fixes for mapping from controller to ROS
	"While many of these approaches have been demonstrated on physical robots, the low-level integration of the provably-correct, typically symbolic, controller with the hardware is usually a manual and robot-specific ad-hoc process"
		MURI integration 
	Assumes single rosmaster, and no node launches another node as part of execution
		Pretty reasonable
	Talks about a boolean person detector
		This stuff frequently gets really noisy, people can be hard to detect
		Although hitting a person you fail to detect is provably correct, it's not great
	Permits easy integration of provably correct controllers to ROS
		Doesn't deal with the problems with provably correct controllers
			Perfect information
			Perfect sensors/actuators
			
Decentralized Control of Robotic Swarms from High-Level Temporal Logic Specifications
Salar Moarref and Hadas Kress-Gazit
	High level temporal logic to controllers for safe nav of area
	Imperfect synchronization
	Again with the known map
	This paper assumes static and known environment, but the approach used doesn't rule out reaction to dynamic environment

Symbolic Planning and Control of Robot Motion
BY CALIN BELTA, ANTONIO BICCHI, MAGNUS EGERSTEDT, EMILIO FRAZZOLI, ERIC KLAVINS, AND GEORGE J. PAPPAS
	Shows that Klavins, among others was interested in the general problem
	And also that it was considered open as of March 2007
	Gets into LTL descriptions of environment to synthesize controllers
		Points out that "if a solution is found, it might still exist"
		Doesn't work well with incomplete knowledge
		Doesn't work well with dynamic worlds, as they imply incomplete knowledge
	Symbolic approaches make assumptions about holonomic drive
		car-style steering complicates things
	Some approaches work by minimizing/collecting possible control into small segments (motion primitives) and composing those
		Trades completeness/optimiaility for conciseness
		Want to capture an idea of how expressive a given motion primive set is
	Embedded graph grammars as a protocol for controlling robots
		Robots are nodes and have states, communications are edges and change states
		Transitions between states change robot behavior
		Robots update their state based on their surroundings/messages from other robots

Reactive mission and motion planning with deadlock resolution avoiding dynamic obstacles
Javier Alonso-Mora, Jonathan A. DeCastro, Vasumathi Raman, Daniela Rus, Hadas Kress-Gazit
	Small robot team, mission specification written in LTL
	Automaton plus motion planner to avoid moving obstacles
	Scalable with the number of moving obstacles
	Some assumptions about atomic actions and perfect sensing
		Which, sadly, don't really hold
	This system works as long as the obstacles are not intentionally adversarial
		By which they mean "won't block the robot forever"
	Doesn't attempt to model agents at build time
		Modifies plans at runtime instead (to avoid state explosion related to agent count)
	Still for fixed maps, though
	Assumes moving obstacles: 
		Maintain constant velocity during planning (to project motion)
		or try to avoid collisions using the same algorithm as the robots
		are not out to get the robots
	Certifies solution about things like doors
		"This task will complete if the door is opened eventually"
	Has an exponential blow-up in number of robots in team
		Single robot has 16 revisions
		Two robots have 1306 revisions
		Optimization can bring this down some, but still exponential

An Integrated System for Perception-Driven Autonomy with Modular Robots
Jonathan Daudelin, Gangyuan Jing, Tarik Tosun, Mark Yim, Hadas Kress-Gazit, and Mark Campbell
	Modular robot that reconfigures in response to previously unknown environment
		But known task types, classifies task as discovered and shapeshifts to match
	Does permit exploring and figuring things out about the environment
	Sensitive to failure of modules
	
\section{Why not simulate everything}

Hobby robotics is very underrepresented in the scientific literature, searches on google scholar didn't turn up much about the platforms used
	- Could try screen-scraping instructables or something
	- Seems like more trouble/time than it's worth

Why use real robots?

Brooks in Artifical Life and Real Robots: "unfortunately there is a vast difference (which is not appreciated by people who have not used real robots) between simulated robots and physical robots and their dynamics of interaction with the environment"
	- Is this still the case? This was in the '90s
	- This paper is REALLY precient
		- evolvable controllers for real robots
		- Golem project evolving bodies/controllers together
	"Without regular validation on real robots, there is a great danger that much effort will go into solving problems that simply do not come up in the real world with a physical robot"
	"There is a real danger (in fact, a near certainty) that programs which work well on simulated robots will completely fail on real robots because of the differences in real world sensing and actuation - it is very hard to simulate the actual dynamics of the real world."

	Sensors are uncertain
		It seems like I have virtual sensors, and so I'm way more certain. 
		This is not the case, I have physical sensors, they're just not the physical sensors that they are presented as. 
		Sonar is not a ruler, but we use it to measure distance. 
		A camera is not a laser, but it's still a physical optical sensor being used to measure distance
			Could test noise of camera based fake laser vs real lasers, characterize difference between multiple sensor readings of same thing

	Nap of carpet can affect odometry
		- which some people tried to work around with nap sensors
			- which is clearly insane, but has kind of a chicken-and-egg feel to it
		- this isn't the kind of thing simulators even attempt to do
			- combinatorical disaster (how many kinds of carpet are there?)

Tim Smithers, "On Why Better Robots Make it Harder"
	The idea that the variation in real robot behavior will go away if the robots are "better" as a justification for simulation
	Better-made components allowing hunting where worse ones provided damping (steam engine governor example)
	Don't view robots as measuring with sensors and reasoning about results
		Sensors act as filters whose output drives internal robot dynamics

Simulation Tools for Model-Based Robotics: Comparison of Bullet, Havok, MuJoCo, ODE and PhysX
Tom Erez, Yuval Tassa and Emanuel Todorov.
	Most simulators aimed at visual plausibility, for video games
	Contact dynamics are particularly hard (NP-Hard, in fact)
		workaround is approximation, which raises concerns about accuracy
	Simulations deviate due to
		numerical instability
		model errors
			e.g. Sims and the evolved robots that push themselves around
	Showed speed accuracy tradeoff, no simulator stood out over all others
	ODE does coriolis forces, which is how Sims bots cheated 
	PhysX and Havoc don't, so that's still possible

	So overall, yes, the situation Brooks described still holds

An Overview about Simulation and Emulation in Robotics
Michael Reckhaus, Nico Hochgeschwender, Jan Paulus, Azamat Shakhimardanov and Gerhard K. Kraetzschmar
	Simulation went out of fashion when people started getting real robots
	Computers got better and games drive for realism pushed development
	Proposes a lot of use cases and reasons to use simulation
		Which are interestingly orthogonal to the reasons not to use it
			One doesn't counter the other, it's just "here are the good parts, here are the bad parts"

Simulation in robotics
Leon Zlajpah
	Overview of tools

Noise and The Reality Gap: The Use of Simulation in Evolutionary Robotics
Nick Jakobi and Phil Husbands and Inman Harvey
	Having too much noise in evolved controller simulation environment can result in controllers that rely on there being more noise than there is in the real world
	
\section{Why Heterogeneity?}
Defence of heterogeneity
	Keeps individual robots from becoming complex by attempting to do everything
		Spreads ability over swarm, if there are multiple gripper robots and multiple scout robots, losing one isn't as problematic
		Swarmanoid
	Practical concern with toys
		May build swarm in stages, change platforms based on ability or cost
		Tries to move platform-specific calculations closer to the hardware, leaving higher-level concerns able to be interchanged
	Impact on compilation
		minimization of mobility heterogeneity means that current translator does not use robot capability in translation
		Could be extended to ensure that tasks were assigned to robots that can perform them
			Task assignment is a well-studied problem in the literature
			Makes the assumption that the robots the user selects have the capability to do what they command
				Error back to user if robot can't do it? Select closest robot that can?
	Heterogeniety is a good model of real-world situations
		Swarm members of different capability, e.g. family groups of pack animals, young-old, ill members
		Military convoy with different vehicles (HUMV, Tank, Jeep)
		Human/robot teams 

	