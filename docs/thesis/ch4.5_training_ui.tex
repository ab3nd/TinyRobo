% !TeX root = ams_thesis.tex
\chapter{UI Design for High Training}
\thispagestyle{fancy}

The work that this study built on was motivated in part by the idea of using robots in search and rescue (SAR) operations, to explore damaged buildings, enter unsafe areas, and provide additional senses and coverage while reducing risk to first responders \citep{micire2010multi}.
First responders are trained on the equipment they use, but such training may be infrequent, and so complex control systems may not be remembered when the time comes to use the equipment. 
Because a SAR situation may develop rapidly, there is little or no time available for retraining, and so use with first responders prioritizes easy learning of the interface and use with minimal training over depth of the possible commands.  

However, SAR is not the only domain in which swarm robots might be used, and so considering only the gestures which maximize the ability of untrained users ignores domains where it is possible to train the users extensively. 
In such a context, adding complexity to the gesture set could allow for additional expressiveness in the interface.


\section{On-line Training}

Because user experience of video games does have some effect on their use of the user interface described in this work, it is interesting to examine other aspects of video games, and how they might be applied to user interface design for swarm robotics. 

The majority of the work on video games and learning is centered around the outcomes of games developed to be educational, rather than on how games that are intended to be purely entertaining still educate their users in the use of the game itself. 
Tony Hawk pro skater and such, game UI includes training elements. User can do basics, as their proficiency increases, system builds additional skills. 

If the system detects repetition of a behavior, suggest more efficient methods (single moves vs select and move)

It has been argued that the use of game-like user interfaces is inappropriate for the design of applications, but better suited to training material\citep{thomas1994games}. The basis for using a game-like user interface for an application is that games are engaging, and as a result, using a similar interaction paradigm should yield an engaging application. 
However, the dissimilarities in user motivation make this approach less useful. 
Games are played for the sake of their own play, rather than to accomplish a task outside of the game. 
Challenge and complexity are controlled in the game to make the play more rewarding, but extending this to applications for external tasks means making the tasks more difficult than they have to be, which will be a difficult proposition for end users to accept. 
Additionally, work and play are culturally strongly divided, and making one like the other will be resisted by users. 
Thomas instead proposes that computer-based training materials are a better target for game-like user interfaces. 
Like games, training materials are intended for high involvement over short periods, and are interacted with for themselves, rather than as part of the completion of an external task. 
