% !TeX root = ams_thesis.tex
\chapter{Swarm Robot System Development}
\thispagestyle{fancy}

Swarm robot platforms tend to fall into one of two groups, from a hardware perspective. 
The first group uses microcontrollers and very limited onboard computation, but is small and relatively cheap.
This group includes Alice, Jasmine, AmIR, and the other tabletop systems. 
Due to their limited computation, these systems do not generally support complex algorithms such as vision processing. 
The second group use more powerful computers, but at a significant cost in weight, power consumption, and financial outlay.

The robot described in this work is intended to occupy a theoretical ``sweet spot'' at the high end of the tabletop swarms or the low end of the room-scale swarms, depending on how large of a mobility platform is used. 
As a result, if it is configured for tabletop operation, the system can be used with a minimum of available space. 
If, on the other hand, it is configured for room-scale operations, the system can be tested in natural or naturalistic human environments. 

\section{Hardware Platform}

The robot swarm developed for this work consists of a hardware module for controlling two motors of a toy, such as a small RC car, for mobility. 
The reasons for choosing this hardware design are explained in more detail below, but the overall intent is to have an inexpensive platform available for swarm research, without having to rely on any particular group of swarm robotics researchers starting and maintaining a side business supporting and selling robots.
Duplication of software and other digital artifacts is trivial, so constructing a duplicate of the hardware becomes the primary difficulty. 
The use of toys for the mechanical components of the robots was intended to reduce the difficulty of constructing the hardware. 
If researchers are not to be expected to become entrepreneurs, they should also not be expected to become expert machine tool operators.
The hardware resulting from this work is designed so that it can be duplicated by a researcher using common tools, and possessed of no more than hobby-level familiarity with electronic hardware.

\begin{figure}
	\centering
	\includegraphics[width=0.9\textwidth]{../robots_with_ruler_small}
	\caption{Toys with controller boards and batteries mounted. The spider has a two-motor holonomic drive, the tank uses differential drive, and the car is Ackerman drive.}
\end{figure}

In order to be both heterogeneous and inexpensive, the robots used for this work were initially designed to be constructed by developing a modular control hardware platform that can be attached to children's toys. 
The controller module was designed to be used as a replacement for the control electronics of children's toys, similar to the Spider-Bots developed by Laird, Price, and Raptis, or Bergbreiter's COTSBots \citep{lairdspider, bergbreiter2003cotsbots}.
However, unlike the Spider-Bots and COTSBots, this platform did not specify a particular toy chassis to use for mobility. 
Most children's toys use either one motor with a mechanical linkage to cause the toy to turn when the motor is reversed, or two motors.
Two-motor toys frequently use either differential steering or have one motor provide drive power and the other provide steering. 
All of these toys can be controlled by the hardware described in this work. 

The robots are intended to be heterogeneous, partly because of the advantages of heterogeneity in a swarm, and partly because toy supplies are unreliable.
While toys in the general case are expected to remain available, a particular line of toys might be discontinued or a modified version released. 
%The software framework in development to support the robots is based on ROS, and so allows modular replacement of the control algorithms used to convert desired motion of the robot into drive signals for the motors. 

\begin{table}
	\begin{tabular}{l l r r r r r}
		Name & Value & Cost & Cost & Count & Subtotal & Subtotal\\
		& & (1) & (100) & & (1) & (100)\\
		\hline
		Battery & 3.7V & 3.41 & 1.50 & 1 & 3.41 & 1.50\\
		C1, C2 & 4.7uF & 0.50 & 0.20 & 2 & 1.00 & 0.40\\
		C3 & 0.1uF & 0.10 & 0.02 & 1 & 0.10 & 0.02\\
		C4 & 10uF & 0.19 & 0.06 & 1 & 0.19 & 0.06\\
		Charge IC & MCP73831 & 0.59 & 0.44 & 1 & 0.59 & 0.44\\
		Charge LED & Green & 0.54 & 0.30 & 1 & 0.54 & 0.30\\
		Diode & GF1A & 0.51 & 0.23 & 1 & 0.51 & 0.23\\
		Motor driver & DRV8830 & 2.27 & 1.64 & 2 & 4.54 & 3.28\\
		Header & 6-pin  & 0.52 & 0.37 & 1 & 0.52 & 0.37\\
		PCB &  & 3.30 & 0.79 & 1 & 3.30 & 0.79\\
		R1 & 470 ohm & 0.10 & 0.02 & 1 & 0.10 & 0.02\\
		R2 & 2k & 0.10 & 0.02 & 1 & 0.10 & 0.02\\
		R3, R4 & 0.22 ohm & 0.46 & 0.13 & 2 & 0.92 & 0.26\\
		R5, R8-12 & 10k & 0.10 & 0.01 & 6 & 0.60 & 0.06\\
		R6, R7 & 1k & 0.10 & 0.01 & 2 & 0.20 & 0.02\\
		Switch & 410-2016 & 0.91 & 0.72 & 1 & 0.91 & 0.72\\
		Thermal Fuse & 0ZB0050FF2G1 & 0.13 & 0.10 & 1 & 0.13 & 0.10\\
		V Regulator & MIC5265 & 1.40 & 1.06 & 1 & 1.40 & 1.06\\
		Wifi & ESP-8266-03 & 4.32 & 2.25 & 1 & 4.32 & 2.25\\
		\hline
		\multicolumn{2}{l}{Total Cost for One Robot} &  &  &  & 23.38 & 11.9\\
	\end{tabular}
	\caption{Prices in US Dollars for TinyRobo components.}
	\label{tab:per_robot_parts_cost}
\end{table}

The mobility platforms used for the existing TinyRobo swarm cost 12-20 dollars in single quantities, putting the total cost for a single robot at \$35-45.
Where bulk ordering is available, the cost of 100 mobility platforms is \$8 per unit, reducing the per-unit cost of a 100-member TinyRobo swarm to \$20. 
This is roughly in line with the parts cost of the Kilobot, which is \$15 \citep{rubenstein2014kilobot}.

However, this should not be taken to mean that the TinyRobo platform is a competitor with the Kilobots. 
Kilobots were designed to have scalable interactions, which is to say that programming, charging, and even turning on the Kilobots does not take more time as the number of robots increases. 
To have programming take constant time, the Kilobots are programmed in parallel using an infrared broadcast device that illuminates the entire swarm at once. 
To charge together, the Kilobots have one charging contact on their legs, and the other on a leaf spring on their tops. 
By sandwiching the robots between two conductive plates, the entire swarm can be charged at once.
Finally, to all be turned on quickly, the Kilobots never turn off.  
Instead, they enter a low power sleep mode, and wake occasionally to check for an infrared signal to become fully active. 
This last attribute makes it very difficult to have a robot that has both wifi and a very low power sleep mode. 
The wifi module used in the TinyRobo platform consumes 15mA in its highest-power sleep mode, and 20$\mu$A in its lowest power mode. 
Unfortunately, only the highest power sleep mode can remain connected to an access point and receive a wake-up command, and so it will deplete the 750mAH battery used in the TinyRobos in just over two days. 
In contrast, a Kilobot can sleep for 3 months. 

The processor of the controller is an ESP-8266 wifi module.
The ESP-8266 costs approximately \$3-5, and contains both a wireless interface and a micro controller that can be programmed from a variety of programming environments and languages, including Lua and the Arduino variant of C/C++. 
The ESP-8266 module is based on the ESP-8266 IC, made by Expressif Systems. The IC itself has an 80Mhz Tensilica Xtensa L106 processor with 64kB of instruction memory and 96kB of data RAM. The modules come equipped with 512kB to 16MB of flash memory for program storage, and some combination of the 16 GPIO lines of the IC available for use. 
The ESP-8266 is available in several form factors, each designated by a different suffix. 
The version selected is the ESP-8266-03, which offers more GPIO pins than most other versions, and includes an internal antenna.

In addition to 802.11 b/g/n WiFi, the ESP-8266 supports a variety of serial protocols, including a UART, I$^2$C, SPI. 
The I$^2$C interface is used on the board to connect to two DRV8830 motor driver ICs by Texas Instruments. 
The DRV8830 provides 1A of drive current.
Experimental tests with 8 different toys indicate that small toys draw well under 1A while moving freely, and peak around 2A when the motors are stalled. 
The tested toys include 3 insect-styled walkers, 3 wheeled vehicles (2 differential drive, 1 Ackerman steering), 1 toy helicopter, and 1 toy quadcopter.
The DRV8830 provides overcurrent limiting, so a stall condition or short circuit of the motor leads will disable the motor drive, but not damage the DRV8830. 

The control module also provides connections for a 3.7V lithium-ion battery pack, as well as charge control circuitry for the battery. 
The charge controller allows the robot to be charged from the same USB connection that is used to change the programming of the ESP-8266. 
Reset and entry into programming mode is controlled by a separate USB-to-serial adapter board, the Sparkfun BOB-11736.
Moving this functionality to the adapter board reduces the size and cost of the control module. 

\subsection{Toy Compatibility} \label{section:Toy_Compatibility}

Children's toys normally use inexpensive brushed DC motors in their construction. 
These motors have not been the subject of extensive study, as they are commodity parts. 
However, it is useful to quantify their behavior to some extent, to determine which kinds of toys can be used with the controller. 

Two common types of motors found in children's toys are the RE and FA series of motors produced by Mabuchi Motor, or imitations of these motors produced by other companies. 
These motors use simple metal brushes and are constructed to be inexpensive, rather than precise. 
The intended voltage range of the motors varies with different winding types, but according to datasheets available from Mabuchi Motor, the voltage ranges and current draws for motors in this range are as shown in Table \ref{tab:properBrandedMotors}.

\begin{table}
	\begin{tabular}{l r r r r}
		Model & Voltage & No Load Current & Max Efficiency & Stall Current\\
		\hline
		RE-140RA-2270 & 1.5-3 & 0.21A & 0.66 & 2.1A\\
		RE-140RA-18100 & 1.5-3 & 0.13A & 0.37 & 1.07A\\
		RE-140RA-12240 & 3-6 & 0.05A & 0.14 & 0.39A\\
		FA-130RA-2270 & 1.5-3 & 0.2A & 0.66 & 2.2A\\
		FA-130RA-18100 & 1.5-3 & 0.15A & 0.56 & 2.1A\\
		FA-130RA-14150 & 1.5-4.5 & 0.11A & 0.31 & 0.9A\\
	\end{tabular}
	\caption{Current draw for Mabuchi-branded motors.}
	\label{tab:properBrandedMotors}
\end{table}

These motors have a volume of around 2cm$^3$.
For smaller toys, coreless motors are more common. 
The values in Table \ref{tab:coreless} were measured from six of the toys used in constructing the swarm.
The measurements from the toy helicopter and toy quadcopter are included for comparison.
While the board can supply sufficient current to control all of these toys, it has not been tested in flying platforms.

\begin{table}
	\begin{tabular}{l r r}
		Motor number & No-Load Current & Stall Current (measured)\\
		\hline 
		Hexbug brand mini spider & 0.03A & 0.13A (see caption) \\
		Hexbug brand 6-legged insect & 0.06A & 0.25A \\
		Miniature toy RC car & 0.21A & 0.8A \\
		Miniature toy RC insect & 0.19A & 1.13A \\
		Miniature toy RC vehicle & 0.37A & 0.8A \\
		Miniature toy RC vehicle & 0.06A & 0.74A \\
		Toy helicopter & 0.07A & 1.12A \\
		Toy quadcopter & 0.74A & 1.99A \\
	\end{tabular}
	\caption{No-load and stall current for coreless DC micromotors. Measurements were performed at 3V supply voltage. The Hexbug mini spider includes a slip clutch, so attempting to stall the motor by holding the toy does not prevent the motor from turning.}
	\label{tab:coreless}
\end{table}


\subsection{Potential for Expansion} \label{section:Potential_for_Expansion}

The current design for the robots does not include sensors as a cost-saving decision. 
However, the communication between the ESP-8266 and the motor drivers uses the industry standard I2C bus serial interface. 
Due to the non-proprietary nature of this interface standard, it has been widely adopted, and many sensors are available to connect to an I2C bus. 
For example, Vishay Semiconductor makes the VCNL3020, an infrared proximity sensor with a 200mm range. 
If greater range is required, The ST Microelectronics VL53L0X Time-of-flight (ToF) laser ranger and gesture sensor provides a 2M range and 1D gesture sensing in a 4.4mm x 2.4mm package. 
As of this writing, the VCNL3020 is \$3.44 and the VL53L0X costs \$6.28 in single quantities.
These prices are reduced significantly when buying components in bulk, but because they increase the cost, size, and power draw of the hardware, they have not yet been integrated with this platform. 
Numerous multichannel ADC ICs with I2C interfaces are also available, which permits the addition of analog sensors to the platform. 

As a thought experiment, the cost of adding a 6-direction IR transmission and reception board to the TinyRobo platform was explored. 
This is not a finished design, but as much of the cost in a device of this type is in the semiconductors and PCB, it gives an estimate of the cost. 
The IR receiver selected would have been the TSOP5700TR, as used in the E-Puck Range and Bearing sensor board, but it is listed as obsolete by the manufacturer, Vishay Semiconductors. 
The modern replacement appears to be the TSMP6000.
The IR LED was also selected to match the E-Puck Range and Bearing board.
The AND gate is intended for use as in the Colias swarm robots \citep{arvin2009development}.
By providing a clock signal gated through the AND gate by the data signal, the IR LEDs can transmit coded data to other robots. 
If coding is not required, as in rangefinding, the clock and data signals can be varied to turn the LED on and off. 
Colias implements rangefinding by detecting the intensity of received IR light from either reflections of IR emitted by the robot, or the strength of IR messages arriving from other roobts. 

\begin{table}
	\begin{tabular}{l r r r r r }
		Part number & Cost (1) & Cost(100) & Count & Subtotal(1) & Subtotal (100)\\
		\hline 
		ATMega168 & 1.40 & 1.12 & 1 & 1.40 & 1.12  \\
		TSMP6000 & 1.81 & 1.03 & 6 & 10.86 & 6.18  \\
		SFH4255  & 1.06 & 0.59 & 6 & 6.36 & 3.54 \\
		SN74ACT08DR & 0.51 & 0.32 & 2 & 1.02 & 0.64 \\
		PCB & 3.30 & 0.79 & 1 & 3.30 & 0.79\\
		\hline 
		Totals & & & & 22.94 & 12.27\\
	\end{tabular}
	\caption{Semiconductors for a simple IR communication ring, and their prices, in US Dollars. The PCB is the same size and type as the TinyRobo controller, and so has the same cost.}
	\label{tab:ir_ranger_board}
\end{table}

Adding a 6-direction range and bearing board to the TinyRobo system could be expected to cost in the neighborhood of \$12-22 USD, depending on the quantity of boards produced. 

\subsection {Firmware}

The current version of the robots' firmware is developed in the open-source Arduino development environment.
Arduino programs are written in a dialect of C++. 

Every robot runs the same firmware. 
The firmware listens for connections on port 4321 for TCP/IP packets containing one of two types of messages. 
Messages starting with a 0x51 byte (ASCII `Q') cause the firmware to respond with a message containing the ASCII string ``TinyRobo". 
This function allows automatic detection of robots on a network by querying all connected systems to determine if they respond in this way. 

Messages starting with a 0x4D byte (ASCII `M') followed by four bytes are motor speed commands.
The firmware interprets the first two bytes as the speed and direction for the first motor, and the second two bytes as speed and direction for the second motor.
The control bytes are converted to a single byte command for the DRV8830 motor driver and transmitted over the I2C bus to set the motor speed.

The DRV8830 driver is a voltage-controlled motor driver. 
It accepts a single-byte command for each motor. 
Bits 7-2 of the byte define the output voltage to be applied to a motor, and the driver attempts to maintain that output voltage.
The valid range of motor voltage commands for the DRV8830 driver is 0x06 to 0x3F, which corresponds to a range of 0.48V to 5.06V in 0.08V increments. 
Because the robot battery is nominally 3.7V, the motor command 0x30 is the highest output available. 
Bits 1 and 0 of the command byte control the polarity of the output voltage, and so the direction of the motor, as per Table \ref{tab:DRV8830_truth}.

\begin{table}
	\centering
	\begin{tabular}{l l l l l}
		Bit 1 & Bit 0 & Out 1 & Out 2 & Function\\
		\hline
		0 & 0 & Z & Z & Coast\\
		0 & 1 & L & H & Reverse\\
		1 & 0 & H & L & Forward\\
		1 & 1 & H & H & Brake\\				
	\end{tabular}
	
	\caption{Truth table for DRV8830 drive direction bits. Coast allows the motor to turn freely. Brake connects the motor leads, resulting in braking using the motor's back-EMF. Z indicates the output is in a high-impedance state}
	\label{tab:DRV8830_truth}
\end{table}

\begin{figure}
	\centering
	\begin{bytefield}[bitheight=\widthof{~Sign~},
		boxformatting={\centering\small}]{8}
		\bitheader[endianness=big]{7,2,0}\\
		\colorbitbox{lightred}{5}{Speed}\colorbitbox{lightgreen}{2}{Dir}	
	\end{bytefield}
	%Can't caption here because this is outside float
	\caption{Layout of bits in motor command byte for DRV8830}
\end{figure}

Once the motor speed is set, the firmware reads the fault bytes from the DRV8830, and sends the motor command and the fault bytes for each motor back to the client over WiFi. 
The client uses the fault bytes to detect overcurrent conditions in the motor drivers and reduce output power. 

The decision to have all of the robots have the same firmware and control the speed of the motors from ROS was made because different toys have different control schemes. 
Toy tanks use differential drive, toy cars have Ackerman steering, and so forth. 
By moving the control to the main computer, the firmware can be kept simple while still allowing researchers to adapt the system to the available toys by modifying the software. 
 
\subsection{Why Heterogeneity?} \label{section:Why_Heterogeneity_}

Heterogeneity is a good model of many real-world systems where members of a group have different capabilities. 
Family groups of pack animals have young and old members, sometimes ill members, and sometimes infant members that cannot participate fully in pack activities. 
In human groups, work is divided according to ability, so a contractor may hire an electrician, a framing carpenter, and a plumber to build a house, to much better effect than attempting to do it with a team consisting entirely of plumbers. 

Another use of heterogeneity in swarms is to prevent individual robots from becoming overly complex by sparing them from having to be capable of doing everything. 
The presence of multiple robots with a given ability in a swarm strikes a compromise between all robots having that ability (and being complex and expensive) and only one robot having that ability (and so providing a single point of failure). 
Perhaps the most impressive recent demonstration of a highly heterogeneous swarm is the Swarmanoid project's video ``Swarmanoid: The Movie", in which three different kinds of medium-sized robots cooperate to retrieve a book from a shelf \citep{oʼgrady2011swarmanoid}.
The movie explicitly mentions that one sub-team of robots (two mobile robots and one gripper robot) is positioned as back-up, in case the first sub-team fails. 

Beyond the possible utility of robots with multiple abilities, the swarm design presented in this work was intended to be heterogeneous as a matter of convenience of implementation. 
As toys go out of production and are replaced by others, it may not be possible to continue to operate the swarm on an entirely homogeneous mobility platform. 
Because of this possibility, the software infrastructure tries to keep platform-specific calculations in a single module, and allow the rest of the system to operate using standard ROS messages. 
At present, these calculations only consist of conversion of ROS twist messages, which contain rotational and angular velocities in 3 axes each, into motor speed commands for the robot, which consist of a speed and direction for up to two motors. 

Handling the motion of the robots in this way means that the heterogeneity of the mobility platforms has a minimal impact on the conversion of user gestures into programs. 
However, as a direction for future work, it will become increasingly important to consider mobility as an aspect of program generation. 
For example, if the system is extended to include UAVs, and the user directs the robots to the center of a lake as part of a task, only the UAVs can be reasonably expected to reach the location undamaged. 
The system could be extended with some capacity for reasoning about the task environment, to determine how the capabilities of the robots interact with that environment. 
Such an augmented system could then refuse to direct ground robots into water, and report if there are not a sufficient number of aerial robots to perform the task over the lake. 
It could also permute task assignments based on robot capabilities in order to meet other goals, such as minimizing the number of robots used or maximizing available robot battery life. 

\section{Swarm Robot Software Framework} \label{section:Swarm_Robot_Software_Framework}

The individual robots being developed for this research have minimal sensing capacity and relatively weak processors. 
The majority of the processing is performed on a host computer running the ROS software framework. 
Each robot's processor is mostly concerned with controlling the motors of the robot. 
The structure of the software framework is such that as available processing power on each individual robot increases, more of the processing can be handled locally, without changing the overall design of the system.

\begin{figure}
	\centering
	\includegraphics[width=0.45\textwidth]{../swarm_table_no_bg_small}
	\includegraphics[width=0.45\textwidth]{../overhead_view}
	\caption{The image on the left shows the swarm arena. The top-down camera is mounted on the crossbar at the top. The image on the right shows the camera view before ROS image rectification removes barrel distortion.} 
	\label{fig:robot_arena}
\end{figure}

The central computer has a top-down camera over the ``arena'' the robots are active in. 
Each robot has an AprilTag on top of it, so that the central computer can localize them within the arena \citep{olson2011tags}. 
The central computer uses the location information to create ``virtual sensors'' for each robot. 
Since the central computer knows the location of each robot, the relevant information can be sent to each robot's control process as if it were coming from a sensor on the robot. 
For example, a range and bearing sensor that allows each robot to detect the distance and angle of the nearby robots is simple to implement in software. 
Range and bearing sensor functionality is available in hardware on E-pucks and Marxbots, but since each robot must be equipped with it, the cost scales linearly with the number of robots to equip.  
It is possible to calculate the odometry for individual robots by watching the change in position in their tags over time. 
The calculated odometry could then be published as a ROS topic, just like odometry collected from e.g. wheel encoders. 
The virtual sensors can also be configured to emulate error conditions such as noisy sensors, failed sensors, degraded localization, and so forth.
Virtual parameter tweaking allows fine-grained testing of the behavior of algorithms under imperfect conditions, and the response of human users to unreliability in the swarm. 

%Graphic of the software environment as a whole
\begin{figure}[h]
	\centering
	\digraph[scale=0.6]{Framework}{
	
	graph[nodesep=0.5];
	
	subgraph clusterRobot1 {
		motor[shape=box; label="Motor Driver"];
		robotCode[label=<Robot <br/> Firmware>];
		robotCode -> motor;
		label="Robot 1";
		shape=box;
	}
	
	subgraph clusterRobot2 {
		motor2[shape=box; label="Motor Driver"];
		robotCode2[label=<Robot <br/> Firmware>];
		robotCode2 -> motor2;
		label="Robot 2";
 		shape=box;
	}
	
	subgraph clusterRobot3 {
		motor3[shape=box; label="Motor Driver"];
		robotCode3[label=<Robot <br/> Firmware>];
		robotCode3 -> motor3;
		label="Robot 3";
		shape=box;

	}
	
	subgraph clusterRobotN {
		motorN[shape=box; label="Motor Driver"];
		robotCodeN[label=<Robot <br/> Firmware>];
		robotCodeN -> motorN;
		label="Robot N";
		shape=box;
	}
	
	subgraph clusterComp {
		concentrate=true;
		label="Control Computer";
		{rank=source;
			vrSense [label="Virtual Sensors"];
			vrNet [label="Virtual Network"];
			worldModel [label="World Model"];
			worldModel -> vrSense;
			worldModel -> vrNet;			
		}
		rp1 [label=<Robot <br/> Process 1>];
		rp2 [label=<Robot <br/> Process 2>];
		rp3 [label=<Robot <br/> Process 3>];
		rpN [label=<Robot <br/> Process N>];
		vrNet -> {rp1, rp2, rp3, rpN} [dir="both"];
		vrSense -> {rp1, rp2, rp3, rpN};	
	}
	
	camera[label=<Overhead<br/>Camera>;shape=box;]
	camera->worldModel;
	
	rp1 -> robotCode [label="WiFi", dir="both"];
	rp2 -> robotCode2 [label="WiFi", dir="both"];
	rp3 -> robotCode3 [label="WiFi", dir="both"];
	rpN -> robotCodeN [label="WiFi", dir="both"];
	}
	\caption{Overview of the software framework. Rectangular nodes are hardware, oval nodes are software.}
\end{figure}

Since the robots are reporting to a central server, and the central server also receives the video from the overhead camera, it may appear that this is a highly centralized system. 
However, the central computer provides a framework for implementing a decentralized control scheme on the individual robots. 
Rather than controlling each robot, the central computer maintains a separate process for each robot in the swarm. 
Each of these robot processes only has access to the information that would be available to that robot, based on its physical location, and so acts as a local control program for the robot, but with the full processing resources of the host computer. 
As a result, the individual robots can be small, lightweight, and consume relatively little electrical power, but the system as a whole gives them significant computing power. 
When more powerful and lower power consumption processors become available, more of the processing can be moved from the virtualized robot processors and onto the actual robots, enabling a smooth transition from a simulated decentralized system to a real decentralized system. 

\subsection{Virtual Localization} \label{section:Virtual_Localization}

The AprilTag tracking of the robots provides localization of the robots within a common coordinate frame. 
It should be stressed that while the central computer can localize the robots, both relative to each other and by absolute position within the arena, this information may be withheld from the individual robots, or given to them if required. 
The code virtually operating on the robot may be neither aware of its own position in the world, nor the location of other robots, if the experiment calls for such a lack of information. 

Currently, the AprilTag-based localization is used to implement virtual laser scanners similar to the Sick or Hokyuo brand laser scanners used on larger robots. 
It is also used to limit the range of messages sent between the robots through a virtual network, and to implement range and bearing sensing between robots. 

Range and bearing sensing to other robots is a very important sense for swarm robots, as indicated by its presence in algorithms and frequent implementation in swarm robot hardware. 
As discussed above, Colias and E-Pucks both include IR-based range and bearing sensing. 
The parts cost of a Colias-style six-direction range and bearing sensor is around \$12, while the more sophisticated E-Puck range and bearing extension board is \$397. 
Virtualizing this hardware results in an immense cost savings. 

\subsection{Virtual Laser Scanners} \label{section:Virtual_Laser_Scanners}

The AprilTag localizations and the image of the arena are used to provide virtual laser rangers for each robot. 
The virtual laser ranger consists of two ROS nodes, a service and clients for the service. 
The service is called ``laser\_oracle\_server''. 
It subscribes to the AprilTag detections and the images from the arena overhead camera. 
 
When a client requests a laser scan, the virtual laser service masks the modified arena image with a circle with a radius of the laser range, centered on the robot requesting the scan.
This masking removes all of the objects that are out of range of the laser, and so reduces the time spent calculating the laser scan points. 

Each sample of the laser scan is represented as a line segment, located based on the requested start, stop and inter-measurement angles for the virtual laser scanner. 
Each line segment is checked for intersection with the lines defining the contours of the blue objects in the image. 
As the virtual laser service receives images, it draws a blue dot over the location of every robot. 
This dot provides the outer edge of each robot in the virtual laser scan. 
The approach of using blue objects as obstacles was chosen because if the laser scanner service treats anything blue as an obstacle, then ``walls'' can be created in the arena by making lines of blue masking tape on the arena floor. 
If multiple intersections are found for a line segment, the intersection closest to the robot is used, as the laser would stop after reflecting off an object.
The service then formats the distances to the intersection points as a ROS sensor\_msgs/LaserScan and returns it as the service response to the requesting client. 

The virtual laser clients take the place of the laser driver ROS nodes that would be used to control a real linear laser scanner. 
The laser client is initialized with some parameters, such as the sweep angle and angular resolution of the virtual laser, and polls the laser service regularly. 
As it receives laser scans from the service, it publishes them to a ROS topic in the same manner as a ROS node for a hardware laser. 

The apriltags\_ros node publishes the detected locations of the tags in meters, but the computer vision detection of blue objects in the arena camera image operates in pixels. 
In order to convert from pixels to real-world distances, the apriltags\_ros node was forked and a modified version was created that provides the locations of the tags in pixel as well as real-world coordinates. 
The modified version is available at https://github.com/ab3nd/apriltags\_ros.

\begin{figure}
 	\centering
	\digraph[scale=0.6]{VirtualLaserSystem}{
	
	vls -> vsc [label=<std\string_msgs/Integer&nbsp;&nbsp;&nbsp;&nbsp;>];
	vsc -> vls [label=<sensor\string_msgs/LaserScan>];
	vsc -> sub1 [label=<sensor\string_msgs/LaserScan>];
	cam -> vls [label=<sensor\string_msgs/Image>];
	cam -> atag	[label=<sensor\string_msgs/Image>];
	atag -> vls [label=<apriltags\string_ros/TagDetections>];	
		 
	vls [label="Virtual Laser Service"];
	vsc [label="Virtual Laser Client"];
	atag [label="AprilTag Detector"];
	cam [label="Arena Camera"];
	sub1 [label="Subscriber"];
 	}
	\caption{Data flow in the virtual laser service}
\end{figure}

\subsection{Virtual Networking} \label{section:Virtual_Networking}

If the robots are required to communicate directly with each other, the communication passes through a virtual network.
From the point of view of the robots, messages sent into the virtual network are delivered to other robots as if the messages were sent directly from one robot to another. 
However, all the communication is taking place between processes running on the central computer.
By changing how the messages are delivered by the central system, the virtual network can provide full connectivity, range-limited mesh networking, directional beacons, or other forms of networking. 
The reliability of the network can also be varied, by dropping some messages or otherwise changing them based on information about the robots. 
For example, the likelihood that a message arrives at the robot that it was transmitted to may depend on the distance between the sender and receiver.
Signals that pass through a virtual wall may have a reduced virtual signal strength and range.

 \begin{figure}
 	\centering
	\digraph[scale=0.6]{VirtualNetwork}{
	
	{rank=same atag dist}
	{rank=same tx rx}

	vns -> dist [label="Robot IDs"];
	dist -> vns [label="Distance"];
 	cam -> atag	[label=<sensor\string_msgs/Image>];
	atag -> dist [label=<apriltags\string_ros/TagDetections>];	
	vns -> rx [label="Network Message"];
	tx -> vns [label="Network Message"];	 
	
		
	vns [label="Virtual Network Service"];
	dist [label="Distance Service"];
 	atag [label="AprilTag Detector"];
 	cam [label="Arena Camera"];
	tx [label="Transmitter"];
	rx [label="Receiver"];
 	}
	\caption{Data flow in the virtual network. The virtual network service can take the distance between the transmitting robot and the receiving robot into account when determining if the message is delivered.}
 \end{figure}

\section{Swarm Hardware Results} \label{section:The_Desert_of_the_Real}

Due to mechanical flaws in the toys used as motion platforms in TinyRobo, the robots would sometimes not move as commanded (see Figure \ref{motor-speed-fig}). 
In a single-robot system, transient failures can sometimes be accommodated by repeated effort or replanning. 
However, in a multi-robot system the scale of the system works to offset the reliability of each individual robot. 
If robots have an individual mean time between failures (MTBF), the expected mean time to any failure is the MBTF divided by the number of robots. 
For example, if an individual robot can work for 100 hours between failures, it would be reasonable to expect it to work for at least a day. 
However, if the swarm consists of 1000 such robots, a failure of at least one robot can be expected within 6 minutes. 
Tracking the failure of existing platforms in the field placed the average MBTF of commercial mobile platforms at 24 hours, with indoor robots fairing much better than outdoor ones \citep{carlson2004follow}.


\begin{figure}[t]
	\includegraphics{motion_vs_cmd}
	\centering
	\caption{Commanded velocity (lin\_vel) as opposed to recorded motion (vel). Vel is always positive because it is measured in terms of euclidian distance moved by the center of the AprilTag between successive updates of the tag tracking. 
		Note that while the magnitude of the motion is proportional to the commanded motion, sometimes the robot did not move at all, and when it did move, the recorded velocity is quite noisy. Noise may be removed in software, but mechanical failure cannot.} 
	\label{motor-speed-fig}
\end{figure}

This problem was highlighted by the use of inexpensive toys. 
It is desirable to have a swarm platform be able to run for extended periods, in order to acquire data for experiments. 
Toys, especially cheap ones, are designed for low cost and an MBTF more compatible with the attention span of children than that of researchers. 
The amount of effort devoted to locating and eliminating mechanical problems relative to the runtime of the system was not acceptable.

\subsection{Calibration} \label{section:Calibration}

Children's toys are prone to failure and inaccuracy.
In toys, particularly remote controlled toys, the user acts as a the feedback element of a control system, observing the behavior of the toy and changing their actions as a result. 
If an individual toy, for example, has a bias to turn to the left, the user will learn this and apply a correcting bias.
To extend this to a computerized system requires some form of calibration. 
These calibration steps, combined with appropriate control, such as PID loops, can account for systemic inaccuracies. 
However, the use of toys also introduces some failures that are not consistent or linear. 
For example, the tanks used in some instances of the TinyRobo platform use motors with high speed, but relatively low torque. 
As a result, dirt in the drivetrain near the motor can cause the motor to become difficult to start, but can be removed by the user, or by operation once the motor does start. 
Calibration when the dirt is present means the robot will start very abruptly when the dirt is removed, while calibration when the dirt is absent means the robot may not start if dirt gets into it later. 
It is worth noting, in light of this example, that GritsBots, Kilobots, and mROBerTO all use nearly-sealed drivetrains, either direct motor drive of the wheels or sealed vibration motors \citep{rubenstein2014kilobot, pickem2015gritsbot, Kim2016mROBerTOAM}. 

An early intent of the author was to have the system learn the control law for each robot through observation of the relationship between the commanded motion of the robot and the resulting motion. 
Due to the overarching concern with human control of a swarm, such online calibration was decided to be out of scope for this work. 
However, a computer-vision-guided calibration technique was used in the mROBerTO swarm robots to compensate for manufacturing differences between robots \citep{Kim2016mROBerTOAM}. 
This approach results in learning bad controls if the system observes the robot during a temporary failure. 
In a system with minimal failures, this problem can be minimized, but as discussed earlier, inexpensive children's toys are not such a system. 

GRITSBots have a calibration step, but the calibration is automated, and is only performed once for each robot, after which it is assumed that the calibration variables are constant.
Calibration can be automated in a homogenous platform with reasonably reliable hardware. 
Non-homogenous platforms require different calibration for different platforms, which reduces the benefit of automation. 
Further, requiring calibration works against transitioning to a new platform by adding an additional hurdle in the form of developing a new calibration method.

The lack of an automated calibration and control method drove the TinyRobo swarm to use more differential drive vehicles, as they struck a balance between the fully holonomic drive used in the SpiderBots, which is expensive but easy to control, and the Ackerman drive used in inexpensive RC cars, which is less expensive, but has more complex control math \citep{lairdspider}.
It also increases the difficulty of using a heterogenous system, and so operates against the advantages of heterogeneity as discussed in Section \ref{section:Why_Heterogeneity_}

%\subsection{April Tag Latency} \label{section:April_Tag_Latency}
%
%April tags have very solid position and orientation tracking, but are computationally intensive to detect and localize in typical webcam images. 
%More tags leads to longer computation time, increasing the latency of the robot control loop. 
%Parallelizing the implementation of the AprilTag library could improve it significantly, but is out of scope for this work. 
%
%One version of the GRITSBots used AprilTags and a standard web cam, resulting in position updates at approximately 10Hz. 
%This update rate is consistent with that observed by the author using the ROS implementation of AprilTags. 
%Decreasing the size of the image in which the tags are detected speeds it up, at the cost of requiring larger tags in order to have them legible in the lower-resolution image. 
%The update rate places an upper bound on the ability of the system to respond to the motion of the robots. 
%To decrease the latency of this sensing, the GRITSBots team considered moving to color blob tags, which can be detected by the onboard vision processor of a Pixy CMUCam5 at up to 50Hz \citep{PickemGrits2014}. 


\subsection{Drive Testing} \label{section:Drive_Testing}

In order to characterize the behavior of the various toy hardware that the TinyRobo platform could be used on, seven TinyRobos were commanded by a program that instructed them to move forward until the AprilTag tracking determined that they had moved a fixed distance, stop, and move again, in a loop. This program was run until the robot failed to move or ran into the walls of the robot arena.

Each robot was run 5 times, starting from the same location each time. The robot was power-cycled between runs. Each run was recorded using rosbag, and the data from the recordings were used to generate visualizations of the robot's commanded trajectory from the motion of the center of the robot's AprilTag.

The robots consisted of three toy tanks from the same manufacturer, a differential-drive Hexbug-brand robot, a differential-drive toy car with large wheels, a hexapod bug-like walker, and an Ackerman-drive toy car. 
The bug-like walker and the Ackerman-drive car both have different turning kinematics from the differential drive tanks, but this test only consists of forward drive.
It would have been preferable to have the test include turning, but control of the yaw rotation of a robot via feedback from the AprilTag system was found to be problematic. 
As the number of pixels that an AprilTag takes up on a camera image decreases, the accuracy of the subpixel estimation used to localize the corners of the tag decreases. 
Each estimation of the tag location from the ROS AprilTag detection node can then differ from previous estimations, and so even when the robot remains still, there is some noise in its detected position. 
Because it was assumed early on that the AprilTag fiducial tracking would be sufficiently accurate, the tradeoffs between camera resolution, tag size, and tracking speed have not been fully explored. 
In some configurations, AprilTags can have very solid position and orientation tracking, but are computationally intensive to detect and localize in typical webcam images. 
More tags also leads to longer computation time, increasing the latency of the robot control loop. 
Parallelizing the implementation of the AprilTag library could improve it significantly, but is out of scope for this work. 
The latency and accuracy problems with AprilTags could be mitigated in a number of ways. 
First, the resolution of the camera can be increased while keeping the tags the same size and the camera the same distance away. Increasing the image size requires more computational power to process the larger image, in order to keep the tag detection framerate high enough to be useful for realtime control. 
Decreasing the size of the image being checked for tags increases the speed of tag detection, at the cost of requiring larger tags in order to have them legible in the lower-resolution image. 
The update rate places an upper bound on the ability of the system to respond to the motion of the robots.
The tag size could be increased, either by moving the tags closer to the camera and increasing their visual size, or by actually making the tags larger. 
Moving the camera closer reduces the useful area of the robot arena, and making the tags larger increases the size and weight of the robots, potentially overbalancing some of the less nimble robots. 
%As was observed in the robot drive tests, the bug and big wheel bases were already prone to tipping over. 

The effect of the noise in subpixel position on the perceived rotation of the tags is larger than its effect on the perceived rotation of the tags, since rotation around the center of the tag does not change the position of its center at all. 
As a consequence, the tags could be detected to be in lateral motion by comparing the displacement of the center of the tag and checking that it was greater than the expected noise, but detection of rotational motion and velocity calculation was sometimes swamped by noise. 

The noise was also determined to be unevenly distributed over the robot arena. Because the arena is wide, a 140$^{\circ}$ wide-angle lens is used to ensure that the entire arena is visible. 
ROS provides tools for removing the distortion inherent in the use of a wide-angle lens, but at the cost of decreased effective resolution at the edges of the image. 
This decrease in resolution results in reduction of subpixel location accuracy, and so increased noise in localization of the tag at the edges of the arena closest to the edges of the camera image. 
%
%One version of the GRITSBots used AprilTags and a standard web cam, resulting in position updates at approximately 10Hz \citep{PickemGrits2014}. 
%This update rate is consistent with that observed by the author using the ROS implementation of AprilTags. 
%To decrease the latency of this sensing, the GRITSBots team considered moving to color blob tags, which can be detected by the onboard vision processor of a Pixy CMUCam5 at up to 50Hz, and for a web-based robot deployment environment, settled on ArUco tags\citep{pickem2017robotarium}. 
%ArUco tags were tested briefly during this project, and rejected as being insufficiently accurate, but they are in use on the GRITSBots, so clearly there is some configuration that allows them to be used to track small robots. 
%MROBerTO used color tracking, with a single RGB LED on top of each robot to identify it, and a green LED on the front of each robot to indicate its orientation. 
%Kilobot robots also include an RGB LED, which may be set to any color by the software of the Kilobot. 
%These LEDs can also provide swarm feedback in a manner similar to the LEDs in McLurkin's Swarmish interface \citep{mclurkin2006speaking}.

Because the desired figure-8 motion could not be performed with the rotational localization noise present in the system, the robots were instead commanded to move forward 0.25m, stop, and repeat that sequence of actions until the program was stopped. 
The program was stopped if the robot was about to run into the wall of the arena, or had entered a state in which it could not move forward anymore. 
Acceleration during the movement phases was managed by increasing the commanded velocity of the robot until the AprilTag tracking detected that the robot had begun moving. 
Once the robot began moving, it was not commanded to change velocity until it had moved at least 0.25m.
In the following descriptions of the motions of the robots, the directions left and right are relative to the direction of travel of the robot. 

\begin{figure}[h]
	\centering
	\begin{subfigure}[t]{0.47\textwidth}
		\includegraphics[width=\textwidth]{../hardwareX_paper/robot_17.png}
		\caption{Motion of toy car based robot, showing long tracks across arena\label{fig:car_motion}}
	\end{subfigure}
	\begin{subfigure}[t]{0.47\textwidth}
		\includegraphics[width=\textwidth]{../hardwareX_paper/robot_1.png}
		\caption{Motion of big wheel robot, showing no track due to the loss of tracking as the robot either did not move, or flipped over\label{fig:big_wheel_motion}}
	\end{subfigure}
	\caption{} %Triggers rendering of Figure label
\end{figure}

The Ackerman-steering toy car (Figure \ref{fig:car_motion}) displayed very consistent mobility. In four out of its five runs, it crossed the arena without incident. 
In one run, it started and went a small distance, but then did not start again. 
However, the Ackerman-steering toy car has a gearbox that permits backdriving, so when commanded to stop, it coasts for a few centimeters. 
The toy tanks use a worm gear in their drivetrains, and so do not permit backdriving of the motors by the vehicle's inertia. 
Instead of coasting, they immediately stop when the motor is commanded to 0 velocity. 

The big wheel robot (Figure \ref{fig:big_wheel_motion}) displayed an unfortunately small range of commanded velocities between those that caused it to begin moving, and those that caused it to flip over on its side.
In three of its five runs, the big wheel accelerated quickly and then flipped during the first motion period. 
In the remaining two runs, it did not move at all, possibly due to gears jamming. 


\begin{figure}[h]
	\begin{subfigure}[t]{0.47\textwidth}
		\includegraphics[width=\textwidth]{../hardwareX_paper/robot_6.png}
		\caption{Motion of 6-wheel bug, showing tight arcs and spirals caused by different motor speeds\label{fig:bug_motion}}
	\end{subfigure}
	\begin{subfigure}[t]{0.47\textwidth}
		\includegraphics[width=\textwidth]{../hardwareX_paper/robot_0.png}
		\caption{Motion of green tank, showing one successful run and four tight spirals due to a stopped track on one side\label{fig:green_tank_motion}}
	\end{subfigure}
	\caption{} %Triggers rendering of Figure label
\end{figure}


The Hexbug-branded 6-wheeled bug (Figure \ref{fig:bug_motion}) displayed an asymmetry in its motor drive speeds. 
For three of its five runs, both motors ran, and the wheels on both sides rotated, but the right side was driven more quickly, and so the robot made a wide arc to the right and hit the wall of the arena behind the starting location. 
In one run, one side did not begin moving at all, causing the robot to rotate rapidly around that side. 
In another run, the robot twitched briefly, remained still, and then accelerated backwards quickly. 
This was likely caused by the gradual incrementing of the forward velocity eventually causing an integer overflow, resulting in a large forward velocity command being interpreted as a large negative velocity.

The green tank, carrying the number 0 AprilTag (Figure \ref{fig:green_tank_motion}), experienced problems with one side of its drive train in three of its five runs.
One tread drive did not move, while the other did, resulting in tight turns to the left in two runs, and to the right in one run. In one of the two remaining runs, the green tank did not move at all. In the second, the tank alternated movement and stopped periods until it reached the other side of the arena, which constituted success on this test. 

\begin{figure}[h]
	\begin{subfigure}[t]{0.47\textwidth}
		\includegraphics[width=\textwidth]{../hardwareX_paper/robot_8.png}
		\caption{Motion of blue tank \#8, showing lack of motion\label{fig:tank_motion_8}}
	\end{subfigure}
	\begin{subfigure}[t]{0.47\textwidth}
		\includegraphics[width=\textwidth]{../hardwareX_paper/robot_18.png}
		\caption{Motion of blue tank \#18, showing arc to the right on most runs, and overflow leading to sudden reverse (light yellow track)\label{fig:tank_motion_18}}
	\end{subfigure}
	\caption{} %Triggers rendering of Figure label
\end{figure}

There are two blue tanks, carrying AprilTags numbered 8 and 18. 
Tank number 8 (Figure \ref{fig:tank_motion_8}) moved slightly on four of its 5 runs, and did not move on one of them. 
At no point did it move the full 0.25m. 
Tank number 18 (Figure \ref{fig:tank_motion_18}) moved much more consistently, but displayed an arc to the right in four of its 5 runs. 
In one run, the tank failed to restart after one of the stop phases, and eventually accelerated quickly in reverse. 
As with the 6-wheeled bug, this was likely the failure of the robot to move leading to a long enough delay that incrementing the commanded acceleration resulted in an integer overflow. 

\begin{figure}[h]
	\centering
	\includegraphics[width=0.47\textwidth]{../hardwareX_paper/robot_5.png}
	\caption{Motion of bug robot, showing tendency towards the left and more erratic path than tracked or wheeled robots}
	\label{fig:bug_robot}
\end{figure}

The single-motor Hexbug-branded blue bug (Figure \ref{fig:bug_robot}) moved consistently, but with a heavy bias towards turning to the left. 
In every run, it started and stopped, but the curvature of the path to the left caused it to run into the left side of the arena. 
During one run, it fell over while moving, due to the somewhat ``bouncy'' nature of the toy's gait.

Some of these problems, such as the coasting of the Ackerman-steering car and the biases towards the left or right of some of the robots when commanded to move straight, can be overcome by software. 
For example, assuming the tags could be used to accurately measure the rotation of the robots, error between the detected rotational velocity and perceived rotational velocity could be accounted for in subsequent motion commands. 
The Ackerman-steering car drift could be reduced by sending a ``brake'' command to the motor driver IC to activate its back-EMF braking mode, rather than simply stopping by reducing the power output to zero. 
However, some of the problems are more difficult to alleviate. 
The tendency of the big wheel robot to flip over could be mitigated by very gradually increasing its speed, at the cost of reducing its control responsiveness, or by increasing the frequency of the feedback loop that checks its speed. 
Perhaps the blue bug would not fall over if its height were reduced, or the battery were placed lower on the body of the robot.
However, these mechanical problems could be more easily avoided by simply not using toys that have them. 

\subsection{3D Printed Robots}

Because the toys were demonstrated to be unreliable as motion platforms, a new design for the motion platform was developed using a 3D printer. 
For this version of the TinyRobos, the motor selected was a commercially available miniature gearmotor. 
The gearmotor was selected because it was cheap, easily available from Amazon.com, and used the same voltage that the TinyRobo drive boards used. 
%\footnote{https://www.amazon.com/Planetary-Reducer-Torque-Gearbox-55RPM-100RPM/dp/B01N4G8DSO/ or https://www.amazon.com/NW-Planetary-Reducer-Torque-Gearbox/dp/B07DCLL82G/}
The motors used were listed on Amazon as ``NW 3pcs 3V Micro Planetary Reducer Motor High Torque DC Motor DIY Robot Gearbox Motor". 
Unfortunately, these particular motors are from an unknown supplier, and as such, may become unavailable. 
However, a number of sellers are offering the same motors, or motors of similar size that could be used as replacements if the 3D print design were altered. 
The motors cost \$9.69 for a package of three, so \$3.23 each or \$6.46 for the pair needed for a robot. 
Given that the toys being used as mobility platforms cost \$8-20, depending on the volume purchased, even purchasing these motors at retail cost was a savings over using toys.
The motor has a planetary gearbox, and a plastic output shaft. 

The 3D printer used is the Monoprice Maker Select Mini (MPMSM), which costs approximately \$200 at the time of this writing.
The MPMSM has a 120mm$^3$ build volume, and can print two robot chassis at the same time, as well as wheels for them.
Wheels were designed to be 3D printed along with the robot, and glued to the output shaft of the motor. 
3D printing provides the flexibility to change the motor mounts of the robot to accommodate various motors. 
The 3D printed components of the robots do not add significantly to the cost of the robots. 
Material for 3D printers is generally sold in spools containing 1kg of filament for \$15-35, although some types of material are much more expensive, depending on the qualities of the material.
The robots described were printed in Hatchbox brand polylactic acid (PLA) filament, which costs \$20/kilogram. 
The resulting prints are approximately 25g of material, and so contain \$0.50 worth of material. 

\begin{figure}
	\centering
	\includegraphics[width=0.8\textwidth]{../3dp_robo.png}
	\caption{3D printed robots, two with LED rings and one with an AprilTag for tracking. The LED rings are intended to display computer-trackable constellations and human-readable information.}
	\label{fig:3dp_robos}
\end{figure}

On top of the 3D printed TinyRobo, posts can hold up an LED ring for color blob tracking, or any kind of trackable code. 
The LED rings have 12 RGB LEDs, which can be lit to indicate the robot's heading to the camera, but can also convey information to the user. 
For example, robots could illuminate LEDs based on the value of a sensor, resulting in a visible map over the swarm of the sensed quantity at the location of each robot. 
They could also illuminate a single LED pointing to the next robot in a chain, and so indicate a path that the user could follow.

Because the 3D printed robots all used the same motors, there was no need to have them use an adaptive velocity control for drive testing as in the drive test for the toys. 
Instead, they could be commanded to move at a fixed velocity, and observed from the camera. 
However, the low output speed of the geared motors meant that the thresholds for detecting that the robot had stopped and started had to be changed. 
Once the program was adjusted so that it could detect the robot's motion or lack thereof correctly, the robots were commanded to move 0.25m, stop, move 0.25m and so forth until they ran into a wall of the arena. 
Once the robot hit a wall, it was reset and run again for 5 attempts. 
Each 3D printed robot was tagged with an April tag, numbered 2, 3, and 5. 

Robot 2 missed the first stop command on two of its runs, but worked normally for subsequent stop commands. 
It stopped and started regularly on the other three runs.
Robot 2 made a long arcing path to the left, indicating that one motor was running slightly faster than the other, or acquiring more traction.
This sort of error can be compensated for by feedback, assuming the robot can detect its own trajectory. 

Robot 3 missed all the stops on its first run and the first stop on its second run.
On its last three runs, it started and stopped as expected. 
Rather than arcing left as with Robot 2, Robot 3 arced to the right. 

Robot 5 also had an arc to the right. For its first run, it missed the last stop, but otherwise functioned correctly. 
On its second run, Robot 5 started and stopped correctly, but hit something on the floor of the arena that caused it to turn more sharply right. 
On its third run, the system incorrectly detected that robot 5 had started after coming to a stop, and so began waiting for it to move 0.25m. 
The robot was not actually moving, so this condition would not have resolved. 
For run four, Robot 5 stopped during its first move and did not restart.
The cause of this error is unknown, but it is possible that it was erroneously detected as having started before it did, and so was not sent motion commands to start. 
On the last run, Robot 5 started and stopped as commanded until it was nearly at the wall, and then one wheel got stuck on a bit of wood on the arena floor, causing the robot to spin around that wheel. 

Overall, the 3D printed robots were more reliable than the majority of the toy bases, in terms of runs completed. 
However, a direct comparison cannot be easily made due to the changes that had to be made to accommodate the new bases. 
The use of a fixed velocity command for the forward motion is particularly problematic, as the gear ratio on the 3D printed robot motors is much higher than the gear ratio on most of the toys. 
As a result, a command that moves the 3D printed robots at 3-4cm/sec will set the toy tank bases moving too quickly for the camera to track them. 
The detection of robot motion also had problems due to the noise in the April tag detection and the low speed of the robots. 
The missed stop commands in the runs of the 3D printed robots were caused by the movement of the robot between successive updates of the tag tracking being sufficiently small that it was less than the expected noise of the AprilTag tracking, and so was considered noise and ignored. 
As a result, the robot was never detected to be moving, and so was never commanded to stop. 
Similarly, when the robot was moving, if its detected motion was sufficiently slow that it would fall below the noise floor, the robot was detected as having stopped, and so would not be sent a command to stop, as the system regarded it as having already stopped. 
Given more accurate tracking, the 3D printed robots could be expected to have fewer false starts and erroneously-detected stops.  

Currently, the 3D printed TinyRobo base suffers from a problem also seen in E-Pucks: the edge of the robot can catch on small obstacles, and so there is still at least a little work that could be done to improve the shape of the 3D base. 
The flexibility of the 3D printed design also opens up possible extension to hardware with e.g. whegs or other unusual forms of mobility. 
These directions have not been explored in any depth.
It is unlikely that any single design is best for all users, but development of various designs using the same control hardware could produce a library of designs with different features to suit different tasks. 


\section{Conclusion and Discussion}

One goal of the personal project that this work developed from was to provide an inexpensive swarm platform that could be used by hobbyists and other non-traditional researchers to work with swarms. 
The required reduction in price was intended to be accomplished by combining the diminishing cost and increasing power of Internet of Things (IoT) networking modules with the ready availability of toys to create a system that lowers the barrier to development of multi-robot systems.

The use of toys as a source of drivetrains for swarm robots proved to be unsatisfactory due to the lack of reliable motion provided by inexpensive toys.
Sensors and networking can be virtualized, as in TinyRobo and GRITSBots, but no amount of clever programming will compensate for balky motion. 
The use of direct drive, as in GRITSBots or mROBerTO, provides a more reliable method of locomotion, because the resulting drive train will be sealed against foreign matter. 
Further, the use of stepper motors in GRITSBots provides some degree of precision in motion control by directing the motor in steps of known resolution, rather than commanding a particular speed. 
If the system requires additional torque, sealed micro gearmotors can provide increased torque (although with reduction in speed), and will be more reliable than adapting a drivetrain from a toy. 
 
Adapting drivetrains from toys also adds the expense of purchasing the toy to the total robot cost.  
This cost is frequently unnecessary, as the chassis of the robot can be constructed from the same printed circuit board (PCB) that the electronics are supported on. 
Over the scales of forces present in tabletop swarms, PCB can be considered completely rigid, and electronics solder provides sufficient mechanical strength for motor mounts. 
The use of custom mechanical assemblies (e.g. in Jasmine micro robots) adds complexity to the build process. 
Where possible, small robots should be designed to use the PCB instead. 
Using children's toys in TinyRobo was intended to avoid the use of custom parts, but brought with it additional problems that were outside of the scope of the work to solve, and could have been avoided with a simpler drivetrain. 
The use of hobby-level or toy-like platforms for swarms is not certain to be a dead-end, as the UB Swarm and the Spider-bots demonstrate, but the cost of sufficiently high-quality toys to produce a sufficiently reliable swarm will drive the overall cost up \citep{patil2016ub, price2014spider}. 
 
Ultimately, the resulting swarm hardware platform does demonstrate that Hypothesis 1, that commodity hardware can enable the construction of an indoor swarm for under \$30 per robot, is correct, assuming parts are purchased in sufficient quantities.
As seen in Table \ref{tab:per_robot_parts_cost}, the cost of an individual TinyRobo control board is just under \$12, if parts are purchased in quantities for 100 robots, which is on par with the cost of Kilobots, which cost \$14 for parts, if the parts are purchased in sufficient quantity to produce 1000 robots. 
The 3D printed chassis consumes approximately \$0.50 worth of material, and the motors add just under \$4 in 100-robot quantities. 
The total price per robot is then \$16.50, well under the \$30 target.
Unfortunately, in single quantities, the cost of the TinyRobo controller climbs to \$23, and the cost of the motors to \$6.96, putting the total cost including 3D printer material at \$30.44.
This is only slightly over the target cost, and could likely be reduced by changing the selection of semiconductors in the power regulation section of the TinyRobo board, or by finding slightly cheaper motors.
 
%This does put the cost of parts for 100 robots at \$1650, or approximately the price of two E-pucks.  
%It may be objected that the swarm researcher must first purchase a \$200 3D printer to achieve this low cost, but for the price of a single E-Puck, the researcher can purchase the printer, and have \$600 left over to put towards robot parts.  
 
%Even without solutions to some of the concerns raised in this chapter, the project to develop the TinyRobo swarm robot platform was something of a success. 
%The TinyRobo controller module can control small toys with up to two motors, and by combining it with a 3D printed base, the hardware can be made more reliable. 
%The total cost of the resulting platform is in line with that of other low-cost swarm robots, and the flexibility of the design allows it to be duplicated relatively easily.
%
%While it did not fully achieve this goal, there are a number of lessons that will be relevant to future developers of multi-robot research platforms. 
%


%Parts count for small robots should be minimized.
%TinyRobo was developed around the ESP-8266 WiFi module because the module includes both a powerful microprocessor and wireless communication. 
%Multi-use devices like the ESP-8266 are becoming cheaper and more competent quickly.
%The processor in mROBerTO is both a 32-bit ARM processor, a low-power Bluetooth transceiver, and an ANT Wireless tranceiver. 
%The VL6180X IC combines an ambient light sensor and a laser-based time-of-flight range sensor in a package measuring approximately 5x8x1mm. 
%Reducing parts count reduces cost and speeds assembly. 

%Swarm robots should have autonomous charging, and have each robot be able to monitor its own battery level. 
%Any operation that requires user intervention with each individual robot will not scale well as the swarm size increases. 
%In the case of the Kilobots, the swarm as a whole can be charged in parallel, rather than requiring the individual robots to be connected with a charger. 
%For the GRITSBbots, an early version used  pins on the front of each robot to connect to a charging station, rather than requiring a human to connect it. 
%Later versions used Qi wireless charging \citep{pickem2017robotarium}.
%The TinyRobo platform could be trivially altered to provide GRITSBots-style self-charging, but does not at present have a method to detect a low battery condition, aside from monitoring the motor drivers for an undervoltage signal. 
%The alternative to this, leaving out self-charging and battery monitoring, results in increased effort by humans to keep the batteries charged and failures due to discharged batteries that are not immediately obvious to the control software. 
%At a more general level, hardware health monitoring is required for systems that intend to detect their own failed components and work around them, which increases robustness in a swarm system. 



%The final version of the TinyRobos incorporates many of these design improvements. 
%The robot chassis is a 3D printed assembly, moved by a sealed drivetrain that directly drives the wheels. 
%Changing the drivetrain eliminated many of the problems with unreliable motion. 
%This is not a perfect solution, as the TinyRobos now have a problem that the E-Pucks also have: they can become stuck on small obstacles. 
%E-Pucks and TinyRobos use the front edge of the robot as a skid, so if there is a small obstacle, the front edge of the robot can push against it rather than riding over it. 
%This problem arose during early testing of the swarm robots, when a piece of sawdust that had been adhered to the swarm arena surface by paint was able to stop E-Pucks or deflect their motion. 

%\section{Swarm Simulation} \label{section:Swarm_Simulation}
%
%Because the sensors in the TinyRobo system are simulated, it may appear that the development of the UI and translation components could have been served as well by a simulated swarm as by a real one. 
%This is not the case. 
%In 1992, Rodney Brooks pointed out that ``there is a vast difference (which is not appreciated by people who have not used real robots) between simulated robots and physical robots and their dynamics of interaction with the environment'' \citep{brooks1992artificial}. 
%The gap between real-world sensing and actuation and their simulated counterparts means that developers working solely in simulation are tempted to either invest effort in solving problems that do not occur in the real world, or fail to solve problems that do arise in the real world but are left out of simulation. 
%One of Brooks' examples, the directional nap of carpet having an effect on robot odometry that varied with direction of travel, is still not handled by modern physics simulators. 
%Even if it were handled, the attempt would over-complicate simulation configurations due to the variety of types of carpet available. 
%
%Modern simulators have vastly more powerful computers behind them than they did in 1992, and the development of video games with realistic physics models has contributed greatly to the variety of physics models available in modern robot simulators. 
%As of 2015, however, Brooks' point still holds, and the dynamics of robots interaction with the environment are still not perfectly simulated \citep{erez2015simulation}.
%Generally, aspects of physics are elided or approximated from gaming physics engines, in favor of rapid simulation which is visually plausible. 
%In particular, contact dynamics simulations are NP-hard, and so are approximated in the interests of speed. 
%The PhysiX and Havoc physics engines also do not implement Coriolis forces at all.
%These sorts of omissions are mostly harmless, but evolutionary techniques for developing controllers for robots or simulated creatures can exploit them to develop control schemes that cannot work under real physics \citep{Brooks2000}. 
%
%In the real world, sensors are also uncertain. 
%Simulated sensors can provide a level of accuracy and certainty about that accuracy that is not available to real sensors operating with the sensor noise that affects real-world systems. 
%As the TinyRobo system simulates its laser and distance sensors, it may appear that it falls into this trap of simulation, and provides sensors that are ``too good to be true''. 
%However, the TinyRobo sensors are physical sensors, they're just not the same physical sensors that they are presented as, in much the way that a sonar sensor is not a ruler, but rather measures time and sound and presents it as distance. 
%Similarly, the arena camera is not a robot-mounted laser scanner, but it is still a physical optical sensor being used to measure distance
%\todo{test noise of camera based fake laser vs real lasers, characterize difference between multiple sensor readings of same thing}
%Adding noise to simulated sensors can also move them more towards the sensors available on real robots, but as Brook's carpet nap example indicates, the type of noise may be more complex than a simple normal distribution about the ground truth value. 
%With evolutionary approaches, the system can even come to depend on the noise, and fail when it is given real sensors that are less noisy then their simulated counterparts \citep{jakobi1995noise} 
%%Tim Smithers, "On Why Better Robots Make it Harder"
%	The idea that the variation in real robot behavior will go away if the robots are "better" as a justification for simulation
%	Better-made components allowing hunting where worse ones provided damping (steam engine governor example)
%	Don't view robots as measuring with sensors and reasoning about results
%		Sensors act as filters whose output drives internal robot dynamics
%
%An Overview about Simulation and Emulation in Robotics
%Michael Reckhaus, Nico Hochgeschwender, Jan Paulus, Azamat Shakhimardanov and Gerhard K. Kraetzschmar
%	Simulation went out of fashion when people started getting real robots
%	Computers got better and games drive for realism pushed development
%	Proposes a lot of use cases and reasons to use simulation
%		Which are interestingly orthogonal to the reasons not to use it
%			One doesn't counter the other, it's just "here are the good parts, here are the bad parts"
%
%Simulation in robotics
%Leon Zlajpah
%	Overview of tools



